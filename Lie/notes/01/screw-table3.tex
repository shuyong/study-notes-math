\documentclass[border=2pt]{standalone}
\usepackage{amsmath}
\usepackage{amssymb}
\usepackage[UTF8]{ctex}

\begin{document}

\begin{tabular}{|c|c|c|c|}
\hline   &  几何意义  &  瞬时运动学  &  静力学 \\
\hline  主部  &  \begin{tabular}{@{}c@{}}
直线\\
 (姿态向量)
\end{tabular}  &  \begin{tabular}{@{}c@{}}
具有幅度的轴线\\
\ensuremath{\boldsymbol{\omega}=\dot{\theta}\hat{\mathbf{s}}}
\end{tabular}  &  \begin{tabular}{@{}c@{}}
力的作用线\\
\ensuremath{\mathbf{f}=f\hat{\mathbf{s}}}
\end{tabular} \\
\hline  副部  &  \begin{tabular}{@{}c@{}}
关于原点的矢距\\
 (直线的位置)
\end{tabular}  &  \begin{tabular}{@{}c@{}}
平移速度\\
\ensuremath{\mathbf{v}=\dot{\theta}\mathbf{s}_{0}}
\end{tabular}  &  \begin{tabular}{@{}c@{}}
力矩\\
\ensuremath{\mathbf{m}=f\mathbf{s}_{0}}
\end{tabular} \\
\hline \multicolumn{4}{|c|}{旋距的影响 }\\
\hline  非零旋距  &  \begin{tabular}{@{}c@{}}
旋量\\
\ensuremath{\left[\begin{array}{c}
\hat{\mathbf{s}}\\
\mathbf{r}\times\hat{\mathbf{s}}+h\,\hat{\mathbf{s}}
\end{array}\right]}
\end{tabular}  &  \begin{tabular}{@{}c@{}}
旋转 + 平移\\
\ensuremath{\dot{\theta}\left[\begin{array}{c}
\hat{\mathbf{s}}\\
\mathbf{r}\times\hat{\mathbf{s}}+h\,\hat{\mathbf{s}}
\end{array}\right]}
\end{tabular}  &  \begin{tabular}{@{}c@{}}
纯力 + 力偶\\
\ensuremath{f\left[\begin{array}{c}
\hat{\mathbf{s}}\\
\mathbf{r}\times\hat{\mathbf{s}}+h\,\hat{\mathbf{s}}
\end{array}\right]}
\end{tabular} \\
\hline  零旋距  &  \begin{tabular}{@{}c@{}}
线向量\\
\ensuremath{\left[\begin{array}{c}
\hat{\mathbf{s}}\\
\mathbf{r}\times\hat{\mathbf{s}}
\end{array}\right]}
\end{tabular}  &  \begin{tabular}{@{}c@{}}
纯旋转\\
\ensuremath{\dot{\theta}\left[\begin{array}{c}
\hat{\mathbf{s}}\\
\mathbf{r}\times\hat{\mathbf{s}}
\end{array}\right]}
\end{tabular}  &  \begin{tabular}{@{}c@{}}
纯力\\
\ensuremath{f\left[\begin{array}{c}
\hat{\mathbf{s}}\\
\mathbf{r}\times\hat{\mathbf{s}}
\end{array}\right]}
\end{tabular} \\
\hline  无穷大旋距  &  \begin{tabular}{@{}c@{}}
偶量\\
\ensuremath{\left[\begin{array}{c}
\mathbf{0}\\
h\,\hat{\mathbf{s}}
\end{array}\right]}
\end{tabular}  &  \begin{tabular}{@{}c@{}}
纯平移\\
\ensuremath{\dot{\theta}\left[\begin{array}{c}
\mathbf{0}\\
\hat{\mathbf{s}}
\end{array}\right]}
\end{tabular}  &  \begin{tabular}{@{}c@{}}
纯力偶\\
\ensuremath{c\left[\begin{array}{c}
\mathbf{0}\\
\hat{\mathbf{s}}
\end{array}\right]}
\end{tabular} \\
\hline
\end{tabular}

\end{document}
