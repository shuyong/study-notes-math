%In case you encounter the following error:
%Error 1010 The PDF file may be corrupt (unable to open PDF file) OR
%Error 1000 An error occurred while parsing a contents stream. Unable to analyze the PDF file.
%This is a known problem with pdfLaTeX conversion filter. The file cannot be opened with acrobat reader
%Please use one of the alternatives below to circumvent this error by uncommenting one or the other
%\pdfobjcompresslevel=0
%\pdfminorversion=4

% See the \addtolength command later in the file to balance the column lengths
% on the last page of the document

% The following packages can be found on http:\\www.ctan.org
\usepackage{graphics} % for pdf, bitmapped graphics files
%\usepackage{epsfig} % for postscript graphics files
%\usepackage{mathptmx} % assumes new font selection scheme installed
%\usepackage{times} % assumes new font selection scheme installed
\usepackage{amssymb}  % assumes amsmath package installed
\usepackage{textglos}
%\usepackage[utf8]{inputenc} % allow utf-8 input
%\usepackage[T1]{fontenc}    % use 8-bit T1 fonts
\usepackage{fontspec}
%\usepackage[xetex]{hyperref}       % hyperlinks
\usepackage{url}            % simple URL typesetting
\PassOptionsToPackage{hyphens}{url}\usepackage[xetex]{hyperref}
%\usepackage[hyphenbreaks]{breakurl}
\usepackage{booktabs}       % professional-quality tables
\usepackage{amsfonts}       % blackboard math symbols
\usepackage{nicefrac}       % compact symbols for 1/2, etc.
\usepackage{microtype}      % microtypography
\usepackage{lipsum}
\usepackage{tabularx} % extra features for tabular environment
\usepackage{amsmath}  % improve math presentation
\usepackage[xetex]{graphicx} % takes care of graphic including machinery
\usepackage{dblfloatfix}
\usepackage[margin=1in,letterpaper]{geometry} % decreases margins
\usepackage{cite} % takes care of citations
%\usepackage[final]{hyperref} % adds hyper links inside the generated pdf file
\usepackage{titlesec}
%\usepackage{caption}
\usepackage{mathtools, amssymb, nccmath}
\usepackage{wasysym}
\usepackage{algorithm}
\usepackage{algorithmic}
\usepackage{multicol}
\let\labelindent\undefined
\usepackage{enumitem}
\usepackage{float}
\usepackage{mathtools}
%\usepackage{subcaption}
\usepackage{amsmath,amsfonts,amssymb}
%\usepackage{algpseudocode}
\usepackage{cmll}
%\usepackage[toc,page]{appendix}
\usepackage{tensor}
%\usepackage[compact]{titlesec}       
\usepackage{booktabs}
\usepackage{tabularx}
\usepackage{makecell}
% you need this package
\newlength{\tempdima}
\usepackage{tablefootnote}
\usepackage{threeparttable}
\usepackage{multirow}

%\usepackage{xurl}


% added a package here for reducing gap between figure and caption- it worked
\usepackage[font=small,skip=5pt]{caption}

% color package
\usepackage[dvipsnames]{xcolor}

\newcolumntype{Z}{ >{\centering\arraybackslash}X }

\renewcommand\theadfont{\bfseries}

\newtheorem{definition}{Definition}
\newtheorem{corollary}{Corollary}
\newtheorem{lemma}{Lemma}
\newtheorem{theorem}{Theorem}
\newtheorem{remark}{Remark}

\newtheorem{property}{Property}

\newtheorem{note}{Note}

\newcommand{\norm}[1]{\left\lVert#1\right\rVert}

\newcommand{\crff}{\,\overline{\!\times\!}{}^{\,*}}
\newcommand{\vJhat}{\hat{\vJ}}
\newcommand{\vJhatDot}{\dot{\hat{\vJ}}}
\newcommand{\vqh}{\hat{\vq}}
\newcommand{\vqhd}{\dot{\hat{\vq}}}
\newcommand{\vJhT}{\vJhat{\vphantom{\vJ}}\T\!}
\newcommand{\vChat}{\hat{\vC}}
\newcommand{\vHhat}{\hat{\vH}}
\newcommand{\vHhatDot}{\dot{\hat{\vH}}}

\newcommand{\pq}[2]{ \frac{\partial q_{#1}}{\partial \hat{q}_{#2}}}
\newcommand{\ph}[2]{ \frac{\partial H_{#1}}{\partial q_{#2}}}
\newcommand{\phh}[2]{ \frac{\partial \hat{H}_{#1}}{\partial \hat{q}_{#2}}}
\newcommand{\pj}[2]{ \frac{\partial \hat{J}_{#1}}{\partial \hat{q}_{#2}}}
\newcommand{\ppq}[3]{\frac{\partial^2 q_{#1}}{\partial \hat{q}_{#2} \partial \hat{q}_{#3}}} 

\newcommand{\Motion}{\mathcal{M}}
\newcommand{\Force}{\mathcal{F}}

% Partial C new commands
\newcommand{\pC}[3]{\frac{\partial \mathbf{C}_{#1,#2} }{\partial q_{#3} }}

% Partial M FO new commands
\newcommand{\pM}[3]{\frac{\partial \mathbf{M}_{#1,#2} }{\partial q_{#3} }}

% SO partial M new command
\newcommand{\pMSO}[4]{\frac{\partial^{2} \mathbf{M}_{#1,#2} }{\partial q_{#3}\partial q_{#4} }}

% Partial g FO new commands
\newcommand{\pgFO}[2]{\frac{\partial \mathbf{g}_{#1} }{\partial q_{#2} }}

% Partial g SO new commands
\newcommand{\pgSO}[3]{\frac{\partial^{2} \mathbf{g}_{#1} }{\partial q_{#2}\partial q_{#3} }}

% Partial Cqdot new commands
\newcommand{\pCqdot}[2]{\frac{\partial [\mathbf{C}\dot{q} ]_{#1} }{\partial q_{#2} }}

% Partial Cqdot with respect to qdot new commands
\newcommand{\pCqdotdq}[2]{\frac{\partial [\mathbf{C}\dot{q} ]_{#1} }{\partial \dot{q}_{#2} }}

% Partial Mqdd FO new commands
\newcommand{\pMqdd}[2]{\frac{\partial [\mathbf{M}\ddot{q} ]_{#1} }{\partial q_{#2} }}

% Partial Mqdd SO new commands
\newcommand{\pMqddSO}[3]{\frac{\partial^{2} [\mathbf{M}\ddot{q} ]_{#3}  }{\partial q_{#1}\partial q_{#2} }}

% Partial tau with respect to q new commands FO
\newcommand{\tauFOq}[2]{\frac{\partial \tau_{#1} }{\partial {q}_{#2} }}

% Partial tau with respect to qd new commands FO
\newcommand{\tauFOqd}[2]{\frac{\partial \tau_{#1} }{\partial \dot{q}_{#2} }}

% Partial tau with respect to q new commands SO
\newcommand{\tauSOq}[3]{\frac{\partial^2 [\bar{\tau} ]_{#1} }{\partial \bar{q}_{#2} \partial \bar{q}_{#3} }}

% Partial tau with respect to q new commands MSO
\newcommand{\tauMSOq}[3]{\frac{\partial^2 [\bar{\tau} ]_{#1} }{\partial \bar{q}_{#2} \partial \dot{\bar{q}}_{#3} }}
% Partial Cqd SO new commands
\newcommand{\pCqdSO}[3]{\frac{\partial^{2} [\mathbf{C}\ddot{q} ]_{#3}  }{\partial q_{#1}\partial q_{#2} }}

% Partial Cqd MSO new commands
\newcommand{\pCqdMSO}[3]{\frac{\partial^{2} [\mathbf{C}\ddot{q} ]_{#1}  }{\partial q_{#2}\partial \dot{q}_{#3} }}

\newcommand{\rel}{\alpha}
\newcommand{\vqddd}{\dddot{\vq}}

\newcommand{\vPhiDdot}{\dot{\vPhi}}
\newcommand{\mysp}{.7ex}
\newcommand{\vPhiDot}{\dot{\vPhi}}
\newcommand{\rightspace}{185px}
\newcommand{\jerk}{\dot{\ba}}
\newcommand{\BF}{\mathbf F}
\newcommand{\bw}{\mathbf w}

\newcommand{\bwd}{\dot{\mathbf w}}

\newcommand{\Mot}{\mathcal{M}}
\newcommand{\For}{\mathcal{F}}
\newcommand{\M}{\mathcal{M}}
\newcommand{\F}{\mathcal{F}}
\newcommand{\I}{\mathcal{I}}
\newcommand{\defn}{:=}
\newcommand{\crf}{\times^*}
\newcommand{\crm}{\times}
\newcommand{\T}{^\top}

\newcommand{\before}[1]{\,\preceq\, #1}
\newcommand{\after}[1]{\,\succeq\, #1}
\renewcommand{\rel}[1]{\,\sim\, #1}

\newcommand{\sumbefore}[2]{#1 \before{#2}}
\newcommand{\sumafter}[2]{#1\after{#2}}
\newcommand{\sumrel}[2]{#1\rel{#2}}

\newcommand{\w}{\mathbf{w}}

\newcommand{\Rn}{\mathbb{R}^n}
\newcommand{\Rnn}{\mathbb{R}^{n\times n}}
\newcommand{\R}{\mathbb{R}}
\newcommand{\C}{\vC}
\renewcommand{\H}{\vH}

\newcommand{\Hd}{\dot{\H}}
\newcommand{\Cc}{\overline{\C}}
\newcommand{\XM}[2]{ \tensor[^{#1}]{\mathbf{X}}{_{#2}}}

%%%%%%%%%%%%%%%%%%%%%%%%%%%%%%%%%%%%%%%%%%%%%%%%%
%%%%%%%%%%%%%%%%%%%%%%%%%%%%%%%%%%%%%%%%%%%%%%%%%
% Single point of change for vector formatting
\newcommand{\rmvec}[1]{\boldsymbol{#1}}
\newcommand{\greekvec}[1]{\boldsymbol{#1}}
%%%%%%%%%%%%%%%%%%%%%%%%%%%%%%%%%%%%%%%%%%%%%%%%%
%%%%%%%%%%%%%%%%%%%%%%%%%%%%%%%%%%%%%%%%%%%%%%%%%

\newcommand{\B}{\boldsymbol{B}{}}
\renewcommand{\C}{\boldsymbol{C}}
\renewcommand{\M}{\boldsymbol{M}}

\renewcommand{\I}{\boldsymbol{I}{}}
\newcommand{\g}{\rmvec{g}}
\newcommand{\q}{\rmvec{q}}
\renewcommand{\b}{\rmvec{b}}
\renewcommand{\u}{\rmvec{u}}

\newcommand{\m}{\rmvec{m}}
\newcommand{\f}{\rmvec{f}}
\newcommand{\p}{\rmvec{p}}

\renewcommand{\v}{\rmvec{v}}
\newcommand{\vJ}[1]{\v_{J_{#1}}}
\newcommand{\aJ}[1]{\a_{J_{#1}}}
\renewcommand{\t}{\rmvec{t}}

\renewcommand{\a}{\rmvec{a}}
\newcommand{\ag}[1]{\a_{g_{#1}}}

\newcommand{\qd}{\dot{\q}}
\newcommand{\qdd}{\ddot{\q}}


\newcommand{\Psibar}{\greekvec{\Psi}}
\newcommand{\Psibardot}{\,\dot{\!\Psibar}}
\newcommand{\Psibarddot}{\,\ddot{\!\Psibar}}

\newcommand{\psibar}{\greekvec{\psi}}
\newcommand{\psibardot}{\,\dot{\!\psibar}}
\newcommand{\psibarddot}{\,\ddot{\!\psibar}}

\newcommand{\taubar}{\greekvec{\tau}}
\newcommand{\gammabar}{\greekvec{\gamma}}

\newcommand{\etabar}{\greekvec{\eta}}
\newcommand{\xibar}{\greekvec{\xi}}
\newcommand{\zetabar}{\greekvec{\zeta}}

\newcommand{\red}[1]{{\color{black}#1}}
\newcommand{\blue}[1]{{\color{black}#1}}

%\newcommand{\phibar}{\greekvec{\phi}}
%\newcommand{\Phibar}{\greekvec{\Phi}}
%\newcommand{\phibardot}{\,\dot{\!\phibar}}
%\newcommand{\Phibardot}{\,\dot{\!\Phibar}}
%\newcommand{\phibarddot}{\,\ddot{\!\phibar}}
%\newcommand{\Phibarddot}{\,\ddot{\!\Phibar}}

% new notation-with phi being replaced by S

\newcommand{\phibar}{\boldsymbol{s}}
\newcommand{\Phibar}{\boldsymbol{S}}
\newcommand{\phibardot}{\,\dot{\!\phibar}}
\newcommand{\Phibardot}{\,\dot{\!\Phibar}}
\newcommand{\phibarddot}{\,\ddot{\!\phibar}}
\newcommand{\Phibarddot}{\,\ddot{\!\Phibar}}


\def\autodiff{AD}
\newcommand{\subtree}{\nu}
\newcommand{\subtreeb}{\overline{\nu}}
