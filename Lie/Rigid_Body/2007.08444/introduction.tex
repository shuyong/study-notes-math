
\section{\normalfont\bfseries 简介}

在过去的三十年里,已经有大量的论文涉及机器人建模的不同表示。在Featherstone \cite{Featherstone2008Book,Featherstone2010,Featherstone2010a}、McCarthy \cite{McCarthyBook1990,Dooley1991,Perez2004}、Selig \cite{Selig2004,Selig2005},以及Bayro-Corrochano \cite{Selig2010},还有许多其他人的工作中都可以找到声名狼藉的例子。

进行这种调查的原因之一是机器人系统的复杂性远远超出了机械装置本身的复杂性。一个典型的机器人系统涉及运动/力/阻抗控制、路径规划、任务规划,以及更多的更高级层面的东西。因此,对于机器人建模非常有用的表示方法,例如齐次变换矩阵,在执行位姿控制或阻抗控制时不一定容易使用,例如文献 \cite{Yuan1988}。这正是为什么通常使用齐次变换矩阵获得机器人运动学,然后间接找到几何雅可比矩阵,最后使用四元数和位置向量在任务空间中执行位姿控制的原因 \cite{Xian2004}。使用上述策略有几个缺点。不同表示方法的混合不必要地使整体表示复杂化,并且这些不同表示方法之间的映射通常会引入数学伪影,例如算法奇异性和不连续性。

相反,对偶四元数代数的元素具有很强的几何意义,如旋量理论中的元素,并且也表示为单个元素中的耦合实体。在运动学中,该表示方法已被广泛地探索,以获得机器人的运动学和微分运动学 \cite{Perez2004,Adorno2011,Gouasmi2012,Cohen2016,Ozgur2016,Kong2017,Dantam2020}。此外,在最近的工作中,对偶四元数被用于执行导纳控制 \cite{Fonseca2020},这是物理上的人机交互的基础;约束运动控制 \cite{Marinho2019,Quiroz-Omana2019},它考虑了工作空间施加的几何约束;混合控制,它考虑了刚性运动空间的拓扑结构 \cite{Kussaba2017} 和最优控制,其对于机器人机械臂使用线性二次型最优跟踪控制器 \cite{Marinho2015};分布式位姿形成控制 \cite{Savino2020} 和协同操作 \cite{Adorno2010,Figueredo2014IROS},包括涉及人-机器人协作的操作 \cite{Adorno2015};以及定义高级几何任务 \cite{Lana2015}。

此外,诸如单位对偶四元数和纯对偶四元数等元素,当配备有标准的乘法和加法运算时,形成具有相关李代数的李群。因此,基于对偶四元数代数的阐述方式提供了旋量理论的几何直觉,李代数的严格性以及如在空间代数中 \cite{Featherstone2008Book} 的动力学模型的简单代数处理,通常减少了对机械装置进行广泛几何分析的必要性,这与基于旋量理论的矩阵表示 \cite{Huang2015,Renda2017} 的方法形成鲜明对比。

在过去的几十年里,一些工作已经使用对偶四元数代数来描述刚体动力学 \cite{Yang1964,Yang1966,Yang1967,Yang1971},尽管不一定能创建出一个用于多体系统分析的一般形式。在寻求这种形式的工作中,大多数是基于三维对偶向量,并需要将其映射到更高维向量以获取系统动力学方程
\cite{Pennock1983,Dooley1991,Shoham1993,Valverde2018a},因此失去了仅基于对偶四元数代数的分析的优雅性和紧凑性,有时,会导致符号的滥用
\cite{Dooley1991},或要求对向量进行人为交换 \cite{Valverde2018},以处理表示的混合。其他工作侧重于对偶四元数的传播 \cite{Hachicho2000} 或基于对偶四元数代数的算法的计算方面 \cite{MirandadeFarias2019},而不是在处理更复杂的机器人(如非完整移动机械臂)和更一般类型的关节(如螺旋、圆柱、$6$自由度等等)时提供代数的和几何的直觉。

综上所述,由于目前还没有基于对偶四元数代数的方法能够充分涵盖移动机械臂的动力学模型和一般类型的关节,因此在使用基于对偶四元数的高级算法以连接到低级动力学模型时,仍然存在理论鸿沟,造成不必要的中间映射。本文的目的是通过使用对偶四元数代数,提议一种适应具有任意关节的移动机械臂的动力学模型,以填补这一空白。

\subsection{\normalfont\bfseries 贡献声明}

本文提出了两种使用对偶四元数代数求解串联机械臂动力学方程的方法。第一种方法基于牛顿-欧拉递归公式,而第二种方法应用高斯最小约束原理,以获得具有非完整约束的串联移动机械臂的动力学模型。本文对最新技术的贡献如下:
\begin{enumerate}
\item 对于动力学模型,使用对偶四元数代数和牛顿-欧拉形式,以获得移动机械臂的递归方程的一个系统过程,其具有与连杆数量相关的线性成本。这种
方法简化了经典的计算过程,消除了详尽的几何分析的必要性,因为动力旋量和运动旋量是通过高级代数运算传播的。与以前的工作相比,我们的方法更通用,因为它适用于任意类型的关节,并且对于运动旋量的传播我们不强加任意特定的参数化约定;
\item 基于高斯最小约束原理(Gauss's Principle of Least Constraint, GPLC)和对偶四元数代数的串联机械臂动力学模型的一个封闭形式。我们在GPLC公式中施加附加的约束以建模非完整移动机械臂。我们应用Udwadia-Kalaba \cite{FirdausE.UdwadiaandRobertE.Kalaba1992}提出的基本方程,其采用比拉格朗日乘子更简单的方法,尽管它们是等效的,以强制执行等式约束。此外,我们还提出了与惯量矩阵和科里奥利矩阵相关的斜对称性质,这在设计基于无源性的控制器时至关重要。最后,我们使用基于对偶四元数代数的公式,展示了高斯最小约束原理与Gibbs-Appell和Kane方程的联系。
\end{enumerate}
%
我们使用三种不同的机器人在模拟中验证了所提议的算法:一个固定基座 $50$ 自由度串联机械臂、一个 $9$ 自由度完整移动机械臂,以及一个 $8$ 自由度非完整移动机械臂。此外,我们将我们的结果与真实的模拟器提供的结果进行了比较,并与最新技术的实现进行了比较。此外,我们给出了所提议方法的计算成本。

本文的组织结构如下:第~\ref{sec:Mathematical-Preliminaries} 节简要介绍了对偶四元代数的数学背景;第~\ref{sec:Newton-Euler-Model} 节介绍了基于对偶四元代数的一般牛顿-欧拉公式,而第~\ref{sec:Gauss-Principle} 节介绍了基于高斯最小约束原理的对偶四元代数公式;第~\ref{sec:Results} 节通过模拟以及它们的计算成本展现了所提议的两种方法的可用性;最后,第~\ref{sec:Conclusions} 节给出了最后的结论,并指出了进一步的研究方向。
