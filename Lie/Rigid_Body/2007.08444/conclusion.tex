
\section{\normalfont\bfseries 结论\label{sec:Conclusions}}

本项工作对于基于对偶四元数代数的移动机械臂动力学模型的表示方法,提出了两种策略。第一种是基于递归牛顿-欧拉公式,并使用运动旋量和动力旋量代替自由向量。这种表示方法消除了对运动链进行详尽几何分析的必要性,因为动力旋量和运动旋量是通过高级代数运算传播的。此外,我们的公式适用于任意类型的关节,因为它考虑了任意运动旋量。因此,我们的策略比Miranda等人\cite{MirandadeFarias2019Journal}的工作更一般,他们只考虑具有旋转关节的机械臂。

第二种方法基于高斯最小约束原理,并且它也被制定基于对偶四元数代数所表示的运动旋量和动力旋量的矩阵形式,该策略允许在优化公式中直接加入等式约束。

所提议方法与经典方法的成本比较表明,在乘法和加法次数方面,对偶四元数的使用不会显著增加牛顿-欧拉形式的成本,因为该算法对运动链中的刚体数量具有线性复杂度。然而,使用高斯最小约束原理和对偶四元数代数得到欧拉-拉格朗日模型的成本高于在文献中发现的最佳经典欧拉-拉格朗日递推解。尽管如此,我们的方法远比这些经典的方法更通用。另外,我们没有做出任何努力来优化我们的实现(如果有的话),因为我们目前更感兴趣的是使用对偶四元数代数进行动态建模的理论方面,而不是确保计算效率。在我们目前的MATLAB实现中,dqNE和dqGP平均分别需要 23.17s 和 8.73s 以产生$50$自由度机械臂机器人的关节加速度。在C++实现中,这些值预计将分别减少到 99 ms 和 37 ms 左右 \cite{AdornoDQRobotics2020}。

通过牛顿-欧拉形式获得欧拉-拉格朗日模型需要多次执行该算法。一次执行以获得重力向量,一次执行以获得科里奥利项和离心项的向量,并且对于惯量矩阵 $\mymatrix M$ 的每一行都执行一次。对于一个$50$自由度机械臂的机器人,每个模拟步骤执行 $52$ 次dqNE。然而,对于控制应用,我们通常是对寻找关节扭矩感兴趣,它只需要执行一次dqNE以产生每个控制输入。因此,使用dqNE计算$50$自由度机械臂机器人的关节扭矩,在C++实现中,执行时间预计将减少 $52$ 倍,从大约 99~ms 减少到 1.9~ms。

最后,对于三种不同机器人,我们通过提议的策略获取的关节加速度,与从 ${\text{V-REP PRO EDU V3.6.2}}$ (一个真实的模拟器)中获取的值进行了比较。结果表明,我们所有的方法对于固定基座串联机械臂和移动机械臂都是精确的。

未来的工作重点是将对偶四元数牛顿-欧拉算法推广到非串联多体系统(如仿人系统),以及动力旋量控制策略中。对于使用高斯最小约束原理和对偶四元数代数获得的欧拉-拉格朗日模型,未来的工作将集中于在优化表示方法中运用不等式约束。
