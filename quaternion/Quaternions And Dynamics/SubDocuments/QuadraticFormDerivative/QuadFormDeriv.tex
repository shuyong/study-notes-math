%\documentclass[a4paper]{article}
%\usepackage{graphicx}
%\usepackage{eepic}
%\usepackage{pst-all}
%\usepackage{amsmath, amsthm,amsfonts, stmaryrd}
%%\usepackage[latin1]{inputenc}
%\usepackage[cyr]{aeguill}
%%\usepackage[francais]{babel}
%\usepackage{listings}
%\usepackage{psfrag}
%\usepackage{epsfig}

%%\usepackage{pdftricks}

%%\usepackage{bookman}
%%\usepackage{helvet}
%%\usepackage{newcent}
%%\usepackage{aeguill}

%

%\DeclareMathOperator{\sinc}{sinc}

%
%\renewcommand{\v}{\vec} 
%\renewcommand{\c}{\times} 
%\renewcommand{\o}{\circ}
%\renewcommand{\b}{\mathbf} 

%\newcommand{\w}{\vec{\omega}} 
%\newcommand{\IM}{\vec{\textrm{Im}}} 
%\newcommand{\RE}{\textrm{Re}} 

%\renewcommand{\wp}{\v{\omega}'}
%\newcommand{\dwp}{\dot{\v{\omega}}'}
%\newcommand{\wgpp}{\v{\omega}_g''}
%\newcommand{\dwgpp}{\dot{\v{\omega}}_g''}

%\newcommand{\bq}{\b{q}}
%\newcommand{\dbq}{\dot{\b{q}}}
%\newcommand{\half}{\frac{1}{2}}

%\newcommand{\dG}{\dot{G}}

%
%%\lstset{language=MATLAB}
%%\lstset{basicstyle=\footnotesize,frame=single}

%%\newenvironment{nom}[nb_arg]{avant}{aprs}

%\newcommand{\projectName}{\emph{hydroMEX3 }} %current project name

%%MATLAB code listings:
%\newcommand{\MATcode}
%{
%	\lstset{language=MATLAB}
%	\lstset{basicstyle=\footnotesize,frame=single,showstringspaces=false}
%}
%%MATLAB code line:
%\newcommand{\MATline}
%{
%	\lstset{language=MATLAB}
%	\lstset{basicstyle=\normalsize,frame=none,showstringspaces=false}
%}
%%C code listings:
%\newcommand{\Ccode}
%{
%	\lstset{language=C}
%	\lstset{basicstyle=\footnotesize,frame=single,showstringspaces=false}
%}
%%C code line:
%\newcommand{\Cline}
%{
%	\lstset{language=C}
%	\lstset{basicstyle=\normalsize,frame=none,showstringspaces=false}
%}

%
%\newcommand{\vbeg}{\left( \begin{array}{c} } 
%\newcommand{\vend}{ \end{array} \right)} 

%\definecolor{cBlue}{rgb}{0.0,0.0,1.0}
%\definecolor{cViol}{rgb}{0.4,0.0,0.6}
%\definecolor{cRed}{rgb}{0.7,0.0,0.3}
%\definecolor{cGreen}{rgb}{0.3,0.5,0.0}
%\definecolor{cGrey}{rgb}{0.2,0.2,0.2}
%\definecolor{cGreyy}{rgb}{0.5,0.5,0.5}
%\definecolor{cBlack}{rgb}{0.0,0.0,0.0}

%\newcommand{\Blue}{\color{cBlue}} 
%\newcommand{\Viol}{\color{cViol}} 
%\newcommand{\Red}{\color{cRed}} 
%\newcommand{\Green}{\color{cGreen}} 
%\newcommand{\Grey}{\color{cGrey}} 
%\newcommand{\Greyy}{\color{cGreyy}} 
%\newcommand{\Black}{\color{cBlack}} 

%
%\begin{document}

%%titre etc...:
%\title{Derivatives and Quaternions}
%\author{Basile Graf }
%\date{\today}
%\maketitle







\section{导数和四元数}

\subsection{四元数的二次型导数}
\label{dQuadForm_dq}

为了能够由非惯性四元数模型的 $\bq$ 分量导出拉格朗日方程,需要执行如下操作

\begin{equation*}
\frac{\partial (\v{v}^TR\v{w})}{\partial \bq} \ \ \ ,  \qquad \qquad   \frac{\partial (\v{v}^TR^T\v{w})}{\partial \bq}
\end{equation*}

并且还有

\begin{equation*}
\frac{\partial (\v{u}^TRJR^T\v{u})}{\partial \bq}.
\end{equation*}

但是因为 $R=EG^T$ 和

\begin{equation*}
E =
\left( \begin{array}{cccc}
-q_1 &  q_0 & -q_3 & q_2  \\
-q_2 &  q_3 & q_0  & -q_1 \\
-q_3 & -q_2 & q_1  & q_0  \\
\end{array} \right)
\qquad \qquad
G = 
\left( \begin{array}{cccc}
-q_1 &  q_0 & q_3 & -q_2 \\
-q_2 & -q_3 & q_0  & q_1 \\
-q_3 & q_2 & -q_1  & q_0 \\
\end{array} \right)
\end{equation*}

要导出的二次型矩阵在 $\bq$中不是常数,这意味着这些运算不再是平凡的。然而,由于 $\bq$ 分量中的 $R$ 依赖的特殊形式,高阶张量可以避免,如下所示。

\subsubsection{``Single $R$'' 二次型}

通过计算二次型并取偏导数,我们得到 (将它们放在列向量中)

\begin{equation*}
\frac{\partial (\v{v}^TR\v{w})}{\partial \bq} =\left( \frac{\partial (\v{v}^TR\v{w})}{\partial \bq_i}\right)_i =
\end{equation*}

\footnotesize
$$  2\left( \begin {array}{c} {\it w_1}\,{\it v_1}\,{\it q_0}+{\it w_1}\,{\it v_2}\,{\it q_3}-{\it w_1}\,{\it v_3}\,{\it q_2}-{\it w_2}\,{\it v_1}\,{\it q_3}+{\it w_2}\,{\it v_2}\,{\it q_0}+{\it w_2}\,{\it v_3}\,{\it q_1}+{\it w_3}\,{\it v_1}\,{\it q_2}-{\it w_3}\,{\it v_2}\,{\it q_1}+{\it w_3}\,{\it v_3}\,{\it q_0}\\\noalign{\medskip}{\it w_1}\,{\it v_1}\,{\it q_1}+{\it w_1}\,{\it v_2}\,{\it q_2}+{\it w_1}\,{\it v_3}\,{\it q_3}+{\it w_2}\,{\it v_1}\,{\it q_2}-{\it w_2}\,{\it v_2}\,{\it q_1}+{\it w_2}\,{\it v_3}\,{\it q_0}+{\it w_3}\,{\it v_1}\,{\it q_3}-{\it w_3}\,{\it v_2}\,{\it q_0}-{\it w_3}\,{\it v_3}\,{\it q_1}\\\noalign{\medskip}-{\it w_1}\,{\it v_1}\,{\it q_2}+{\it w_1}\,{\it v_2}\,{\it q_1}-{\it w_1}\,{\it v_3}\,{\it q_0}+{\it w_2}\,{\it v_1}\,{\it q_1}+{\it w_2}\,{\it v_2}\,{\it q_2}+{\it w_2}\,{\it v_3}\,{\it q_3}+{\it w_3}\,{\it v_1}\,{\it q_0}+{\it w_3}\,{\it v_2}\,{\it q_3}-{\it w_3}\,{\it v_3}\,{\it q_2}\\\noalign{\medskip}-{\it w_1}\,{\it v_1}\,{\it q_3}+{\it w_1}\,{\it v_2}\,{\it q_0}+{\it w_1}\,{\it v_3}\,{\it q_1}-{\it w_2}\,{\it v_1}\,{\it q_0}-{\it w_2}\,{\it v_2}\,{\it q_3}+{\it w_2}\,{\it v_3}\,{\it q_2}+{\it w_3}\,{\it v_1}\,{\it q_1}+{\it w_3}\,{\it v_2}\,{\it q_2}+{\it w_3}\,{\it v_3}\,{\it q_3}\end {array} \right) 
 .$$ \normalsize \\

得到的向量很难看,但可以看出它在 $\bq$ 中是线性的,因此可以用矩阵向量积重写:

\footnotesize
$$ 2  \underbrace{
\left( \begin {array}{cccc} {\it v_1}\,{\it w_1}+{\it v_2}\,{\it w_2}+{\it v_3}\,{\it w_3}&{\it v_3}\,{\it w_2}-{\it v_2}\,{\it w_3}&-{\it v_3}\,{\it w_1}+{\it v_1}\,{\it w_3}&{\it v_2}\,{\it w_1}-{\it v_1}\,{\it w_2}\\\noalign{\medskip}{\it v_3}\,{\it w_2}-{\it v_2}\,{\it w_3}&{\it v_1}\,{\it w_1}-{\it v_2}\,{\it w_2}-{\it v_3}\,{\it w_3}&{\it v_1}\,{\it w_2}+{\it v_2}\,{\it w_1}&{\it v_1}\,{\it w_3}+{\it v_3}\,{\it w_1}\\\noalign{\medskip}-{\it v_3}\,{\it w_1}+{\it v_1}\,{\it w_3}&{\it v_1}\,{\it w_2}+{\it v_2}\,{\it w_1}&{\it v_2}\,{\it w_2}-{\it v_1}\,{\it w_1}-{\it v_3}\,{\it w_3}&{\it v_2}\,{\it w_3}+{\it v_3}\,{\it w_2}\\\noalign{\medskip}{\it v_2}\,{\it w_1}-{\it v_1}\,{\it w_2}&{\it v_1}\,{\it w_3}+{\it v_3}\,{\it w_1}&{\it v_2}\,{\it w_3}+{\it v_3}\,{\it w_2}&{\it v_3}\,{\it w_3}-{\it v_1}\,{\it w_1}-{\it v_2}\,{\it w_2}\end {array} \right) 
}_{\mbox{\normalsize $\Delta[\v{v},\v{w}]$}}
 \left( \begin {array}{c} {\it q_0}\\\noalign{\medskip}{\it q_1}\\\noalign{\medskip}{\it q_2}\\\noalign{\medskip}{\it q_3}\end {array} \right) $$ \normalsize


通过仔细检查 $\Delta[\v{v},\v{w}]$,我们可以确定矩阵中允许紧凑符号的结构


\begin{equation}
\Delta[\v{v},\v{w}] = 
\left( \begin {array}{cc}
\v{w} \cdot \v{v}    &      (\v{w} \c \v{v})^T   \\
\v{w} \c \v{v}         &       \v{w}\v{v}^T + \v{v}\v{w}^T  - \v{w} \cdot \v{v} \ I_3 
\end {array} \right).
\label{eqD_DELTA}
\end{equation}

那就是


\begin{equation}
\frac{\partial (\v{v}^TR\v{w})}{\partial \bq} =
2 \Delta[\v{v},\v{w}] \bq
\label{eqD_DELTAq}
\end{equation}

并且因为 $\v{v}^TR^T\v{w} = \v{w}^TR\v{v}$ ,我们还有

\begin{equation}
\frac{\partial (\v{v}^TR^T\v{w})}{\partial \bq} =
2 \Delta[\v{w},\v{v}] \bq
\label{eqD_DELTAq2}
\end{equation}


\subsubsection{``Double $R$'' 二次型}

我们现在对包含 $RJR^T$的二次型的导数感兴趣,也就是说,与 $\bq$ 相关的矩阵 $R$ 出现两次。 $J$ 是一个惯性矩阵,因此, $J=J^T$ 。这次,左边和右边的向量是一样的,命名为 $\v{u}$ 。

\begin{align*}
\half\frac{\partial }{\partial \bq} \left( \v{u}^TRJR^T\v{u}\right)  &=
\half \left(   \v{u}^T  \frac{\partial R}{\partial \bq_i}  JR^T   \v{u}   \right)_i  +
\half \left(   \v{u}^T RJ  \frac{\partial R^T}{\partial \bq_i}   \v{u}   \right)_i  \\
  &=   \left(   \v{u}^T  \frac{\partial R}{\partial \bq_i}  JR^T   \v{u}   \right)_i .
\end{align*}

因此

\begin{equation}
\half\frac{\partial }{\partial \bq} \left( \v{u}^TRJR^T\v{u}\right)  = 2 \Delta[\v{u},JR^T\v{u}] \bq
\label{eqD_DELTAq3}
\end{equation}

\subsubsection{特性}

通过观察方程 \eqref{eqD_DELTA},可以注意到以下关系


\begin{equation}
 \Delta[\v{v}_1+\v{v}_2,\v{w}] = \Delta[\v{v}_1,\v{w}] + \Delta[\v{v}_2,\v{w}] 
\label{eqD_DELTA_r1}
\end{equation}

\begin{equation}
 \Delta[\v{v},\v{w}_1+\v{w}_2] = \Delta[\v{v},\v{w}_1] + \Delta[\v{v},\v{w}_2] 
\label{eqD_DELTA_r2}
\end{equation}

\begin{equation}
 \Delta \left[\sum_{i=1}^{n}\v{v}_i,\sum_{j=1}^{m}\v{w}_j \right] = \sum_{i=1}^{n} \sum_{j=1}^{m} \Delta[\v{v}_i,\v{w}_j] 
\label{eqD_DELTA_r3}
\end{equation}

\begin{equation}
 \Delta[\alpha \v{v},\beta \v{w}] = \alpha \beta \Delta[\v{v},\v{w}]  
\label{eqD_DELTA_r4}
\end{equation}


%%%%%%%%%%%%%%%%%%%%%%%%%%%%%%%%%%%%%%%%%%
%%%%%%%%%%%%%%%%%%%%%%%%%%%%%%%%%%%%%%%%%%
%%%%%%%%%%%%%%%%%%%%%%%%%%%%%%%%%%%%%%%%%%
%%%%%%%%%%%%%%%%%%%%%%%%%%%%%%%%%%%%%%%%%%
%%%%%%%%%%%%%%%%%%%%%%%%%%%%%%%%%%%%%%%%%%

\vspace{12pt}

%\subsection{Time Derivative of $R\v{w}$ and $R^T\v{w}$}
%\label{sec_dRv_dt}

%In computing $\frac{\partial}{\partial t}\frac{\partial L}{\partial \dbq}$ in the non-inertial quaternion model, we see that we need to take the time derivative of expressions of the form $R\v{w}$ with $R$ the time dependent rotation matrix. Remembering that $R=EG^T$ and verifying that $\dot{E}G^T=E\dot{G}^T$, we may write

%\begin{align*}
%\dot{R} &= \dot{E}G^T + E\dot{G}^T = 2 E\dot{G}^T \\
%\dot{R}^T &= \dot{G}E^T + G\dot{E}^T = 2 G\dot{E}^T.
%\end{align*}

%We can now concentrate on the products $\dot{G}^T\v{w}$ and $\dot{E}^T\v{w}$

%\begin{equation*}
%\dot{G}^T\v{w} = 
% \left( \begin {array}{c} -{\it \dot{q}_1}\,{\it w_1}-{\it \dot{q}_2}\,{\it w_2}-{\it \dot{q}_3}\,{\it w_3}\\\noalign{\medskip}{\it \dot{q}_0}\,{\it w_1}-{\it \dot{q}_3}\,{\it w_2}+{\it \dot{q}_2}\,{\it w_3}\\\noalign{\medskip}{\it \dot{q}_3}\,{\it w_1}+{\it \dot{q}_0}\,{\it w_2}-{\it \dot{q}_1}\,{\it w_3}\\\noalign{\medskip}-{\it \dot{q}_2}\,{\it w_1}+{\it \dot{q}_1}\,{\it w_2}+{\it \dot{q}_0}\,{\it w_3}\end {array} \right)  =
% \underbrace{
%  \left( \begin {array}{cccc} 0&-{\it w_1}&-{\it w_2}&-{\it w_3}\\\noalign{\medskip}{\it w_1}&0&{\it w_3}&-{\it w_2}\\\noalign{\medskip}{\it w_2}&-{\it w_3}&0&{\it w_1}\\\noalign{\medskip}{\it w_3}&{\it w_2}&-{\it w_1}&0\end {array} \right) 
%  }_{\Gamma[ \v{w}]}
%  \left( \begin {array}{c} {\it \dot{q}_0}\\\noalign{\medskip}{\it \dot{q}_1}\\\noalign{\medskip}{\it \dot{q}_2}\\\noalign{\medskip}{\it \dot{q}_3}\end {array} \right) 
%\end{equation*}

%\begin{equation*}
%\dot{E}^T\v{w} = 
% \left( \begin {array}{c} -{\it \dot{q}_1}\,{\it w_1}-{\it \dot{q}_2}\,{\it w_2}-{\it \dot{q}_3}\,{\it w_3}\\\noalign{\medskip}{\it \dot{q}_0}\,{\it w_1}+{\it \dot{q}_3}\,{\it w_2}-{\it \dot{q}_2}\,{\it w_3}\\\noalign{\medskip}-{\it \dot{q}_3}\,{\it w_1}+{\it \dot{q}_0}\,{\it w_2}+{\it \dot{q}_1}\,{\it w_3}\\\noalign{\medskip}{\it \dot{q}_2}\,{\it w_1}-{\it \dot{q}_1}\,{\it w_2}+{\it \dot{q}_0}\,{\it w_3}\end {array} \right) =
% \underbrace{
%  \left( \begin {array}{cccc} 0&-{\it w_1}&-{\it w_2}&-{\it w_3}\\\noalign{\medskip}{\it w_1}&0&-{\it w_3}&{\it w_2}\\\noalign{\medskip}{\it w_2}&{\it w_3}&0&-{\it w_1}\\\noalign{\medskip}{\it w_3}&-{\it w_2}&{\it w_1}&0\end {array} \right) 
%  }_{\bar{\Gamma}[ \v{w}]}
%   \left( \begin {array}{c} {\it \dot{q}_0}\\\noalign{\medskip}{\it \dot{q}_1}\\\noalign{\medskip}{\it \dot{q}_2}\\\noalign{\medskip}{\it \dot{q}_3}\end {array} \right) .
%\end{equation*}

%That is $\frac{\partial}{\partial t}(R\v{w}) = \dot{R}\v{w}+R\dot{\v{w}}$ and $\frac{\partial}{\partial t}(R^T\v{w}) = \dot{R^T}\v{w}+R^T\dot{\v{w}}$ can both be computed using

%\begin{equation}
% \dot{R}\v{w} = 2E\dot{G}^T\v{w} = 2E \Gamma[\v{w}] \dbq = E \Gamma[\v{w}] G^T \wp
% \label{dRv_dt}
%\end{equation}

%\begin{equation}
% \dot{R}^T\v{w} = 2G\dot{E}^T\v{w} = 2G \bar{\Gamma}[\v{w}] \dbq = G \bar{\Gamma}[\v{w}] G^T \wp
% \label{dRTv_dt}
%\end{equation}
%See next section for a more useful form.


\subsection{ $R$ 的时间导数}

首先要注意的是,通过标识,我们可以确认
\begin{equation}
G^T G = E^T E = I_4 - \b q \b q^T
\label{identity_in_R4}
\end{equation}
其中 $I_4$ 是 $\mathbb{R}^4$ 中的特征矩阵。也要记住 
\begin{equation*}
\Omega' = 2G\dot G^T = -2\dot G G^T \qquad \textrm{with} \qquad \Omega' \v v = \wp \times \v v
\end{equation*}
和
\begin{equation*}
\wp = 2G \dot{\b q} = -2\dot G \b q.
\end{equation*}

现在观察
\begin{align}
\Omega' R^T &= 2G\dot G^T G E^T \nonumber \\
                        &= -2\dot G G^T G E^T \nonumber \\
                        &=-2\dot G (I_4 -\b q \b q^T) E^T \nonumber \\
                        &=-2\dot G E^T -2\dot G \b q \underbrace{\b q^T E^T}_{(E\b q)^T=\v 0} \nonumber \\
                        &= -2\dot G E^T = -\dot R^T. \nonumber
\end{align}

我们终于可以写为 

\begin{gather}
\dot R^T   = -\Omega' R^T      \label{dotRt_with_Omega} \\
\dot R       = -R \Omega'^T = R \Omega'.
\label{dotR_with_Omega}
\end{gather}



%     \v{v}
%     \v{w}
%     \v{u}







%Bibliographie:
%\begin{thebibliography}{9}

%\bibitem{QFREP}
%	Quaternion, Finite Rotation and Euler Parameters\\
%	Arend L. Schwab \\
%	\emph{http://tam.cornell.edu/\~{}als93/quaternion.pdf}
%	
%\bibitem{Gros}
%	Quaternion based dynamics - Single Turbine Aircraft - Lagrange and Hamiltonian approaches \\
%	S. Gros\\
%	LA, EPFL.
%		
%\bibitem{Gold}
%	Classical Mechanics \\
%	Herbert Goldstein.
%	
%\bibitem{Wells}
%	Lagrangian Dynamics \\
%	Dare A. Wells \\
%	Schaum's Outline Series.
	
	

%	\bibitem{lamport94}
%	  Leslie Lamport,
%	  \emph{\LaTeX: A Document Preparation System}.
%	  Addison Wesley, Massachusetts,
%	  2nd Edition,
%	  1994.

%rajouter: bouquin de Papa (mcanique), bouquin OpenGL	
	
%\end{thebibliography}





%\end{document}
