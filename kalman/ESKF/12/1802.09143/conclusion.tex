\section{讨论与结论}
\label{secConclusion}

在本文中,我们比较了四种多旋翼飞机的姿态控制器,其中三种来自文献,一种为新的控制器。 
所有的性能都与一阶相同。
然而,事实表明,角加速度与旋转矩阵的斜对称部分成比例的控制器存在安全问题,因为如果总的角误差接近 $180^\circ$,则大的姿态误差可能会持续任意一段时间(注意,该行为存在于该点附近的一个相当大的邻域)。
使用与姿态误差的旋转向量表示成比例的角加速度被证明是可取的,因为不存在这种危险。
此外,任何纯的初始旋转都会像阻尼二阶系统一样衰减,从而产生更直观的行为。

将姿态误差分解为倾斜和偏航分量,允许控制器优先考虑飞行器的推力方向,并因此有可能更快地收敛飞行器姿态中主导平移运动的那部分。
我们在文献 \cite{brescianini2013nonlinear} 中提出了基于四元数的倾斜优先控制器,其中角加速度与一半误差角的正弦成比例。
该控制器没有像斜对称控制器那样引起任何安全问题。

受此启发,我们还提出了一种新型控制器,该控制器优先考虑飞行器倾斜,但角加速度与角度成比例。
该控制器采用旋转轴和角度构成,便于直观描述,并且稳定性用相对简单的 Lyapunov 函数表示。 
具体而言,控制动作是两种控制动作组合的结果,是与旋转向量成比例控制和与单一的倾斜误差成比例控制的组合。
虽然它受到基于四元数的倾斜优先控制器的启发,但其行为与之不同,而且闭环性能与基于四元数的控制器相比更有优势。
此外,如果倾斜的优先级不高于偏航角,该新型控制器继续保持良好的性能,特别是当偏航方向的控制权重收敛到倾斜方向的控制权重时,在极限情况下会收敛到基于旋转向量的控制。
实验结果验证了现实条件下该控制器的性能。 

因此,我们建议将该新型控制器应用于多旋翼飞机,特别是在需要对大的干扰具有鲁棒性的地方。 
该控制器的性能优于标准控制器,具体而言,它不像斜对称控制器那样在大的姿态误差下收敛性差;在大多数情况下,它优于基于旋转向量的控制器;姿态误差收敛的速度(使用潜在的不太激进的输入)比基于倾斜优先的四元数控制器更快。


 
