%%%%%%%%%%%%%%%%%%%%%%%%%%%%%%%%%%%%%%%%%
% Beamer Presentation
% LaTeX Template
% Version 1.0 (10/11/12)
%
% This template has been downloaded from:
% http://www.LaTeXTemplates.com
%
% License:
% CC BY-NC-SA 3.0 (http://creativecommons.org/licenses/by-nc-sa/3.0/)
%
%%%%%%%%%%%%%%%%%%%%%%%%%%%%%%%%%%%%%%%%%

%----------------------------------------------------------------------------------------
%	PACKAGES AND THEMES
%----------------------------------------------------------------------------------------

\documentclass{beamer}

\mode<presentation> {

% The Beamer class comes with a number of default slide themes
% which change the colors and layouts of slides. Below this is a list
% of all the themes, uncomment each in turn to see what they look like.

%\usetheme{default}
%\usetheme{AnnArbor}
%\usetheme{Antibes}
%\usetheme{Bergen}
%\usetheme{Berkeley}
%\usetheme{Berlin}
%\usetheme{Boadilla}
%\usetheme{CambridgeUS}
%\usetheme{Copenhagen}
%\usetheme{Darmstadt}
%\usetheme{Dresden}
%\usetheme{Frankfurt}
%\usetheme{Goettingen}
%\usetheme{Hannover}
%\usetheme{Ilmenau}
%\usetheme{JuanLesPins}
%\usetheme{Luebeck}
\usetheme{Madrid}
%\usetheme{Malmoe}
%\usetheme{Marburg}
%\usetheme{Montpellier}
%\usetheme{PaloAlto}
%\usetheme{Pittsburgh}
%\usetheme{Rochester}
%\usetheme{Singapore}
%\usetheme{Szeged}
%\usetheme{Warsaw}

% As well as themes, the Beamer class has a number of color themes
% for any slide theme. Uncomment each of these in turn to see how it
% changes the colors of your current slide theme.

%\usecolortheme{albatross}
%\usecolortheme{beaver}
%\usecolortheme{beetle}
%\usecolortheme{crane}
%\usecolortheme{dolphin}
%\usecolortheme{dove}
%\usecolortheme{fly}
%\usecolortheme{lily}
%\usecolortheme{orchid}
%\usecolortheme{rose}
%\usecolortheme{seagull}
%\usecolortheme{seahorse}
%\usecolortheme{whale}
%\usecolortheme{wolverine}

%\setbeamertemplate{footline} % To remove the footer line in all slides uncomment this line
%\setbeamertemplate{footline}[page number] % To replace the footer line in all slides with a simple slide count uncomment this line

\setbeamertemplate{navigation symbols}{} % To remove the navigation symbols from the bottom of all slides uncomment this line
}

\usepackage{graphicx} % Allows including images
\usepackage{booktabs} % Allows the use of \toprule, \midrule and \bottomrule in tables
\usepackage{hyperref}
\usepackage{amsmath}
\usepackage[UTF8]{ctex}

%----------------------------------------------------------------------------------------
%	TITLE PAGE
%----------------------------------------------------------------------------------------

\title[ESKF 简介]{误差状态卡尔曼滤波器(ESKF)简介} % The short title appears at the bottom of every slide, the full title is only on the title page

\author{Martin Brandt} % Your name
\date{January 22, 2020} % Date, can be changed to a custom date

\begin{document}

\begin{frame}
\titlepage % Print the title page as the first slide
\end{frame}

\begin{frame}
\frametitle{概述} % Table of contents slide, comment this block out to remove it
\tableofcontents % Throughout your presentation, if you choose to use \section{} and \subsection{} commands, these will automatically be printed on this slide as an overview of your presentation
\end{frame}

%----------------------------------------------------------------------------------------
%	PRESENTATION SLIDES
%----------------------------------------------------------------------------------------
\section{动机}

\begin{frame}
    \frametitle{动机}
    \begin{figure}
    \includegraphics[width=0.8\linewidth]{adcs.png}
    \end{figure}
    控制器需要知道姿态,以便控制它,但没有办法直接测量它 $\rightarrow$ 我们必须估计它!
\end{frame}

%------------------------------------------------

\begin{frame}
    \frametitle{警告}
    这些幻灯片总结了\textbf{很多}东西,所以你可能不会理解所有的东西。但希望你能理解什么是卡尔曼滤波器,以及为什么我们使用误差状态卡尔曼滤波器的步骤背后的推理。这也应该是作为一个ADC的新人,希望进入ESKF领域的介绍和参考。 
\end{frame}

%------------------------------------------------

\section{状态空间模型}

\begin{frame}
    \frametitle{状态空间模型}
    我们想用向量的形式来表示一个任意的微分方程组。一般来说: $\dot{x} = f(x, u)$

    \begin{block}{连续LTI状态空间模型}
        \begin{equation}
        \begin{aligned}
            \dot{x} &= A x + B u \\
            y &=C x + D u
        \end{aligned}
        \end{equation}
    \end{block}

    \begin{block}{离散 LTI 状态空间模型}
        \begin{equation}
        \begin{aligned}
            x[k+1] &= A x[k] + B u[k] \\
            y[k] &= C x[k] + D u[k]
        \end{aligned}
        \end{equation}
    \end{block}
\end{frame}

%------------------------------------------------

\begin{frame}
    \frametitle{质量-弹簧-阻尼器示例}

    \begin{block}{你是如何习惯看它的}
        \begin{equation}
            m \ddot{x}+d \dot{x} + kx=u
        \end{equation}
    \end{block}

    \begin{block}{状态空间表示}
        \begin{equation}
        \left[\begin{array}{c}{{\dot{x_1}}} \\ {\dot{x_2}}\end{array}\right] =\left[\begin{array}{cc}{0} & {1} \\ {-\frac{k}{m}} & {-\frac{d}{m}}\end{array}\right] \left[\begin{array}{c}{{x_1}} \\ {x_2}\end{array}\right] +\left[\begin{array}{c}{0} \\ {\frac{1}{m}}\end{array}\right] u
        \end{equation}
    \end{block}
\end{frame}

%------------------------------------------------

\section{卡尔曼滤波器}

\begin{frame}
    \frametitle{Luenberger 观测器}
    假设我们有一个系统的状态空间模型。我们如何估计系统的状态? 

    \begin{block}{逻辑初试(开环观测器)}
        \begin{equation}
            \dot{\hat{x}} = A \hat{x} + B u 
        \end{equation}
    \end{block}

    但由于模型的不确定性,我们的估计值将很快偏离实际值 $\rightarrow$ 包括一个基于测量的校正项(闭环回路) $\rightarrow$ Luenberger 观测器

    \begin{block}{Luenberger 观测器}
        \begin{equation}
            \dot{\hat{x}} = A \hat{x} + B u + L (y - \hat{y}), \quad \hat{y} = C \hat{x}
        \end{equation}
    \end{block}
    但是我们如何决定增益 $L$?
\end{frame}

%------------------------------------------------

\begin{frame}
    \frametitle{卡尔曼滤波器}
    首先假设我们的过程模型和测量模型包括\textbf{正态分布}噪声:
    \begin{block}{随机LTI系统}
        \begin{equation}
            \begin{aligned}
                \dot{x} &= A x + B u + w \\
                y &=C x + D u + v
            \end{aligned}
        \end{equation}
    \end{block}
    卡尔曼滤波器是该系统的最优Luenberger观测器,在这个意义上,它使\textbf{均方误差}最小,即 $E\{(x-\hat{x})^2\}$. 
\end{frame}

%------------------------------------------------

\begin{frame}
    \frametitle{卡尔曼滤波器方程式}
    对于离散情况(这是在微控制器上实现的),卡尔曼滤波方程式为:
    \begin{figure}
        \includegraphics[width=0.8\linewidth]{kf_equations.png}
    \end{figure}
    这里的细节并不重要,但请注意 predict + correct 的步骤。 
\end{frame}

%------------------------------------------------

\begin{frame}
    \frametitle{卫星运动学与动力学}
    让我们尝试将卡尔曼滤波器应用于我们的卫星:
    \begin{block}{卫星运动学}
        \begin{equation}
            \begin{aligned}
                \dot{\mathbf{q}}=\frac{1}{2} \mathbf{q} \otimes \boldsymbol{\omega}
            \end{aligned}
        \end{equation}
    \end{block}
    \begin{block}{卫星动力学}
        \begin{equation}
            \begin{aligned}
                \dot{\omega}=J^{-1}\left[\mathbf{L}-\omega \times\left(J \omega\right)\right]
            \end{aligned}
        \end{equation}
    \end{block}
    问题:该系统是高度非线性的,因此卡尔曼滤波不能直接应用(因为它假设一个线性模型)。
\end{frame}

%------------------------------------------------

\begin{frame}
    \frametitle{扩展卡尔曼滤波}
    解决这个问题最简单的方法是\textbf{在每个时间步线性化非线性动力学} $\rightarrow$ 扩展卡尔曼滤波 (EKF)。 \\~\\

    问题:建模不确定性---动力学要求我们知道卫星的惯性矩阵,由于动力学高度非线性,EKF可能会发散:( \\~\\

    $\rightarrow$ 放下动力学,只使用运动学: $\dot{\mathbf{q}}=\frac{1}{2} \mathbf{q} \otimes \boldsymbol{\omega}$. \\
    我们让 $\omega$ 成为“控制输入”,我们使用 IMU 测量得到它。 \\~\\
\end{frame}

%------------------------------------------------

\section{误差状态卡尔曼滤波器(ESKF)}

\begin{frame}
    \frametitle{误差状态卡尔曼滤波器}
    现在我们正接近一个有点实用的算法,但动力学仍然是高度非线性的,这意味着EKF将表现得很差(容易发散)。\\~\\

    解决方法是研究\textbf{误差四元数}: $q=\delta q \otimes \hat{q}$\\~\\

    如果我们估计 $\delta q \approx \left[\begin{array}{c}{\frac{1}{2} \delta \boldsymbol{\theta}} \\ {1}\end{array}\right]$,忽略一些项我们得到:
        \begin{block}{误差四元数运动学}
            \begin{equation}
                \delta \dot{\theta}=-S\left(\omega_{m}\right) \delta \theta-n_r
            \end{equation}
        \end{block}
    这是线性的,所以我们可以离散它,并使用好的老卡尔曼滤波器。这比EKF稳定得多,非常好,非常好。
\end{frame}

%------------------------------------------------

\begin{frame}
    \frametitle{误差状态卡尔曼滤波器}
    最后一个问题:IMU测量值漂移... $\rightarrow$ 估计速率陀螺偏差 $b$ 以及姿态四元数 $q$。我们假设偏差误差遵循以下动力学(随机游走):
    \begin{block}{偏差动力学}
        \begin{equation}
            \dot{\delta b}=\dot{b}-\dot{\hat{b}}=n_{w}
        \end{equation}
    \end{block}
    新的带有偏差的误差四元数动力学是:
    \begin{block}{误差四元数运动学}
        \begin{equation}
            \delta \dot{\theta}=-S\left(\omega_{m}-b\right) \delta \theta-\delta b-n_r
        \end{equation}
    \end{block}
    如果我们将KF方程式应用到这些方程中,我们最终(经过大量的数学运算)得到了我们目前使用的ESKF算法。我不想费心去写所有的方程式,如果需要的话你可以查一下。
\end{frame}

\begin{frame}
    \frametitle{误差状态卡尔曼滤波器}
    虽然这一切看起来相当复杂,但它实际上只是带有一些额外技巧的常规卡尔曼滤波器(估计误差而不是真实状态,使用陀螺测量作为输入,还估计偏差)。作为ESKF的用户,你真正需要知道的是:
    \begin{figure}
        \includegraphics[width=0.7\linewidth]{ESKF.png}
    \end{figure}
\end{frame}
%------------------------------------------------

\begin{frame}
    \frametitle{ESKF预测方程参考}
    接下来的几张幻灯片将总结ESKF中的实际方程式,在试图理解代码时,主要应作为参考。我出于懒惰跳过了一些最讨厌的表达式,看提供的源码去找它们。\\~\\
    从我们非常简单的模型中预测 $\hat{b}_{k+1|k}$ 和 $\hat{\omega}_{k+1|k}$ :
    \begin{align}
        \hat{b}_{k+1|k} &\leftarrow \hat{b}_{k|k} \\
        \hat{\omega}_{k+1|k} &\leftarrow \omega_m - \hat{b}_{k|k}
    \end{align}
    使用一阶四元数积分预测 $\hat{q}_{k+1|k}$ :
    \begin{equation}
        \hat{q}_{k+1|k} \leftarrow \left({q}\{\bar{\omega} \Delta t\}+\frac{\Delta t^{2}}{24}\left[\begin{array}{c}{0} \\ {\omega_{k|k} \times \omega_{k+1|k}}\end{array}\right]\right) \otimes \hat{q}_{k|k}
    \end{equation}
    使用状态转移矩阵和过程噪声协方差矩阵预测协方差矩阵:
\begin{equation}
    {P}_{k+1 | k} \leftarrow {\Phi} {P}_{k | k} {\Phi}^{\top}+{Q}
\end{equation}
\end{frame}

%------------------------------------------------

\begin{frame}
    \frametitle{ESKF更新方程参考}
    首先计算当前姿态旋转矩阵 $A(\hat{q}_{k+1|k})$:
    \begin{equation}
        A(\hat{q}_{k+1|k}) \leftarrow \begin{bmatrix}
            1 - 2(q_2^2 + q_3^2) & 2(q_1 q_2 + q_3 q_4) & 2(q_1 q_3 - q_2 q_4) \\
            2(q_1 q_2 - q_3 q_4) & 1 - 2(q_1^2 + q_3^2) & 2(q_2 q_3 + q_1 q_4) \\
            2(q_1 q_3 + q_2 q_4) & 2(q_2 q_3 - q_1 q_4) & 1 - 2(q_1^2 + q_2^2)
        \end{bmatrix}
    \end{equation}
\end{frame}

%------------------------------------------------

\begin{frame}
    \frametitle{ESKF更新方程参考}
    然后依次对每个测量执行以下步骤: \\
    使用 $A(\hat{q}_{k+1|k})$ 将 $\hat{z}$ 从地心惯性系(Earth Central Inertial)旋转至机体坐标系:
    \begin{equation}
        \hat{z}_b \leftarrow A(\hat{q}_{k+1|k}) \hat{z}_{ECI}
    \end{equation}
    计算测量矩阵:
    \begin{equation}
        H_k \leftarrow \begin{bmatrix}
            S(\hat{z}_b) & 0_{3x3}
        \end{bmatrix}
    \end{equation}
    计算卡尔曼增益:
    \begin{equation}
        K_k \leftarrow P_{k+1|k} H_k^{\top} S_k^{-1}, \quad S_k \leftarrow H_k P_{k+1|k} H_k^{\top} + R
    \end{equation}
    计算校正:
    \begin{equation}
        \Delta x \leftarrow \Delta x + K (\varepsilon - H_k \Delta x), \quad \varepsilon \leftarrow z - \hat{z}
    \end{equation}
    更新协方差估计:
    \begin{equation}
        {P}_{k+1 | k+1} \leftarrow \left({I}_{6}-{K} {H}\right) {P}_{k+1 | k}\left({I}_{6}-{K} {H}\right)^{\top}+{K} {R} {K}^{\top}
    \end{equation}
\end{frame}

%------------------------------------------------

\begin{frame}
    \frametitle{ESKF更新方程参考}
    更新姿态四元数估计:
    \begin{equation}
        \hat{{q}}_{k+1 | k+1} \leftarrow \hat{{q}}_{k+1 | k} \otimes \delta \hat{{q}}, \quad 
    \delta \hat{{q}}=\left[\begin{array}{c}{\frac{1}{2}\Delta x_{1:3}} \\ {\sqrt{1-\frac{1}{2}\Delta x_{1:3}^{\top} \frac{1}{2}\Delta x_{1:3}}}\end{array}\right]
    \end{equation}
    更新偏差估计:
    \begin{equation}
        \hat{{b}}_{k+1 | k+1}=\hat{{b}}_{k+1 | k}+\Delta x_{4:6}
    \end{equation}
    更新角速度估计:
    \begin{equation}
        \hat{\omega}_{k+1 | k+1}=\omega_{m}-\hat{{b}}_{k+1 | k+1}
    \end{equation}
    周而复始!
\end{frame}

%------------------------------------------------

\begin{frame}
    \frametitle{ESKF code}
    \large{\centerline{Let's have a brief look at the code I guess?}}
   \begin{center}
    \small{Code: \url{https://git.orbitntnu.no/adcs/libeskf} \\
    Tests: \url{https://git.orbitntnu.no/adcs/eskf_test} \\
    Matlab tests: \url{https://git.orbitntnu.no/Martimos/adcs_simulation/tree/eskf-c-test}}
   \end{center}
\end{frame}

%------------------------------------------------

\begin{frame}
\huge{\centerline{Thank you for coming to my TEDx talk}}
\begin{center}
\small{Read more in J. Sola, "Quaternion kinematics for the error-state Kalman filter" and N. Trawny \& S. Roumeliotis, "Indirect Kalman Filter for 3D Attitude Estimation"}
\end{center}
\end{frame}

%----------------------------------------------------------------------------------------

\end{document} 
