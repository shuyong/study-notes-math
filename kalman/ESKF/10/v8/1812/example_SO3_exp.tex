% !TEX root = micro_Lie_theory.tex

%%%%%%%%%%%%%%%%%%%%%%%%% SO3 exp %%%%%%%%%%%%%%%%%%%%%%%%%%%%%%
\begin{fexample}{$\SO(3)$ 的指数映射}
\label{ex:SO3_exp}
%
%We derive here an expression for the exponential map,
%%
%\begin{align*}
%\exp : \so(3)&\to\SO(3);\\
%\hatx{\bth}&\mapsto\exp(\hatx{\bth})
%~.
%\end{align*}
%
我们在 \exRef{ex:SO3} 中看到 
$
\dot\bfR = \bfR\hatx{\bfomega}  \in \mtanat{\SO(3)}{\bfR}.
$
%
对于 $\bw$ 常数,这是一个常微分方程(ODE),其解为 $\bfR(t) = \bfR_0\exp(\hatx{\bfomega}t)$。在原点 $\bfR_0=\bfI$ ,我们有指数映射,
%
\begin{align*}
\bfR(t) &= \exp(\hatx{\bfomega}t) && \in\SO(3)
~. 
\end{align*}
%
我们现在将向量 $\bth\te\bfu\theta \te \bfomega t\in\bbR^3$ 定义为角-轴形式的积分旋转,其中角度 $\theta$ 和单位轴 $\bfu$。 
因此, $\hatx{\bth}\in\so(3)$ 是Lie代数中表示的总旋转。
我们用上面的代换它。 
然后把指数写成幂级数,
%
\begin{align*}
\bfR &= \exp(\hatx{\bth})= \sum_k\frac{\theta^k}{k!}{(\hatx{\bfu})^k} 
~.
\end{align*}
%
为了找到一个封闭形式的表达式,我们写下 $\hatx{\bfu}$ 的几次幂,
%
\begin{align*}
\hatx{\bfu}^0&= \bfI,
&
\hatx{\bfu}^1&= \hatx{\bfu},
\\
\hatx{\bfu}^2&= \bfu\bfu\tr -\bfI,
&
\hatx{\bfu}^3&=-\hatx{\bfu},
\\
\hatx{\bfu}^4&=-\hatx{\bfu}^2,
&
\cdots
\end{align*}
%
并意识到,所有这些都可以表示为 $\bfI$、$\hatx{\bfu}$ 或 $\hatx{\bfu}^2$ 的倍数。
因此,我们将这个级数改写为,
%We substitute them in the series, and group the terms according to the nature of the powers, 
%
\begin{align*}
\bfR = \bfI &+ \hatx{\bfu}\big(\theta - \tfrac1{3!}\theta^3 + \tfrac1{5!}\theta^5 - \cdots\big) \\
  &+ \hatx{\bfu}^2\big(\tfrac12\theta^2-\tfrac1{4!}\theta^4+\tfrac1{6!}\theta^6-\cdots\big)
  ~,
\end{align*}
%
其中,我们确定了 $\sin\theta$ 和 $\cos\theta$ 的级数,得到了封闭形式,
%
\begin{align*}
\bfR=\exp(\hatx{\bfu\theta}) &= 
\bfI + \hatx{\bfu}\sin\theta + \hatx{\bfu}^2(1\!-\!\cos\theta)
~.
\end{align*}
%
这个表达式是众所周知的Rodrigues旋转公式。
它可以用作大写指数方程,只需这么操作 $\bfR=\Exp(\bfu\theta)=\exp(\hatx{\bfu\theta})$。
\end{fexample}
%%%%%%%%%%%%%%%%%%%%%%%%%%%%%%%%%%%%%%%%%%%%%%%%%%%%%
