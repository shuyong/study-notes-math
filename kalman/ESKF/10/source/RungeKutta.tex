% !TEX root = kinematics.tex


%%%%%%%%%%%%%%%%%%%%%%%%%%%%%%%%%%%%%%%%%%%%%%%%%%%%%%%%%%%%%%
\section{Runge-Kutta 数值积分方法}
\label{sec:NumInt}

我们的目标是积分那些形式的非线性微分方程
%
\begin{equation}
\dot\bfx = f(t,\bfx)
\end{equation}
%
在一个有限的时间间隔 $\Dt$上,为了将它们转换成一个差分方程,即,
%
\begin{equation}
\bfx(t+\Dt) =  \bfx(t)+\int_t^{t+\Dt}f(\tau,\bfx(\tau))d\tau~,
\end{equation}
%
或者等效地,如果我们假设 $t_n = n\Dt$ 和 $\bfx_n\triangleq \bfx(t_n)$ ,
\begin{equation}
\bfx_{n+1} =  \bfx_n+\int_{n\Dt}^{(n+1)\Dt}f(\tau,\bfx(\tau))d\tau~.
\end{equation}

最常用的方法之一是 Runge-Kutta 方法 (从现在开始简写为 RK)。
这些方法使用多次迭代来估计区间上的导数,然后使用此导数在步骤 $\Dt$上积分。


在接下来的章节中,我们将介绍几种 RK 法,从最简单的方法到最一般的方法,并根据它们最常用的名称来命名。 


\bigskip
注意:这里所有的材料都来自英文维基百科中的 \emph{Runge-Kutta method} 条目。

%=============================================================
\subsection{欧拉法}
\label{sec:Euler}

欧拉方法假设导数 $f(\cdot)$ 在区间上是常数,因此
%
\begin{equation}
\eqbox{\bfx_{n+1} =  \bfx_n + \Dt\tdot f(t_n,\bfx_n)~.}
\end{equation}

作为一种通用的 RK 方法,它对应于一个单阶段方法,可以描述如下。 
计算初始点的导数,
%
\begin{equation}
k_1 = f(t_n,\bfx_n)~,
\end{equation}
%
用它来计算终点的积分值,
%
\begin{equation}
\bfx_{n+1} = \bfx_n + \Dt\tdot k_1~.
\end{equation}

%=============================================================
\subsection{中点法}

中点法假定导数是区间中点处的导数,并进行一次迭代以计算该中点处 $\bfx$ 的值,即:,
%
\begin{equation}
\eqbox{\bfx_{n+1} =  \bfx_n + \Dt\tdot f \Big( t_n + \frac{1}{2}\Dt \ ,\ \bfx_n + \frac{1}{2} \Dt \tdot f(t_n , \bfx_n) \Big)}~.
\end{equation}

中点法可以解释为两步法,如下所示。 
首先,使用 Euler 方法积分到中点,使用前面定义的 $k_1$ ,
%
%
\begin{align}
k_1 &= f(t_n,\bfx_n) \\
\bfx(t_n+\tfrac12\Dt) &=  \bfx_n + \frac{1}{2} \Dt\tdot k_1~.
\end{align}%
%
然后使用该值计算中点 $k_2$ 处的导数,从而得出积分
%
%
\begin{align}
k_2 &= f(t_n+\tfrac12\Dt \ ,\ \bfx(t_n+\tfrac12\Dt)) \\
\bfx_{n+1} &=  \bfx_n + \Dt\tdot k_2~.
\end{align}%



%=============================================================
\subsection{RK4 方法}

这通常被简单地称为 Runge-Kutta 方法。 
它假定在间隔的开始、中点和结束处 $f()$ 的计算值。 
并且它使用四个阶段或迭代来计算积分,四个导数 $k_1\dots k_4$ 是按顺序得到的。 
然后这些导数或斜率(\emph{slopes})被加权平均,以获得区间内导数的四阶估计。 

RK4 方法更适合指定为一个小算法,而不是像上述两种方法一样,是一步计算公式。RK4 积分步骤是,

\begin{equation}
\eqbox{\bfx_{n+1} = \bfx_n + \frac{\Dt}{6} \Big( k_1 + 2k_2 + 2k_3 + k_4 \Big) }~,
\end{equation}
% 
也就是说,增量是通过假设一个斜率来计算的,该斜率是 $k_1,k_2,k_3,k_4$ 斜率的加权均值,其中
%
%
\begin{align}
k_1 &= f(t_n, \bfx_n) \\
k_2 &= f\Big(t_n + \frac{1}{2}\Dt \ ,\ \bfx_n + \frac{\Dt}{2} k_1\Big) \\
k_3 &= f\Big(t_n + \frac{1}{2}\Dt \ ,\ \bfx_n + \frac{\Dt}{2} k_2\Big) \\
k_4 &= f\Big(t_n + \Dt \ ,\ \bfx_n + \Dt \tdot k_3\Big) ~.
\end{align}%

不同的斜率有以下解释:
%
\begin{itemize}
\item $k_1$ 是间隔开始时的斜率,使用 $\bfx_n$ ,(欧拉法);
\item   $ k_2$ 是间隔中点处的斜率,使用 $\bfx_n + \tfrac12 \Dt\tdot k_1$,(中点法);
\item    $k_3$ 再次是中点处的斜率,但现在使用 $\bfx_n + \tfrac12 \Dt\tdot k_2$ ;
\item    $k_4$ 是间隔结束时的斜率,使用 $\bfx_n + \Dt\tdot k_3 $ 。


\end{itemize}

%=============================================================
\subsection{一般 Runge-Kutta 法}

更详细的 RK 方法是可能的。 
它们的目标是减少误差和/或提高稳定性。 
它们采取一般形式
%
\begin{equation}
\eqbox{\bfx_{n+1} = \bfx_n + \Dt \sum_{i=1}^s b_i k_i} ~,
\end{equation}
%
其中
%
\begin{equation}
k_i = f\Big( t_n+\Dt \tdot c_i ,  \bfx_n + \Dt \sum_{j=1}^s a_{ij}  k_j \Big)~,
\end{equation}
%
也就是说,迭代次数 (方法的顺序) 是 $s$,平均权重由 $b_i$定义,计算时间由 $c_i$定义,而斜率 $k_i$ 是使用 $a_{ij}$ 值确定的。 
依赖于项 $a_{ij}$ 的结构,可以有显式(\emph{explicit})或隐式(\emph{implicit})的 RK 方法。 

\begin{itemize}
\item 在显式方法中,所有 $k_i$ 均按顺序计算,即仅使用先前计算的值。 
这意味着矩阵 $[a_{ij}]$ 是具有零对角线项的下三角形 (即,对于 $j\ge i$ 有 $a_{ij}=0$ )。
Euler 法、中点法和 RK4 方法是显式的。

\item 隐式方法有一个完整的 $[a_{ij}]$ 矩阵,需要解一组线性方程来确定所有 $k_i$ 。 
因此,它们的计算成本更高,但相对于显式方法,它们能够提高精度和稳定性。
\end{itemize}

有关详细信息,请参阅专门文档。
