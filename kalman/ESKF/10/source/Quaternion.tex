% !TEX root = kinematics.tex


%%%%%%%%%%%%%%%%%%%%%%%%%%%%%%%%%%%%%%%%%%%%%%%%%%%%%%%%%%%%%%
\section{四元数定义和属性}

%=============================================================
\subsection{四元数的定义}

我找到的特别有吸引力的四元数的一个介绍是 Cayley-Dickson 给出的构造:
如果我们有两个复数 $A=a+bi$ 和 $C=c+di$,那么构造 $Q=A+Cj$ 并定义 $k\triangleq ij$ ,得到四元数 $\bbH$ 的空间中的一个数,
%
\begin{align}
Q = a + bi + cj + dk \in\bbH ~, \label{equ:ijkQuat}
\end{align}%
%
其中 $\{a,b,c,d\}\in\bbR$,和 $\{i,j,k\}$ 被定义为三个虚数。并且有
%
\begin{subequations}
\label{equ:quatAlgebra}
\begin{align}
i^2=j^2=k^2=ijk=-1~,
\end{align}%
%
从中我们可以得到
%
\begin{align}
ij = -ji = k ~, \quad jk=-kj=i~, \quad ki=-ik=j~.
\end{align}
\end{subequations}
%
从方程 \eqRef{equ:ijkQuat} 我们可以看出,我们可以嵌入复数数在实数和虚数中,因此在四元数的定义中,实数、虚数和复数数确实是四元数的意义上的子集,
%
\begin{align}
Q = a \in \bbR \subset \bbH~,
\Quad 
Q=bi \in \bbI \subset \bbH~,
\Quad 
Q=a+bi \in \bbZ \subset \bbH~.
\end{align}
%
同样,为了完备性,我们可以定义 $\bbH$ 的 3D 虚子空间中的数。
我们称它们为纯四元数(\emph{pure quaternions}),并且可以注意到 $\bbH_p=\Im(\bbH)$ 是纯四元数的空间,
%
\begin{align}
Q=bi+cj+dk \in\bbH_p \subset\bbH~.
\end{align}


值得注意的是,虽然单位长度的正则复数 $\bfz=e^{i\theta}$ 可以编码 2D 平面中的旋转 (用一个复数乘法,$\bfx'=\bfz\tdot\bfx$),“扩展复数”或单位长度四元数 $\bfq=e^{(u_xi+u_yj+u_zk)\theta/2}$ 编码 3D 空间中的旋转 (用两个四元数乘积, $\bfx'=\bfq\otimes\bfx\otimes\bfq^*$,正如我们在本文档后面解释的那样)。



\bigskip

{\bf 注意:} 并非所有四元数定义都相同。 
有些作者把乘积写成 $ib$ 而不是 $bi$,因此得到了 $k = ji = -ij$的性质,这就得到了 $ijk=1$ 的结果和左手四元数。 
此外,许多作者将实数部分放在最后位置,得到 $Q = ia + jb + kc + d$。
这些选择没有基本内涵差异,但使整个公式在细节上有所不同。 
请参阅 \secRef{sec:conventions} 了解进一步的解释和消除歧义。

\bigskip

{\bf 注意:} 还有一些附加的约定也使得公式在细节上有所不同。 
它们涉及我们给予旋转算子的“意义”或“解释”,即旋转向量或旋转参考系 --- 它们本质上构成相反的操作。 
请参阅 \secRef{sec:conventions} 得到有关进一步的解释和消除歧义。

\bigskip 

{\bf 注:} 在以上公开的不同约定中,本文集中讨论了 Hamilton 约定,其最显著的特性是定义方程 \eqRef{equ:quatAlgebra}。正确和有根据的消除歧义首先需要扩展大量的资料;因此,这个消除歧义内容被归入上述提到的 \secRef{sec:conventions}。


%=============================================================
\subsubsection{四元数的替代表示}
\label{sec:altQuat}

实数 + 虚数符号 $\{1,i,j,k\}$ 对于我们的目的并不总是方便的。 
%
%
如果使用代数方程 \eqRef{equ:quatAlgebra} ,四元数可以设为标量 + 向量的和,
%
\begin{align}
Q=q_w+q_xi+q_yj+q_zk
\qquad
\Leftrightarrow
\qquad
Q = q_w + \qv~,
\end{align}
%
其中 $q_w$ 被称为实数(\emph{real})或标量(\emph{scalar})部分,并且 $\qv=q_x i+q_y j+q_z k=(q_x,q_y,q_z)$ 作为虚数(\emph{imaginary})或向量(\emph{vector})部分。\footnote{\label{ftn:quatComponents}我们选择 $(w,x,y,z)$ 下标符号是因为我们对 3D 笛卡尔空间中四元数的几何性质感兴趣。 
其它文档通常使用另外的下标,如 $(0,1,2,3)$ 或 $(1,i,j,k)$,可能更适合数学解释。}%
它也可以定义为标量-向量(scalar-vector)有序对。 
%
\begin{align}
Q = \langle q_w,\qv\rangle ~.
\end{align}
%
我们通常把四元数 $Q$ 表示为4参数向量(4-vector) $\bfq$~,
%
\begin{align}
\bfq \triangleq 
\begin{bmatrix}
q_w\\\qv
\end{bmatrix}=
\begin{bmatrix}
q_w\\q_x\\q_y\\q_z
\end{bmatrix}~,
\end{align}%
%
这允许我们使用矩阵代数来处理四元数运算。
在某些情况下,我们可以通过滥用 ``$=$'' 标志来混合符号。典型的例子是实四元数(\emph{real quaternions})和纯四元数(\emph{pure quaternions}),
%
\begin{align}
\textrm{general: }
\bfq
=q_w+\qv=\begin{bmatrix}
q_w\\\qv
\end{bmatrix} \in \bbH
~,\quad
\textrm{real: }
q_w=\begin{bmatrix}
q_w\\{\bf0}_v
\end{bmatrix} \in \bbR
~,\quad
\textrm{pure: }
\qv=\begin{bmatrix}
0\\\qv
\end{bmatrix} \in \bbH_p
~.
\end{align}




%=============================================================
\subsection{四元数的主要性质}

\subsubsection{加法}

加法很简单,
%
\begin{align}
\bfp\pm\bfq = \begin{bmatrix}
p_w \\ \pv
\end{bmatrix} \pm \begin{bmatrix}
q_w \\ \qv
\end{bmatrix}
  = \begin{bmatrix}
p_w \pm q_w \\ \pv \pm \qv
\end{bmatrix}~.
\end{align}
%
通过构造,加法是可交换的(\textbf{commutative})和可结合的(\textbf{associative}),
%
%
\begin{align}
\bfp+\bfq&=\bfq+\bfp \\
\bfp+(\bfq+\bfr)&=(\bfp+\bfq)+\bfr
~.
\end{align}%
%
%Thus, the set of quaternions endowed with the sum operation form a commutative group, where the identity is the zero quaternion, $\bfq_0 = 0$, and the inverse is the negative  $-\bfq$.

\subsubsection{乘法}

乘法用 $\otimes$ 符号表示,四元数乘积需要使用原始形式方程 \eqRef{equ:ijkQuat} 和四元数代数方程 \eqRef{equ:quatAlgebra}。
将结果以向量形式写入
%
\begin{align}
\bfp\otimes\bfq = \begin{bmatrix}
p_wq_w - p_{x}q_{x} - p_{y}q_{y} - p_{z}q_{z} \\
p_wq_{x} + p_{x}q_w + p_{y}q_{z} - p_{z}q_{y} \\
p_wq_{y} - p_{x}q_{z} + p_{y}q_w + p_{z}q_{x} \\
p_wq_{z} + p_{x}q_{y} - p_{y}q_{x} + p_{z}q_w  
\end{bmatrix}~. \label{equ:quatProd}
\end{align}
%
这也可以用标量和向量部分来表示,
%
\begin{align}
\bfp\otimes\bfq = \begin{bmatrix}
p_wq_w - \pv\tr\qv \\
p_w\qv+q_w\pv+\pv\!\times\!\qv
\end{bmatrix}~, \label{equ:quatProdVec}
\end{align}
%
其中
交叉积的存在
表明四元数乘积在一般情况下是不可交换的(\textbf{not commutative}),
%
\begin{align}
\bfp\otimes\bfq\neq\bfq\otimes\bfp~.
\end{align}
%
这种一般非交换性的例外仅限于 $\pv\!\times\!\qv=0$ 当其中一个四元数是实数(实四元数)的情况, $\bfp=p_w$ 或 $\bfq=q_w$;或者当两个向量部分是平行的情况, $\pv \| \qv$。只有在这些情况下,四元数乘积才是可交换的。

四元数乘积是可结合的(\textbf{associative}),
%
\begin{align}
(\bfp\otimes\bfq)\ot\bfr = \bfp\otimes(\bfq\ot\bfr)~,
\end{align}
%
并且符合分配律(\textbf{distributive over the sum}),
%
\begin{align}
\bfp\ot(\bfq+\bfr) = \bfp\ot\bfq + \bfp\ot\bfr
\qquad \textrm{and} \qquad
%\qquad,\qquad
(\bfp+\bfq)\ot\bfr = \bfp\ot\bfr + \bfq\ot\bfr~.
\end{align}
%



两个四元数的乘积是双线性的,可以表示为两个等价的矩阵乘积,即
%
\begin{align}
\bfq_1\ot\bfq_2 = \QL{\bfq_1}\,\bfq_2 
\qquad \textrm{and} \qquad
\bfq_1\ot\bfq_2 = \QR{\bfq_2}\,\bfq_1 ~, \label{equ:quatMatProd}
\end{align}%
%
其中 $\QL{\bfq}$ 和 $\QR{\bfq}$ 分别是四元数左乘(left-product)矩阵和四元数右乘(right-product)矩阵,由方程 \eqRef{equ:quatProd} 和 \eqRef{equ:quatMatProd} 通过简单检验导出,
%
\begin{align}
\QL{\bfq} = \begin{bmatrix}
q_w &-q_x &-q_y &-q_z\\
q_x & q_w &-q_z & q_y\\
q_y & q_z & q_w &-q_x\\
q_z &-q_y & q_x & q_w\\
\end{bmatrix}, \qquad
\QR{\bfq} = \begin{bmatrix}
q_w &-q_x &-q_y &-q_z\\
q_x & q_w & q_z &-q_y\\
q_y &-q_z & q_w & q_x\\
q_z & q_y &-q_x & q_w\\
\end{bmatrix} ,
\label{equ:quatMatrixComponents}
\end{align}%
%
或者更简洁地说,从方程 \eqRef{equ:quatProdVec} 和 \eqRef{equ:quatMatProd} 开始导出,
%
\begin{align}
\QL{\bfq} = q_w\,\bfI + \begin{bmatrix}
0 & -\qv\tr \\
\qv & \hatx{\qv}
\end{bmatrix}, \qquad
\QR{\bfq} = q_w\,\bfI + \begin{bmatrix}
0 & -\qv\tr \\
\qv & -\hatx{\qv}
\end{bmatrix}~.
\label{equ:quatMatrix}
 ~
\end{align}%
%
这里,斜交算子(\emph{skew operator})\footnote{斜交算子可以在文献中以许多不同的名称和符号找到,或者与交叉算子 $\times$ 相关,或者与帽子“hat”算子 $^\wedge$ 相关,因此下面的所有形式都是等效的,
$$
\hatx{\bfa} \equiv [\bfa_\times] \equiv \bfa\!\times \equiv \bfa_\times \equiv [\bfa] \equiv \widehat{\bfa} \equiv \bfa^\wedge~.
$$
} 
%
$\hatx{\bullet}$ 产生交叉乘积矩阵,
%
\begin{align}
\hatx{\bfa} \triangleq \begin{bmatrix}
0 & -a_z & a_y \\
a_z & 0 & -a_x \\
-a_y & a_x & 0
\end{bmatrix}
\label{equ:skew}
~,
\end{align}
%
这是一个斜对称矩阵, $\hatx{\bfa}\tr=-\hatx{\bfa}$,相当于叉积,即, 
%
\begin{align}
\hatx{\bfa}\bfb = \bfa\tcross\bfb~,\quad \forall\, \bfa,\bfb\in\bbR^3 ~.  
\end{align}



最后,因为
%
\begin{align}
(\bfq\ot\bfx)\ot\bfp = \QR{\bfp}\,\QL{\bfq}\,\bfx 
\qquad \textrm{and} \qquad
\bfq\ot(\bfx\ot\bfp) = \QL{\bfq}\,\QR{\bfp}\,\bfx
~,
\end{align}
%
我们有关系
%
\begin{align}
\QR{\bfp}\,\QL{\bfq} = \QL{\bfq}\,\QR{\bfp}~,
\label{equ:PQ_commute}
\end{align}
%
这就是左乘和右乘四元数矩阵是可交换的。
这些矩阵的进一步性质在 \secRef{sec:isoclinic} 提供。


赋予乘积运算 $\otimes$ 的四元数构成一个非交换群。 
群的元素特征值 $\bfq_1=1$ 和逆式 $\bfq\inv$ 在下面进行探讨。

\subsubsection{特征值}

特征四元数 $\qI$ 相对于乘积是这样 $\qI\otimes\bfq=\bfq\otimes\qI=\bfq$。
它对应于表示为四元数的实数乘积特征值 `1' 。
%
\begin{align*}
\qI = 1 = \begin{bmatrix}
1 \\ {\bf0}_v
\end{bmatrix} ~.
\end{align*}


\subsubsection{共轭}

四元数的共轭定义为
%
\begin{align}
\bfq^*\triangleq q_w-\qv=\begin{bmatrix}
q_w \\ -\qv
\end{bmatrix}
~.
\end{align}
%
有属性为
%
\begin{align}
\bfq\otimes\bfq^* = \bfq^*\otimes\bfq 
=q_w^2 +q_x^2 +q_y^2 +q_z^2
= \begin{bmatrix}
q_w^2 +q_x^2 +q_y^2 +q_z^2 \\ {\bf0}_v
\end{bmatrix}
~,
\end{align}
%
和
%
\begin{align}
(\bfp\ot\bfq)^* = \bfq^*\ot\bfp^* 
~.
\end{align}%


\subsubsection{范数}

四元数的范数定义为
%
\begin{align}\label{equ:q_norm}
\norm{\bfq} \triangleq \sqrt{\bfq\otimes\bfq^*} = \sqrt{\bfq^*\otimes\bfq} = \sqrt{q_w^2 +q_x^2 +q_y^2 +q_z^2 } ~\in \bbR ~.
\end{align}
%
它有属性
%
\begin{align}
\norm{\bfp\ot\bfq} = \norm{\bfq\ot\bfp} = \norm{\bfp}\norm{\bfq}~. \label{equ:norm_prod}
\end{align}•

\subsubsection{逆}

逆四元数 $\bfq\inv$ 使得四元数乘以其逆得到特征值,
%
\begin{align}
\bfq\otimes\bfq\inv = \bfq\inv\otimes\bfq = \qI~.
\end{align}
%
它可以用下式计算
%
\begin{align}
\bfq\inv = \bfq^*/\norm{\bfq}^2~.
\end{align}

\subsubsection{单位或规范化四元数}

对于单位四元数, $\norm{\bfq}=1$,因此
%
\begin{align}
\bfq\inv = \bfq^*~.
\end{align}


当将单位四元数解释为方向规范或旋转算子时,此属性意味着可以使用共轭四元数完成逆旋转。单位四元数总是可以用这种形式写为,
%
\begin{align}
\bfq = \begin{bmatrix}
\cos\theta \\ \bfu\sin\theta
\end{bmatrix}
~,
\end{align}
%
其中 $\bfu = u_x i + u_y j + u_z k$ 是单位向量,并且 $\theta$ 是标量。 

从方程 \eqRef{equ:norm_prod} 开始,赋予乘积操作 $\otimes$ 形式的单位四元数形成一个非交换群,其中逆与共轭重合。 


\subsection{附加四元数属性}

\subsubsection{四元数交换器}

四元数交换器(\emph{commutator})定义为 $[\bfp,\bfq]\triangleq\bfp\ot\bfq-\bfq\ot\bfp$。我们从方程 \eqRef{equ:quatProdVec} 得到,
%
\begin{align}
\bfp\ot\bfq-\bfq\ot\bfp = 2\,\pv\tcross\qv
~.
\label{equ:quatCommutator}
\end{align}
%
这有一个平凡的结果,
%
\begin{align}
\pv\ot\qv-\qv\ot\pv = 2\,\pv\tcross\qv
~.
\label{equ:quatCommutatorPure}
\end{align}
%
稍后我们将使用此属性。


\subsubsection{纯四元数的乘积}

纯四元数是那些实部或标量部为零的四元数, $Q=\qv$ 或 $\bfq=[0,\qv]$。 我们从方程 \eqRef{equ:quatProdVec} 得到,
%
\begin{align}
\pv\ot\qv 
= -\pv\tr\qv + \pv\tcross\qv
= \begin{bmatrix}
-\pv\tr\qv \\
\pv\tcross\qv
\end{bmatrix}~.
\label{equ:quatProdPure}
\end{align}
%
这意味着
%
\begin{align}
\qv\ot\qv = -\qv\tr\qv = -\norm{\qv}^2~,
\label{equ:quatPureSquared}
\end{align}
%
并且对于纯酉四元数 $\bfu\in\bbH_p,~ \norm{\bfu}=1$,
%
\begin{align}
\bfu\ot\bfu = -1
~,
\end{align}
%
这类似于标准的假想情况, $i\cdot i=-1$.

\subsubsection{纯四元数的自然数幂}

让我们用四元数乘积 $\ot$定义 $\bfq^n, ~n\in\bbN$ 为 $\bfq$ 的 $n$次幂。
然后,如果 $\bfv$ 是纯四元数,我们让 $\bfv=\bfu\,\theta$,用 $\theta=\norm{\bfv}\in\bbR$ 和 $\bfu$ 的,我们从方程 \eqRef{equ:quatPureSquared} 得到循环模式
%
\begin{align}
\bfv^2 = -\theta^2 \quad,\quad
\bfv^3 = -\bfu\,\theta^3 \quad,\quad
\bfv^4 = \theta^4 \quad,\quad
\bfv^5 = \bfu\,\theta^5 \quad,\quad
\bfv^6 = -\theta^6 \quad,\quad
\cdots
\label{equ:qvPowers}
\end{align}
%
%并且对于纯酉四元数 $\bfu$,这将减少到模式
%%
%\begin{align}
%\bfu^2 = -1 	\quad,\quad
%\bfu^3 = -\bfu 	\quad,\quad
%\bfu^4 = 1 		\quad,\quad
%\bfu^5 = \bfu 	\quad,\quad
%\bfu^6 = -1 	\quad,\quad
%\cdots
%\label{equ:uPowers}
%\end{align}




\subsubsection{纯四元数的指数}

%我们用实数指数的一个推广定义了一般四元数的指数,目的是捕捉其大部分性质。
四元数指数是四元数上类似于普通指数函数的函数。 
与实数指数情形一样,它被定义为绝对收敛幂级数,
%
\begin{align}
e^\bfq
\triangleq \sum_{k=0}^\infty \frac{1}{k!}\bfq^k \quad \in \bbH
~.
\label{equ:quatExpSeries}
\end{align}
%
显然,实四元数的指数与普通指数函数完全一致。 

更有趣的是,纯四元数 $\bfv=v_xi+v_yj+v_zk$~ 的指数是一个新的四元数,定义为
%
\begin{align}
e^\bfv
= \sum_{k=0}^\infty \frac{1}{k!}\bfv^k \quad \in \bbH
~.
\label{equ:pureQuatExpSeries}
\end{align}
%
设 $\bfv=\bfu\,\theta$,用 $\theta=\norm{\bfv}\in\bbR$ 和 $\bfu$ 的酉,考虑方程 \eqRef{equ:qvPowers} ,我们将级数中的标量项和向量项分组, 
%
\begin{align}
e^{\bfu\theta} 
&= \left(1-\frac{\theta^2}{2!}+\frac{\theta^4}{4!}+\cdots\right) + \left(\bfu\theta - \frac{\bfu\theta^3}{3!}+\frac{\bfu\theta^5}{5!}+\cdots\right)
\end{align}
%
从中分别识别出 $\cos\theta$ 和 $\sin\theta$的级数。
% 
%
\footnote{我们提示 $\cos \theta = 1 - \theta^2/2! + \theta^4/4! - \cdots$,和 $\sin \theta =  \theta - \theta^3/3! + \theta^5/5! - \cdots$。}
%
其结果为
%
\begin{align}
e^\bfv
= e^{\bfu\,\theta} 
= \cos\theta + \bfu\sin\theta = \begin{bmatrix}
\cos\theta \\ \bfu\sin\theta 
\end{bmatrix} ~,
\label{equ:EulerFormulaQuat}
\end{align}
%
这是 Euler 公式的一个漂亮的扩展, $e^{i\theta}=\cos\theta+i\sin\theta$,定义为虚数。 
%
注意,由于 $\norm{e^{\bfv}}^2=\cos^2\theta+\sin^2\theta=1$,纯四元数的指数是一个单位四元数。
还要注意性质,
%
\begin{align}
e^{-\bfv} = \left(e^{\bfv}\right)^*
~.
\end{align}

对于小角度四元数,在方程 $\bfu=\bfv/\norm{\bfv}$ 中,我们通过 $\sin\theta$ 和 $\cos\theta$ 的泰勒级数表达和截断的级数,得到了不同程度的近似,避免了除以零的问题。
%
\begin{align}
e^\bfv 
\approx
\begin{bmatrix}
1-\theta^2/2 \\ \bfv\big(1-\theta^2/6\big) 
\end{bmatrix}
\approx
\begin{bmatrix}
1 \\ \bfv 
\end{bmatrix}
\xrightarrow[\theta\to 0]{}
\begin{bmatrix}
1 \\ {\bf0}
\end{bmatrix}
~.
\end{align}
%


\subsubsection{一般四元数的指数}


由于四元数乘积的非交换性,我们不能为一般的四元数 $\bfp$ 和 $\bfq$ 写出 $e^{\bfp+\bfq}=e^\bfp e^\bfq $。然而,当任一乘积成员是标量时,交换性成立,因此,
%
\begin{align}
e^\bfq = e^{q_w+\qv} = e^{q_w}\,e^{\qv} ~.
\end{align}
%
然后,对 $\bfu\theta=\qv$ 使用方程 \eqRef{equ:EulerFormulaQuat} ,我们得到
%
\begin{align}\label{equ:expGeneralQuat}
e^\bfq 
= e^{q_w}\begin{bmatrix}
\cos\norm{\qv} \\ \frac{\qv}{\norm{\qv}}\sin\norm{\qv} 
\end{bmatrix}~.
\end{align}
%


\subsubsection{单位四元数的对数}
\label{sec:qlog}

很快就能看出,如果 $\norm{\bfq}=1$,
%
\begin{align}\label{equ:qlog}
\log \bfq = \log (\cos \theta + \bfu \sin\theta) = \log(e^{\bfu\,\theta}) = \bfu\,\theta = \begin{bmatrix}
0 \\ \bfu\,\theta
\end{bmatrix}
~,
\end{align}
%
也就是说,单位四元数的对数是纯四元数。通过反转方程 \eqRef{equ:EulerFormulaQuat} 容易获得角-轴(angle-axis)数值,
%
\begin{align}
\bfu &= \qv / \norm{\qv} \\
\theta &= \arctan(\norm{\qv},q_w)
~.
\end{align}
%
对于小角度四元数,我们通过 $\arctan(x)$ 的泰勒级数表达和截断,\footnote{我们提示 $\arctan x = x - x^3/3 + x^5/5 - \cdots$,和 $\arctan(y,x)\equiv\arctan(y/x)$。} 获得不同程度的近似,避免了除以零的问题。
%
\begin{align}
\log(\bfq) 
= \bfu\theta 
&= \qv\frac{\arctan({\norm{\qv},q_w})}{\norm{\qv}} 
\approx \frac{\qv}{q_w} \left(1 - \frac{\norm{\qv}^2}{3q_w^2}\right)
\approx \bfq_v
\xrightarrow[\theta\to 0]{}
{\bf 0}
~.
\end{align}
%
%其中,对于 $\theta\to 0$ 的极限,趋向于 $\log(\bfq)=\bfq_v$.



\subsubsection{一般四元数的对数}

通过扩展,如果 $\bfq$ 是一般的四元数,
%
\begin{align}
\log\bfq = \log(\norm{\bfq}\frac{\bfq}{\norm{\bfq}}) = \log\norm{\bfq} + \log\frac{\bfq}{\norm{\bfq}} = \log\norm{\bfq} + \bfu\,\theta = \begin{bmatrix}
\log\norm{\bfq} \\ \bfu\,\theta
\end{bmatrix}
~.
\end{align}

\subsubsection{ $\bfq^t$ 型的指数形式}

对于 $\bfq\in\bbH$ 和 $t\in\bbR$,我们有,
%
\begin{align}
\bfq^t = \exp(\log(\bfq^t)) = \exp(t\log(\bfq))
~.
\end{align}
%
如果 $\norm{\bfq}=1$,我们可以写出 $\bfq=[\cos\theta,~\bfu\sin\theta]$,因此 $\log(\bfq)=\bfu\theta$,这给出
%
\begin{align}\label{equ:qa}
\bfq^t = \exp(t\,\bfu\theta)=\begin{bmatrix}
\cos t\theta \\
\bfu\sin t\theta
\end{bmatrix}
~.
\end{align}
%
因为指数 $t$ 最终是角 $\theta$的线性乘子,所以它可以看作是一个线性角度插值器。我们将在 \secRef{sec:slerp} 中提出这个想法。

%=============================================================
\section{旋转和交叉关系}
\label{sec:rotations}

%-------------------------------------------------------------
\subsection{3D 向量旋转公式}

\begin{figure}[htbp]
\centering
\includegraphics{figures/rotation3d}
\caption{向量 $\bfx$ 围绕轴 $\bfu$ 旋转 $\phi$ 角度,详情见正文。}
\label{fig:rotation3d}
\end{figure}

我们在 ~\figRef{fig:rotation3d} 中的旋转,遵循右手规则,说明了一个通用的 3D 向量 $\bfx$,围绕用单位向量定义的旋转轴 $\bfu$,旋转 $\phi$ 角度。 
这是通过将向量 $\bfx$ 分解成与 $\bfu$ 平行的分量 $\bfx_{||}$ ,和与 $\bfu$ 正交的分量 $\bfx_\bot$ 来完成,所以有
%
%
\begin{align*}
\bfx=\bfx_{||}+\bfx_\bot~. 
\end{align*}
%
这些部分可以很容易地计算出来 ($\alpha$ 是向量 $\bfx$ 与轴 $\bfu$ 之间的夹角),
%
%
\begin{align*}
\bfx_{||} &= \bfu \, (\norm{\bfx}\cos\alpha)  = \bfu\,\bfu\tr\,\bfx 
\\
\bfx_\bot &= \bfx - \bfx_{||} = \bfx - \bfu\,\bfu\tr\,\bfx~.
\end{align*}%
%
旋转时,平行部分不旋转, 
%
\begin{align*}
\bfx_{||}' = \bfx_{||}~,
\end{align*}
%
并且正交部分在垂直于 $\bfu$ 的平面上经历平面旋转。也就是,如果基于这个平面创建一个正交基 $\{\bfe_1,\bfe_2\}$ 
%
%
\begin{align*}
\bfe_1 &= \bfx_\bot \\
\bfe_2 &= \bfu \tcross \bfx_\bot = \bfu \tcross \bfx  ~, 
\end{align*}%
%
满足 $\norm{\bfe_1} = \norm{\bfe_2}$,则 $\bfx_\bot=\bfe_1\tdot1+\bfe_2\tdot0$。 一个 $\phi$\,rad 的旋转出现在这个平面上。
%
\begin{align*}
\bfx_\bot' = \bfe_1\cos\phi + \bfe_2\sin\phi~,
\end{align*}
%
可扩展为,
%
\begin{align*}
\bfx_\bot' = \bfx_\bot\cos\phi + (\bfu\tcross\bfx)\sin\phi~.
\end{align*}
%
将平行部分相加,得到旋转向量 $\bfx'=
\bfx'_{||}+\bfx'_\bot$~ 的表达式,即著名的向量旋转公式(\emph{vector rotation formula}),
%
\begin{align}
\eqbox{
\bfx'=
\bfx_{||}+\bfx_\bot\cos\phi+(\bfu\times\bfx)\sin\phi}~.
\label{equ:vecRotFormula}
\end{align}
%


%-------------------------------------------------------------
\subsection{旋转群 $SO(3)$}

在 $\bbR^3$ 中,旋转群 $SO(3)$ 是在组合操作下围绕原点旋转的群。旋转是保持向量长度和相对向量方向 (即,惯用手,handedness)。它在机器人学中的重要性在于,它代表了3D空间中刚体的旋转:刚体运动(\emph{rigid motion})要求在运动时精确地保持刚体内的距离、角度和相对方向 --- 否则,如果范数、角度或相对方向能不保持,则不能将机体视为刚体。

%刚体(\emph{Rigid})意味着它们维持了距离、角度和相对方向。
%不变的点不包括转换。
然后,让我们通过满足这些属性的算子定义旋转。 
一个作用于向量 $\bfv\in\bbR^3$ 的旋转算子 $r:\bbR^3\to\bbR^3; \bfv\mapsto r(\bfv)$ 可以从欧几里德空间的度量定义,由点和交叉积构成,如下所示。
%
% $SO(3)$ 群,或 $\bbR^3$ 中的旋转群,是算子 $r:\bbR^3\to\bbR^3$ 的集合,满足以下特性:
%
\begin{itemize}
%
\item 旋转保持向量范数,
%
\begin{subequations}
\begin{align}
\norm{r(\bfv)}=
\sqrt{\langle r(\bfv), r(\bfv)\rangle}=\sqrt{\langle\bfv,\bfv\rangle}\triangleq\norm{\bfv}~,\quad\forall\bfv \in\bbR^3~. \label{eq:keepnorm}
%\sqrt{r(\bfv)\tr r(\bfv)}=\sqrt{\bfv\tr\bfv}\triangleq\norm{\bfv}~,\quad\forall\bfv \in\bbR^3~. \label{eq:keepnorm}
\end{align}
%
\item 旋转保持向量之间的角度, 
%
\begin{align}
\langle r(\bfv), r(\bfw)\rangle  = \langle \bfv,\bfw\rangle = \norm{\bfv}\norm{\bfw}\cos\alpha~,\quad\forall\bfv,\bfw\in\bbR^3~.
%r(\bfv)\tr r(\bfw) = \bfv\tr\bfw = \norm{\bfv}\norm{\bfw}\cos\alpha~,\quad\forall\bfv,\bfw\in\bbR^3~.
\end{align}
\end{subequations}
%
\item 旋转保持向量的相对方向,
%
\begin{align}
\bfu\times\bfv=\bfw 
\iff 
r(\bfu)\times r(\bfv)=r(\bfw) 
~. 
\label{equ:keeporientation}
\end{align}
\end{itemize}
%
很容易证明前两个条件是等价的。因此我们可以把旋转群 $SO(3)$ 定义为,
%旋转集合作为3参数(3-vectors)向量上的算子集合有 (a) 保持范数不变,并且 (b) 维持相对方向,也就是说,
%
\begin{align}
SO(3):\{r:\bbR^3\to\bbR^3\,/\,\forall\, \bfv,\bfw\in\bbR^3~,~ \norm{r(\bfv)}=\norm{\bfv}~,~ r(\bfv)\tcross r(\bfw)=r(\bfv\tcross\bfw)\} 
~.
\end{align}
%

旋转群通常由旋转矩阵集合表示。然而,四元数也是它的一个很好的表示。 
本章的目的是证明这两种表述是同等有效的。
它们在概念上和代数上表现出许多相似之处,读者将在 \tabRef{tab:Rq} 中体会。
%
\begin{table}[htp]
\renewcommand{\arraystretch}{1.5}
\begin{center}
\caption{表示 $SO(3)$ 的旋转矩阵和四元数。}
\label{tab:Rq}
\begin{tabular}{|c|c|c|}
\hline
& 旋转矩阵, $\bfR$ & 四元数, $\bfq$ \\
\hline
\hline
参数 & $3\times3=9$ & $1+3=4$ \\
自由度(DOF) & 3 & 3 \\
约束条件 & $9-3=6$ & $4-3=1$ \\
约束条件 & $\bfR\bfR\tr=\bfI~~;~~\det(\bfR)=+1$ & $\bfq\ot\bfq^* = 1$ \\
\hline
\hline
ODE & $\dot\bfR=\bfR\hatx{\bfomega}$ & $\dot\bfq=\frac12\bfq\ot\bfomega$ \\
指数映射 & $\bfR=\exp(\hatx{\bfu\phi})$ & $\bfq=\exp(\bfu\phi/2)$ \\
对数映射 & $\log(\bfR) = \hatx{\bfu\phi}$ & $\log(\bfq) = \bfu\phi/2$ \\
与 $SO(3)$ 关系 & 单倍覆盖 & 双倍覆盖 \\
\hline
\hline
特征值 & $\bfI$ & $1$ \\
逆 & $\bfR\tr$ & $\bfq^*$ \\
组合 & $\bfR_1\,\bfR_2$  & $\bfq_1\ot\bfq_2$ \\
\hline
旋转算子 & $\bfR = \bfI + \sin\phi\hatx{\bfu} + (1-\cos\phi)\hatx{\bfu}^2$ & $\bfq = \cos\phi/2 + \bfu\sin\phi/2$ \\
旋转动作 & $\bfR\,\bfx$ & $\bfq\ot\bfx\ot\bfq^*$ \\
\hline
\multirow{3}{*}{插值} & $\bfR^t=\bfI + \sin t\phi\hatx{\bfu} \!+ (1\!-\!\cos t\phi)\hatx{\bfu}^2$ & $\bfq^t=\cos t\phi/2+\bfu\sin t\phi/2$\\
 & $\bfR_1(\bfR_1\tr\bfR_2)^t$ & $\bfq_1\ot(\bfq_1^*\ot\bfq_2)^t$ \\
 & & $\bfq_1\frac{\sin((1-t)\Delta\theta)}{\sin(\Delta\theta)}+\bfq_2\frac{\sin(t\Delta\theta)}{\sin(\Delta\theta)}$ \\
\hline
\hline
交叉关系 & \multicolumn{2}{|c|}
{$\begin{aligned}
\rule{0pt}{2.7ex} % add some vertical spacing for cosmetics
\bfR\{\bfq\} &= (q_w^2-\qv\tr\qv)\,\bfI + 2\,\qv\qv\tr + 2\,q_w\hatx{\qv} \\
\bfR\{-\bfq\}&=\bfR\{\bfq\} ~~~~~~~~~~~~~~~~~~~~\, \text{双倍覆盖} \\
\bfR\{1\}&=\bfI ~~~~~~~~~~~~~~~~~~~~~~~~~~~ \text{特征值}\\
\bfR\{\bfq^*\}&=\bfR\{\bfq\}\tr ~~~~~~~~~~~~~~~~~~~ \text{逆}\\
\bfR\{\bfq_1\ot\bfq_2\}&=\bfR\{\bfq_1\}\,\bfR\{\bfq_2\} ~~~~~~~~~~ \text{组合}\\
\bfR\{\bfq^t\} &= \bfR\{\bfq\}^t ~~~~~~~~~~~~~~~~~~~\, \text{插值} 
\end{aligned}$} \\
\hline
\end{tabular}
\end{center}
\end{table}%
%
或许,最重要的区别在于,单位四元数组构成了 $SO(3)$ 双倍覆盖(因此技术上不是 $SO(3)$ 本身),这在我们的大多数应用中并不重要。\footnote{在旋转空间进行插值时,需要考虑双倍覆盖的效果。这个很容易,我们将在 \secRef{sec:slerp} 看到细节。}
为了快速比较和评估,该表是预先插入的。
接下来的章节将探讨 $SO(3)$ 的旋转矩阵和四元数表示。



\subsection{旋转群和旋转矩阵}

%The name $SO(3)$ is explained by the fact that it can be represented by the set of \emph{Special Orthogonal} $3\times3$ matrices, otherwise called \emph{proper orthogonal} matrices, as we explore in this section. 

算子 $r()$ 是线性的,因为它是由线性的标量和向量乘积定义的。 
因此,它可以用矩阵 $\bfR\in\bbR^{3\times3}$来表示,该矩阵在矩阵乘积上对向量 $\bfv\in\bbR^3$ 产生旋转,
%
\begin{align}
r(\bfv) = \bfR\,\bfv~.
\end{align}
%
将其注入方程 \eqRef{eq:keepnorm} ,使用点积 $\langle\bfa,\bfb\rangle=\bfa\tr\bfb$ 和扩展,我们对所有的 $\bfv$ 有,
%
\begin{align}
(\bfR\bfv)\tr(\bfR\bfv) = \bfv\tr\bfR\tr\bfR\bfv = \bfv\tr\bfv
~,
\end{align}
%
得到 $\bfR$ 上的正交(\emph{orthogonality})条件,
%
\begin{align}
\eqbox{
\bfR\tr\bfR = \bfI = \bfR\,\bfR\tr
}
~. 
\label{equ:Rorthogonal}
\end{align}
%
上面的条件确实是正交的条件,因为我们可以从中观察到,通过写 $\bfR=[\bfr_1,\bfr_2,\bfr_3]$ 并代入上面, $\bfR$ 的列向量 $\bfr_i$ ,其中 $i\in\{1,2,3\}$, 是单位长度且相互正交,
%
\begin{align*}
\langle\bfr_i,\bfr_i\rangle &= \bfr_i\tr\bfr_i = 1 \\%~,\quad \textrm{if } i = j \\
\langle\bfr_i,\bfr_j\rangle &= \bfr_i\tr\bfr_j = 0 ~,\quad \textrm{if } i\neq j~.
\end{align*}
%
由于这个原因,保持向量范数和角度的变换集合被称为正交群(\emph{Orthogonal group}),表示为 $O(3)$。
正交群包括旋转 (刚体运动) 和反射 (非刚体运动)。
这里的群(\emph{group})的概念实质上 (和非正式地) 意味着两个正交矩阵的乘积总是一个正交矩阵,%
\footnote{\label{ftn:O3}%
让 $\bfQ_1$ 和 $\bfQ_2$ 正交,
并且构建 $\bfQ=\bfQ_1\,\bfQ_2$ 。
则 $\bfQ\tr\bfQ=
\bfQ_2\tr\bfQ_1\tr\bfQ_1\bfQ_2=\bfQ_2\tr\bfI\bfQ_2=\bfI$ 。}
并且每个正交矩阵允许一个逆。
实际上,正交条件方程 \eqRef{equ:Rorthogonal} 意味着利用转置矩阵实现逆旋转,
%
\begin{align}
\bfR\inv=\bfR\tr~.
\end{align}


添加相对方向条件方程 \eqRef{equ:keeporientation} 可确保刚体运动 (从而丢弃反射运动),并且其结果对 $\bfR$ 产生一个附加约束%
\footnote{注意反射运动满足 $|\bfR|=\det(\bfR)=-1$,并且不构成群,因为 $|\bfR_1\bfR_2|=1\neq-1$ 。}
%
\begin{align}\label{equ:unitDet}
\eqbox{
\det(\bfR)=1
}
~.
\end{align}
%
%
具有正单位行列式的正交矩阵通常被称为真(\emph{proper})矩阵或特殊(\emph{special})矩阵。这类特殊正交矩阵(\emph{special orthogonal matrices})的集合被称为特殊正交群(\emph{Special Orthogonal group})$SO(3)$ ,是 $O(3)$ 的一个子群。
作为一个群,两个旋转矩阵的乘积总是一个旋转矩阵。%
\footnote{%
参见脚注 \ref{ftn:O3} 对于 $O(3)$ 并且对 $SO(3)$ 附加这个条件:让 $|\bfR_1|=|\bfR_2|=1$,则 $|\bfR_1\bfR_2|=|\bfR_1|\,|\bfR_2|=1$。}




\subsubsection{指数映射}

指数映射 (和对数映射,我们在下一节中看到) 是一个强大的数学工具,可以轻松而精确地在旋转的 3D 空间中工作。它表示了适合旋转空间的微积分语料库的入口。指数映射允许我们正确定义导数、扰动和速度,并对它们进行操作。因此,它在旋转或方向空间中的估计问题中是必要的。

旋转构成刚性运动。这种刚性意味着可以在 $SO(3)$ 中定义一个连续的轨迹或路径(\emph{path}), $r(t)$,该轨迹或路径连续地将刚体从其初始方向,$r(0)$,旋转到其当前方向, $r(t)$。
由于变换是连续的,因此研究这种变换的时间导数是合理的。
我们通过推导我们刚刚看到的性质方程 \eqRef{equ:Rorthogonal} 和 \eqRef{equ:unitDet} 来实现推导。

%We explore here the case of the rotation matrix representation $\bfR(t)$, as follows.
首先,我们注意到不可能在满足连续逃逸单位行列式条件方程 ~\eqRef{equ:unitDet} 的同时,满足方程 ~\eqRef{equ:Rorthogonal} ,因为这意味着行列式从 $+1$ 跳到 $-1$。\footnote{换而言之:旋转不能通过连续变换成为反射。}
因此我们只需要研究正交条件方程 \eqRef{equ:Rorthogonal} 。上面写着
%
\begin{align}
\dif{}{t}{(\bfR\tr\bfR)} = \dot\bfR\tr\bfR+\bfR\tr\dot\bfR = 0
~,
\end{align}
%
其结果为
%
\begin{align}
\bfR\tr\dot\bfR 
= -(\bfR\tr\dot\bfR)\tr
~,
\end{align}
%
这意味着矩阵 $\bfR\tr\dot\bfR$ 是一个斜对称矩阵 (即,它等于其转置的负数)。 
将 $3\times3$ 的斜对称矩阵集合表示为 $\so(3)$,并得到 $SO(3)$ 的李代数(\emph{Lie algebra})的名称。
斜对称 $3\times3$ 矩阵的形式是,
%These matrices have 3\,DOF, and it is convenient to write them as the cross-product matrix,
%
\begin{align}
\hatx{\bfomega} \triangleq \begin{bmatrix}
0 & -\omega_z & \omega_y \\
\omega_z & 0 & -\omega_x \\
-\omega_y & \omega_x & 0
\end{bmatrix}
~;
\end{align}
%
它们有3个自由度(3\,DOF),并对应于交叉积矩阵,如我们已经在方程 \eqRef{equ:skew} 所介绍的。这就建立了一个一对一映射 $\bfomega\in\bbR^3\leftrightarrow\hatx{\bfomega}\in\so(3)$.
%
%Because $\so(3)$ is the space where the derivatives of $r(t)$ live, it constitutes the \emph{tangent space} to $SO(3)$, or the \emph{velocity space}, also known as its \emph{Lie algebra}.
让我们取一个向量 $\bfomega=(\omega_x,\omega_y,\omega_z)\in\bfR^3$ 并且写
%
\begin{align}
\bfR\tr\dot\bfR = \hatx{\bfomega}
~.
\end{align}
%
%and let us call $\bfomega$ the vector of instantaneous angular velocities. 
这就产生了常微分方程 (ODE),
%
\begin{align}
\label{equ:Rdot}
\dot\bfR=\bfR\hatx{\bfomega}~.
\end{align}
%
在原点附近,我们有 $\bfR=\bfI$ 则上面的方程降为 $\dot\bfR=\hatx{\bfomega}$。
因此,我们可以将李代数 $\so(3)$ 解释为原点处 $r(t)$ 导数的空间;
%Because $\so(3)$ is the space where the derivatives of $r(t)$ live, 
它构成了对 $SO(3)$ 的切线空间(\emph{tangent space})或速度空间(\emph{velocity space})。
根据这些事实,我们可以很好地称 $\bfomega$ 为瞬时角速度向量。 

如果 $\bfomega$ 是常数,上面的微分方程可以用时间积分为
%
\begin{align}
\bfR(t) = \bfR(0)\,e^{\hatx{\bfomega}t} = \bfR(0)\,e^{\hatx{\bfomega t}} 
\end{align}
%
其中指数 $e^{\hatx{x}}$ 由它的泰勒级数定义,如我们在下一节中看到的。
因为 $\bfR(0)$ 和 $\bfR(t)$ ,那么显然 $e^{\hatx{\bfomega t}}=\bfR(0)\tr\bfR(t)$ 是旋转矩阵。
定义向量 $\bfphi\triangleq\bfomega\Dt$ 为在 $\Dt$ 周期内完全编码旋转的旋转向量,我们得到
%
\begin{align}
\eqbox{
\bfR = e^{\hatx{\bfphi}} \label{equ:vectomat}
}~.
\end{align}
%
这就是所谓的指数映射,是从 $\so(3)$ 到 $SO(3)$ 的应用,
%
\begin{align}
\exp: \so(3) \to SO(3) ~;~ \hatx{\bfphi} \mapsto \exp(\hatx{\bfphi})=e^{\hatx{\bfphi}}
~.
\end{align}
%

\subsubsection{大写的指数映射}

上面的指数映射有时会滥用一些符号表示,即 $\bfphi\in\bbR^3$ 和 $\hatx{\bfphi}\in\so(3)$ 混淆.
%
为了避免可能的歧义,我们选择用大写 $\Exp$ 的显式表示法写这个新的应用 $\bbR^3\to SO(3)$ ,得到 (参见 \figRef{fig:exp_map_R})
%
\begin{figure}[tb]
\begin{center}
\includegraphics{figures/exp_map_R}
\caption{旋转矩阵的指数映射。}
\label{fig:exp_map_R}
\end{center}
\end{figure}
%
\begin{align}
\Exp: \bbR^3 \to SO(3) ~;~ \bfphi \mapsto \Exp(\bfphi) = e^{\hatx{\bfphi}}
~.
\end{align}
%
它与指数映射的关系是平凡的,
%
\begin{align}
\Exp(\bfphi) \triangleq \exp(\hatx{\bfphi})
~.
\end{align}


在下面的部分中,我们将看到向量 $\bfphi$,称为旋转向量或角-轴(angle-axis)向量,通过 $\bfphi=\bfomega\Dt=\phi\bfu$ 编码旋转的角度 $\phi$ 和轴 $\bfu$ 。


%-------------------------------------------------------------
\subsubsection{旋转矩阵和旋转向量: Rodrigues 旋转公式}
%\subsubsection{}


旋转矩阵由旋转向量 $\bfphi=\phi\bfu$ 通过指数映射方程 \eqRef{equ:vectomat} 定义,
同时交叉积矩阵 $\hatx{\bfphi}=\phi\hatx{\bfu}$ 如方程 \eqRef{equ:skew} 中定义。
方程 \eqRef{equ:vectomat} 的泰勒展开式,用 $\bfphi=\phi\bfu$ 读到, 
%
\begin{align}
\bfR=e^{\phi\hatx{\bfu}} = 
	  \bfI 
	+ 			\phi\hatx{\bfu} 
	+ \frac12	\phi^2\hatx{\bfu}^2
	+ \frac1{3!}\phi^3\hatx{\bfu}^3 
	+ \frac1{4!}\phi^4\hatx{\bfu}^4 
	+ \dots
\end{align}
%
当应用于单位向量, $\bfu$,矩阵 $\hatx{\bfu}$ 满足
%
%
\begin{align}
\hatx{\bfu}^2 &= \bfu\bfu\tr-\bfI
\label{equ:prop1}
\\
\hatx{\bfu}^3 &= -\hatx{\bfu}
~, \label{equ:prop2}
\end{align}%
%
并且因此 $\hatx{\bfu}$ 的所有幂可以用循环模式中的 $\hatx{\bfu}$ 和 $\hatx{\bfu}^2$ 来表示,
%
\begin{align}
\hatx{\bfu}^4 &= -\hatx{\bfu}^2 
& \hatx{\bfu}^5 &= \hatx{\bfu} 
& \hatx{\bfu}^6 &= \hatx{\bfu}^2 
& \hatx{\bfu}^7 &=-\hatx{\bfu} 
~~\cdots 
~.
\end{align}
%
然后,根据 $\hatx{\bfu}$ 和 $\hatx{\bfu}^2$,对泰勒级数进行分组,分别识别出 $\sin\phi$ 和 $\cos\phi$的级数,得到一个从旋转向量得到旋转矩阵的闭合形式,即所谓的Rodrigues旋转公式(\emph{Rodrigues rotation formula}),
%
\begin{align}
\eqbox{
\bfR %= e^{\hatx{\bfphi}} 
= \bfI + \sin\phi\hatx{\bfu} + (1-\cos\phi)\hatx{\bfu}^2
}~, \label{equ:rodrigues}
\end{align}%
%
其中我们表示 $\bfR\{\bfphi\}\triangleq\Exp(\bfphi)$ 。
这个公式允许一些变体,例如,使用方程 \eqRef{equ:prop1} ,
%
\begin{align}
\bfR &= \bfI\cos\phi + \hatx{\bfu}\sin\phi + \bfu\bfu\tr(1-\cos\phi)
~.
\end{align}%

\subsubsection{对数映射}

我们把对数映射定义为指数映射的逆,
%
\begin{align}
\log : SO(3)\to\so(3)~;~ \bfR \mapsto \log(\bfR)=\hatx{\bfu\,\phi}
~,
\end{align}
%
其中
%
\begin{align}
\phi &= \arccos\left(\frac{\trace(\bfR)-1}{2}\right) 
\\
\bfu &= \frac{(\bfR-\bfR\tr)^\vee}{2\sin\phi} 
~,
\end{align}
%
其中 $\bullet^\vee$ 是 $\hatx{\bullet}$的倒数,即, $(\hatx{\bfv})^\vee=\bfv$ 和 $\hatx{\bfV^\vee}=\bfV$ 。

我们还定义了一个大写版本 $\Log$,它允许我们从旋转矩阵中直接恢复旋转向量 $\bfphi=\bfu\phi\in\bbR^3$ , 
%
\begin{subequations}
\begin{align}
\Log: SO(3) \to \bbR^3 ~;~ \bfR\mapsto\Log(\bfR) = \bfu\,\phi 
~.
\end{align}
\end{subequations}
%
它与对数映射的关系是平凡的,
%
\begin{align}
\Log(\bfR) \triangleq (\log(\bfR))^\vee
~.
\end{align}



\subsubsection{旋转动作}

旋转一个向量 $\bfx$ 围绕单位轴 $\bfu$ 转过一个角度 $\phi$ %following the right-hand rule 
要用线性乘积来执行
%
\begin{align}
\bfx'=\bfR\,\bfx
~, 
\label{equ:rotWithMat}
\end{align}
%
其中 $\bfR=\Exp(\bfu\phi)$ 。
这可以通过扩展方程 \eqRef{equ:rotWithMat} ,
使用方程 \eqRef{equ:rodrigues} , \eqRef{equ:prop1}  和 \eqRef{equ:prop2} 来表示,
%in a way akin to the last steps of \eqRef{equ:quatRotFormula}, 
%to obtain the vector rotation formula \eqRef{equ:vecRotFormula},
%
\begin{align}
\begin{split}
\bfx' &= \bfR\,\bfx  \\
&= (\bfI + \sin\phi\hatx{\bfu} + (1-\cos\phi)\hatx{\bfu}^2)\,\bfx  \\
&= \bfx + \sin\phi\hatx{\bfu}\bfx + (1-\cos\phi)\hatx{\bfu}^2\bfx  \\
&= \bfx + \sin\phi(\bfu\tcross\bfx) + (1-\cos\phi)(\bfu\bfu\tr-\bfI)\,\bfx  \\
&= \bfx_\| + \bfx_\bot + \sin\phi(\bfu\tcross\bfx) - (1-\cos\phi)\,\bfx_\bot  \\
&= \bfx_\| + (\bfu\tcross\bfx)\sin\phi + \bfx_\bot\cos\phi
~,
\end{split}
\end{align}%
%
这正是向量旋转公式 \eqRef{equ:vecRotFormula} ,


%-------------------------------------------------------------
\subsection{旋转群和四元数}

为了教学的目的,我们有兴趣强调四元数和旋转矩阵之间的联系,作为旋转群 $SO(3)$的表示。 
为此,四元数旋转动作的众所周知的公式是,
%For this, it will be convenient to assume as hypothesis that the rotation action using quaternions is achieved with the double product,
%
\begin{align} \label{equ:qrot}
r(\bfv)=\bfq\ot\bfv\ot\bfq^*
~,
\end{align}
%
这里是最初的假设。
这使我们能够发展出完整的四元数部分,其中包含一个追溯我们用于旋转矩阵的部分的论述。 
这一假设的正确性将在稍后的,在 \secRef{sec:qRotAction} 中得到证明,从而验证该方法。 
%This enables us to position the unit quaternion as a powerful representation of the rotation group $SO(3)$.

然后,我们将上述旋转注入正交条件方程 \eqRef{eq:keepnorm} ,并使用方程 \eqRef{equ:norm_prod} 做为
%
\begin{align}
\norm{\bfq\ot\bfv\ot\bfq^*}=\norm{\bfq}^2\norm{\bfv} = \norm{\bfv}
~.
\end{align}
%
这就得到 $\norm{\bfq}^2=1$,这就是四元数的单位范数条件,它是,
%
\begin{align} \label{equ:q_unit}
\eqbox{
\bfq^*\ot\bfq = 1 = \bfq\ot\bfq^*
}
~.
\end{align}
%
这个条件类似于我们在旋转矩阵中遇到的条件,参见方程 \eqRef{equ:Rorthogonal} ,它读为 $\bfR\tr\bfR=\bfI=\bfR\bfR\tr$。我们建议读者暂停一会讨论它们的相似之处。

%
类似地,我们证明相对方向条件方程 \eqRef{equ:keeporientation} 通过构造(我们使用方程 \eqRef{equ:quatCommutatorPure} 两次,如下所示)得到满足,
%
\begin{align}
\begin{split}
r(\bfv)\times r(\bfw) 
&= (\bfq\ot\bfv\ot\bfq^*) \times (\bfq\ot\bfw\ot\bfq^*) \\
%
\eqRef{equ:quatCommutatorPure}~~
&= \frac12\big((\bfq\ot\bfv\ot\bfq^*) \ot (\bfq\ot\bfw\ot\bfq^*) - (\bfq\ot\bfw\ot\bfq^*) \ot (\bfq\ot\bfv\ot\bfq^*) \big) \\
&= \frac12(\bfq\ot\bfv\ot\bfw\ot\bfq^* - \bfq\ot\bfw\ot\bfv\ot\bfq^*) \\
&= \frac12(\bfq\ot(\bfv\ot\bfw - \bfw\ot\bfv)\ot\bfq^*) \\
%
\eqRef{equ:quatCommutatorPure}~~
&= \bfq\ot(\bfv\times\bfw)\ot\bfq^* \\
&= r(\bfv\times\bfw)
~.
\end{split}
\end{align}


单位四元数的集合在乘法操作下构成一个群。这个群在拓扑上是一个 3-sphere,即 $\bbR^4$ 的单位球的 3D 表面,通常称为 $S^3$ 。
%The group $S^3$ constitutes a double cover of $SO(3)$.
%Since $\bfq$ and $-\bfq$ produce the same rotation, see \eqRef{equ:qrot}, we have that $S^3$ is a double cover of $SO(3)$.

\subsubsection{指数映射}

让我们考虑一个单位四元数 $\bfq\in S^3$ ,就是, $\bfq^*\ot\bfq=1$ ,并让我们继续处理,就像我们对旋转矩阵 $\bfR\tr\bfR=\bfI$ 的正交性条件所做的那样。
取时间导数,
%
\begin{align}
\dif{(\bfq^*\ot\bfq)}{t} = \dot\bfq^*\ot\bfq+\bfq^*\ot\dot\bfq=0
~,
\end{align}
%
由此可见
%
\begin{align}
\bfq^*\ot\dot\bfq = -(\dot\bfq^*\ot\bfq) = -(\bfq^*\ot\dot\bfq)^*~,
\end{align}
%
这意味着 $\bfq^*\ot\dot\bfq$ 是一个纯四元数 (即,它等于它的负共轭,因此它的实部为零)。
%In the quaternion case, however, this space is not directly the velocity space, but rather the space of the half-velocities, as we will se soon.
%The set of pure quaternions is denoted $\bbH_p=\Im(\bbH)$ and constitutes the Lie Algebra of $\bbH$.
因此我们得到一个纯四元数 $\bfOmega\in\bbH_p$ 并写为,
%
\begin{align}
\bfq^*\ot\dot\bfq = \bfOmega = \begin{bmatrix}
0\\\bfOmega
\end{bmatrix} 
\in\bbH_p
~.
\end{align}
%
左乘 $\bfq$ 得到微分方程,
%
\begin{align}
\label{equ:qdotOmega}
\dot\bfq = \bfq\ot\bfOmega~.
\end{align}
% 
在原点附近,我们有 $\bfq=1$ 则上面的方程降为 $\dot\bfq=\bfOmega\in\bbH_p$ 。
因此,纯四元数空间 $\bbH_p$ 构成四元数的单位球面 $S^3$ 的切线空间(\emph{tangent space})或李代数(Lie Algebra)。 
%In the quaternion case, vectors in the tangent space correspond to the half of the angular velocity vectors.
然而,在四元数的情况下,这个空间并不直接是速度空间,而是一半速度的空间,我们很快就会看到。


如果 $\bfOmega$ 是常数,微分方程可以积分为
%
\begin{align}\label{equ:qexpWt}
\bfq(t) = \bfq(0)\ot e^{\bfOmega\, t}
~,
\end{align}
%
其中,因为 $\bfq(0)$ 和 $\bfq(t)$ 是单位四元数,所以指数 $e^{\bfOmega t}$ 也是单位四元数 --- 这是我们从四元数指数方程 ~\eqRef{equ:EulerFormulaQuat} 中已经知道的。
%
定义 $\bfV\triangleq\bfOmega\Dt$ 我们有
%
\begin{align}\label{equ:qexpV}
\eqbox{
\bfq = e^{\bfV}
}~.
\end{align}
%
这又是一个指数映射:从纯四元数空间到单位四元数表示的旋转空间的应用,
%
\begin{align}\label{equ:q_expmap}
\exp:\bbH_p\to S^3~;~ \bfV\mapsto \exp(\bfV) = e^{\bfV}
\end{align}
%

\subsubsection{大写指数映射}

%At this point, we still have to relate the pure quaternion $\bfV$ in the exponential map~\eqRef{equ:q_expmap} with the angle-axis rotation parameters in Cartesian space. 
我们将看到,纯四元数 $\bfV$ 在指数映射方程 ~\eqRef{equ:q_expmap} 中的编码,通过 $\bfV=\theta\bfu = \phi\bfu/2$,旋转轴 $\bfu$ 和旋转角的一半, $\theta=\phi/2$,来进行。
我们将很快对这个一半角度的事实提供充分的解释,主要在第 \ref{sec:qRotAction}, \ref{sec:double_cover} 和 \ref{sec:isoclinic} 节。现在,可以说,由于旋转动作是由双倍乘积 $\bfx'=\bfq\ot\bfx\ot\bfq^*$ 完成的,向量 $\bfx$ 经历的旋转是 $\bfq$中编码的旋转的“两倍”,或者等价地说,四元数 $\bfq$ 编码 ~$\bfx$上的“一半”旋转。

为了表达角-轴(angle-axis)旋转参数, $\bfphi=\phi\bfu\in\bbR^3$,和四元数之间的直接关系,我们定义指数映射的一个大写版本,它捕获半角效应 (参见 \figRef{fig:exp_map_q}),
%
\begin{figure}[tb]
\begin{center}
\includegraphics{figures/exp_map_q}
\caption{四元数的指数映射。}
\label{fig:exp_map_q}
\end{center}
\end{figure}
%
\begin{align}
\Exp:\bfR^3\to S^3~;~\bfphi\mapsto\Exp(\bfphi)=e^{\bfphi/2}
\end{align}
%
它与指数映射的关系是平凡的,
%
\begin{align}
\Exp(\bfphi) \triangleq \exp(\bfphi/2)
~.
\end{align}


这也方便引入角速度向量 $\bfomega=2\bfOmega\in\bbR^3$,使得方程 \eqRef{equ:qdotOmega} 和 \eqRef{equ:qexpWt} 变成,
%
\begin{align}
\dot\bfq &= \frac12\bfq\ot\bfomega \label{equ:qdot} \\ 
\bfq &= e^{\bfomega t/2}
~.
\end{align}
%%
%where we call $\bfomega$ the vector of instantaneous angular velocities ---the presence of the `half' term will become clear in the following sections, especially in Sections \ref{sec:quatAndVector} and \ref{sec:isoclinic}.
%Left-multiplication by $\bfq$ yields the differential equation,
%%
%\begin{align}
%\label{equ:qdot}
%\dot\bfq = \frac12\,\bfq\ot\bfomega~.
%\end{align}
%% 
%If $\bfomega$ is constant, this can be integrated as
%%
%\begin{align}
%\bfq(t) = \bfq(0)\ot e^{\bfomega t/2}
%~,
%\end{align}
%%
%where, since $\bfq(0)$ and $\bfq(t)$ are unit quaternions, the exponential $e^{\bfomega t/2}$ is also a unit quaternion ---something we already knew from the properties of the quaternion exponential.
%%
%Defining the vector $\bfphi\triangleq\bfomega\Dt$ as the angle-axis vector encoding the full rotation over a period $\Dt$, we have
%%
%\begin{align}\label{equ:vectoquatEuler}
%\eqbox{
%\bfq = e^{\bfphi/2}
%}~.
%\end{align}
%
%This is again an exponential map, from the space of pure quaternions to the space of unit quaternions.
%%
%As we did for the exponential map of the rotation matrix, we opt for an explicit notation using a capitalized $\Exp$ function, which relates directly the rotation vector to the quaternion,
%%
%\begin{align}
%\Exp: \bbR^3 \to \bbH ~;~  \bfphi \mapsto \Exp(\bfphi) = e^{\bfphi/2} 
%~.
%\end{align}
%%
%Its relation with the quaternion exponential is trivial,
%%
%\begin{align}
%\Exp(\bfphi) \triangleq \exp(\bfphi/2)
%~.
%\end{align}


%-------------------------------------------------------------
\subsubsection{四元数和旋转向量}
\label{sec:quatAndVector}

设 $\bfphi=\phi\bfu$ 是一个旋转向量表示绕着轴 $\bfu$ 的 $\phi$\,rad 的旋转,
然后,
指数映射可以使用Euler公式(\emph{Euler formula})进行扩展(对于一个完整的扩展参见方程 \eqsRef{equ:qvPowers}{equ:EulerFormulaQuat}),
%
\begin{align}
\eqbox{
\bfq \triangleq \Exp(\phi\bfu) = e^{\phi\bfu/2} = \cos \frac{\phi}{2} + \bfu\sin\frac{\phi}{2}=\begin{bmatrix}
\cos(\phi/2) \\
\bfu\sin(\phi/2)
\end{bmatrix}
}
~.   \label{equ:vectoquat}
\end{align}
%
我们称之为旋转向量到四元数(\emph{rotation vector to quaternion})的转换公式,并且在本文中标记为 
$\bfq=\bfq\{\bfphi\}\triangleq\Exp(\bfphi)$. 




\subsubsection{对数映射}

我们把对数映射定义为指数映射的逆,
%
\begin{align}
\log:S^3\to\bbH_p ~;~ \bfq\mapsto \log(\bfq) = \bfu\theta
~,
\end{align}
%
这当然是我们在 \secRef{sec:qlog} 中给出的四元数对数的定义。
我们还定义了大写的对数映射,在笛卡尔 3-space 中,它直接提供了旋转的角度 $\phi$ 和轴,
%
\begin{align}
\Log:S^3\to\bbR^3 ~;~ \bfq\mapsto \Log(\bfq) = \bfu\phi
~.
\end{align}
%
它与对数映射的关系是平凡的,
%
\begin{align}
\Log (\bfq) \triangleq 2\log(\bfq)
~.
\end{align}
%

对于它的实现,我们使用 $\arctan(y,x)$的四象限版本。
从方程 \eqRef{equ:vectoquat} 中,
%
\begin{subequations}
\begin{align}
\phi &= 2\arctan(\norm{\qv},q_w) \\
\bfu &= \qv / \norm{\qv} \label{equ:qvec}
~.
\end{align}
\end{subequations}
%
对于小角度四元数,方程 \eqRef{equ:qvec} 发散。于是我们对 $\arctan()$ 函数使用泰勒级数并截断,得到,
%
\begin{equation}
\Log(\bfq) = \theta\bfu 
\approx 2\,\frac{\qv}{q_w} \left(1 - \frac{\norm{\qv}^2}{3q_w^2}\right) \label{equ:log_q_small}
~.
\end{equation}


\subsubsection{旋转动作}
\label{sec:qRotAction}

我们终于在这里证明用四元数旋转向量的假设方程 ~\eqRef{equ:qrot} ,  
%$\bfx'=\bfq\otimes\bfx\otimes\bfq^*$, 
从而验证到目前为止所有的资料。
%, and stating the unit quaternion $\bfq$ as a proper representation of the rotation group $SO(3)$.
%
用双四元数乘积,也称为“三明治”乘积来执行一个向量 $\bfx$ 绕轴 $\bfu$ 旋转一个角度 $\phi$ 这个动作,
%
\begin{align}
\bfx' = \bfq\otimes\bfx\otimes\bfq^* ~, \label{equ:sandwichProd}
\end{align}
%
式中 $\bfq=\Exp(\bfu\phi)$,并且
其中向量 $\bfx$ 已经写为四元数形式,就是 
%
\begin{align}
\bfx= x i + y j + z k = \begin{bmatrix}
0 \\ \bfx
\end{bmatrix} \in \bbH_p
~. \label{equ:quatvec}
\end{align}%
%
%
为了证明这个双乘积确实执行了所需的向量旋转,我们使用方程 \eqRef{equ:quatProdVec} ,
\eqRef{equ:vectoquat},和基本向量和三角恒等式来扩展方程 ~\eqRef{equ:sandwichProd} 如下,
%
\begin{align}\label{equ:quatRotFormula}
\begin{split}
\bfx'
&= \bfq \ot \bfx \ot \bfq^* \\
&= \Big(\cos \frac{\phi}{2} + \bfu \sin \frac{\phi}{2}\Big)
 \ot (0+\bfx)
 \ot \Big(\cos \frac{\phi}{2} - \bfu \sin \frac{\phi}{2}\Big)
 \\
&= \bfx \cos^2 \frac{\phi}{2} + (\bfu\ot\bfx - \bfx\ot\bfu) \sin \frac{\phi}{2} \cos \frac{\phi}{2} - \bfu\ot\bfx\ot\bfu \sin^2 \frac{\phi}{2} \\
&= \bfx \cos^2 \frac{\phi}{2} + 2 (\bfu \tcross \bfx) \sin \frac{\phi}{2} \cos \frac{\phi}{2} - (\bfx (\bfu \tr \bfu) - 2 \bfu (\bfu \tr \bfx)) \sin^2 \frac{\phi}{2} \\
&= \bfx (\cos^2 \frac{\phi}{2} - \sin^2 \frac{\phi}{2}) + (\bfu \tcross \bfx) (2\sin\frac{\phi}{2} \cos\frac{\phi}{2}) + \bfu (\bfu \tr \bfx) (2\sin^2 \frac{\phi}{2}) \\
&= \bfx \cos \phi + (\bfu \tcross \bfx) \sin \phi + \bfu (\bfu \tr \bfx) (1 - \cos \phi) \\
&= (\bfx - \bfu \,\bfu \tr \bfx) \cos \phi + (\bfu \tcross \bfx) \sin \phi + \bfu \,\bfu \tr \bfx \\
&= \bfx_{\bot} \cos \phi + (\bfu \tcross \bfx) \sin \phi + \bfx_{||} ~,
\end{split}
\end{align}%
%
这正是向量旋转公式 ~\eqRef{equ:vecRotFormula} 。


\subsubsection{ $SO(3)$ 流形的双倍覆盖。}
\label{sec:double_cover}

考虑一个单位四元数 $\bfq$。当其被视为一个规则的 4-vector时, $\bfq$ 和特征四元数 $\bfq_1=[1,0,0,0]$ 之间的角度 $\theta$ 表示方向的原始值,
%
\begin{align}
\cos\theta = \bfq_1\tr\bfq = \bfq(1) = q_w
~.
\end{align}
%
同时,在 3D 空间中被四元数 $\bfq$ 旋转的物体的旋转角 $\phi$ 满足
%
\begin{align}
%\bfq_1^*\ot\bfq = 
\bfq = \begin{bmatrix}
q_w \\ \qv
\end{bmatrix} = \begin{bmatrix}
\cos\phi/2 \\ \bfu\sin\phi/2
\end{bmatrix}
~.
\end{align}
%
也就是说,我们有 $q_w = \cos\theta = \cos\phi/2$,
所以四元数向量和 4D 空间中的特征值之间的夹角是 3D 空间中四元数旋转的夹角的一半,
%
\begin{align}
\theta = \phi/2
~.
\end{align}

我们在 \figRef{fig:double_cover} 中演示了这个双倍覆盖。
当两个四元数向量之间的夹角是 $\theta=\pi/2$ 时, 3D 旋转已经到达 $\phi=\pi$,这是半圈。 
当四元数向量旋转半圈时, $\theta=\pi$ , 3D 旋转已经完成了一个完整的旋转。 
%This is in accordance with a previous result showing that the negated quaternion represents the same orientation. 
四元数向量的第二个半圈, $\pi<\theta<2\pi$,表示 3D 旋转的第二个全圈 $2\pi<\phi<4\pi$,这就是旋转流形的第二个覆盖。

\begin{figure}[htbp]
\begin{center}
\includegraphics{figures/double_cover}
\caption{旋转流形的双倍覆盖。左边:四元数 $\bfq$ 在单位 3-sphere 中用单位四元数 $\bfq_1$ 定义一个角度 $\theta$ 。中间: 3D 旋转的结果 $\bfx'=\bfq\ot\bfx\ot\bfq^*$ 比原来的四元数有双倍的角度 $\phi$ 。右边: 将 4D 和 3D 旋转平面叠加,观察四元数 $\bfq$ 在 3-sphere 上的一圈(红色)如何表示 3D 空间中被旋转向量 $\bfx$ 的两圈 (蓝色)。}
\label{fig:double_cover}
\end{center}
\end{figure}


%\subsubsection{The quaternion as a representation of $SO(3)$}




%-------------------------------------------------------------
\subsection{旋转矩阵和四元数}

正如我们刚才看到的,给定一个旋转向量 $\bfphi=\bfu\,\phi$,对于单位四元数和旋转矩阵的指数映射产生旋转算子 
$\bfq=\Exp(\bfu\,\phi)$ 
和
$\bfR=\Exp(\bfu\,\phi)$ 
这旋转向量 $\bfx$ 实际上围绕着相同的 $\bfu$ 有着相同的角度 $\phi$ 。% 
%
\footnote{指数映射 $\bfR=\Exp(\bfphi)$ 和 $\bfq = \Exp(\bfphi)$ 之间明显的符号歧义性很容易由上下文解决:
在某些情况下,它只是返回值的类型, $\bfR$ 或 $\bfq$; 
其它情况下是四元数乘积 $\ot$ 存在或不存在。}
即,如果
%
\begin{align}
\forall \bfphi,\bfx \in \bbR^3,~ 
\bfq = \Exp(\bfphi) 
,~ 
\bfR=\Exp(\bfphi)
%\rightarrow 
%\bfq\ot%\ol
%\bfx\ot\bfq^* = \bfR\,\bfx ~,
\end{align}
%
%or, seen side by side,
%%
%\begin{align*}
%%\ol
%\bfx' &= \bfq\otimes
%%\ol
%\bfx\otimes\bfq^* ~,
%& 
%\bfx' &= \bfR\,\bfx~.
%\end{align*}%
%
%Here, we used the bar notation $\ol\bfx$ to indicate that the vector $\bfx$ in the left equation is expressed in quaternion form~\eqRef{equ:quatvec}, thus differentiating it from that on the right. 
%However, this circumstance is mostly unambiguous and can be derived from the context, and especially by the presence of the quaternion product $\otimes$. 
%In what is to follow, we are omitting this bar and writing simply,
%%
%$\bfx' = \bfq\otimes\bfx\otimes\bfq^*$\,.
%
%This allows us to 
%write,%
%\footnote{More explicit expressions would be,~ 
%$\bfq\otimes\ol\bfx
%\otimes\bfq^* = \ol{ \bfR\bfx
%}$~,
%~or~ 
%$\bfq\otimes\begin{bmatrix}
%0\\\bfx
%\end{bmatrix}\otimes\bfq^* = \begin{bmatrix}
%0 \\ \bfR\bfx
%\end{bmatrix}$~. See \secRef{sec:altQuat}.
%}
%
那么,
%
\begin{align}
\bfq\otimes\bfx\otimes\bfq^* = \bfR\,\bfx~.
\end{align}
%
由于这个恒等式的两边在 $\bfx$中是线性的,所以通过展开左边和识别右边的项,找到了与四元数等价的旋转矩阵的表达式,从而得到四元数到旋转矩阵(\emph{quaternion to rotation matrix})公式,
%
\begin{align}
\eqbox{
\bfR = \begin{bmatrix}
q_w^2+q_x^2-q_y^2-q_z^2 & 2(q_xq_y-q_wq_z) & 2(q_xq_z+q_wq_y) \\ 
2(q_xq_y+q_wq_z) & q_w^2-q_x^2+q_y^2-q_z^2 & 2(q_yq_z-q_wq_x) \\
2(q_xq_z-q_wq_y) & 2(q_yq_z+q_wq_x) & q_w^2-q_x^2-q_y^2+q_z^2
\end{bmatrix}
}~,
\end{align}%
%
在整个文档中用 $\bfR=\bfR\{\bfq\}$表示。
四元数乘积方程 \eqsRef{equ:quatMatProd}{equ:quatMatrix} 的矩阵形式为我们提供了另一个公式 % for the rotation matrix
,因为
%
%
\begin{align}
\bfq\otimes%\ol
\bfx\otimes\bfq^*
&= \QR{\bfq^*}\,\QL{\bfq}\begin{bmatrix}
0 \\ \bfx
\end{bmatrix} 
= \begin{bmatrix}
0 \\ \bfR\,\bfx
\end{bmatrix} 
\label{equ:quatRotMatrixForm}
~,
\end{align}
%
经过一些简单的扩展
%
\begin{align}
\eqbox{\bfR = (q_w^2-\qv\tr\qv)\,\bfI + 2\,\qv\qv\tr + 2\,q_w\hatx{\qv}}~.
\end{align}



旋转矩阵 $\bfR$ 对于四元数具有以下性质,
%
%
\begin{align}
\bfR\{[1,0,0,0]\tr\} &= \bfI \label{equ:rotident}\\
\bfR\{-\bfq\} &= \bfR\{\bfq\} \label{equ:rotneg} \\
\bfR\{\bfq^*\} &= \bfR\{\bfq\}\tr \label{equ:rotconj} \\
\bfR\{\bfq_1\ot\bfq_2\} &= \bfR\{\bfq_1\}\bfR\{\bfq_2\} \label{equ:rotprod}%\\
%\bfR\{\bfq^t\}=\bfR\{\bfq\}^t \label{equ:rotslerp}
~, 
\end{align}%
%
其中我们观察到: 
方程 \eqRef{equ:rotident}~ 特征四元数编码为零旋转;  
方程 \eqRef{equ:rotneg}~ 一个四元数及其负数编码为同一旋转,定义 $SO(3)$的双倍覆盖;
方程 \eqRef{equ:rotconj}~ 共轭四元数编码为逆旋转;并且
方程 \eqRef{equ:rotprod}~ 四元数积按与旋转矩阵相同的顺序组成连续旋转。 

另外,我们还有性质% (see \eqRef{equ:Rslerp} a few pages forward),
%
\begin{align}
\bfR\{\bfq^t\}=\bfR\{\bfq\}^t
~,
\end{align}
%
它将四元数的球面插值和旋转矩阵关联到一个连续标量 $t$ 上。


\subsection{旋转组合}

四元数旋转组合类似于旋转矩阵,即使用相同的四元数和矩阵乘积顺序,则有相同的旋转顺序 (\figRef{fig:composition}),
%
\begin{align}
\bfq_{\cA\cC} &= \bfq_{\cA\cB}\ot\bfq_{\cB\cC} ~,
&
\bfR_{\cA\cC} &= \bfR_{\cA\cB}\,\bfR_{\cB\cC} ~.\label{equ:rotComposition}
\end{align}%
%
%
%这里采用Hamilton的 local-to-global 约定确定,当向组合链的向左移动时,旋转组合从局部走向全局,或者当向右移动时,旋转组合从全局走向局部。 
%This means that $\bfq_{\cB\cC}$ and $\bfR_{\cB\cC}$ are specifications of a frame $\cC$ that is local \wrt frame $\cB$. 
%
这直接来自于相关乘积的关联性,
%
\begin{align*}
\bfx_\cA 
&= \bfq_{\cA\cB}\ot\bfx_\cB\ot\bfq_{\cA\cB}^* 
& \bfx_\cA
&= \bfR_{\cA\cB}\,\bfx_\cB
\\
&= \bfq_{\cA\cB}\ot(\bfq_{\cB\cC}\ot\bfx_\cC\ot\bfq_{\cB\cC}^*)\ot\bfq_{\cA\cB}^* 
&&= \bfR_{\cA\cB}\,(\bfR_{\cB\cC}\,\bfx_\cC) 
\\
&= (\bfq_{\cA\cB}\ot\bfq_{\cB\cC})\ot\bfx_\cC\ot(\bfq_{\cB\cC}^*\ot\bfq_{\cA\cB}^*) 
&&= (\bfR_{\cA\cB}\,\bfR_{\cB\cC})\,\bfx_\cC 
\\
&= (\bfq_{\cA\cB}\ot\bfq_{\cB\cC})\ot\bfx_\cC\ot(\bfq_{\cA\cB}\ot\bfq_{\cB\cC})^* 
&&= \bfR_{\cA\cC}\,\bfx_\cC ~.
\\
&= \bfq_{\cA\cC}\ot\bfx_\cC\ot\bfq_{\cA\cC}^* 
%&&= \bfR_{\cA\cC}\,\bfx_\cC 
~,
\end{align*}

\begin{figure}[htbp]
\begin{center}
\includegraphics{figures/composition}
\caption{旋转组合。在 $\bbR^2$ 中,我们只需要做 $\theta_{\cA\cC} = \theta_{\cA\cB}+\theta_{\cB\cC}$,因为“加法”操作是可交换的。 
在 $\bbR^3$ 中组合满足 $\bfq_{\cA\cC} = \bfq_{\cA\cB}\ot\bfq_{\cB\cC}$ ,并且在矩阵形式中,$\bfR_{\cA\cC} = \bfR_{\cA\cB}\,\bfR_{\cB\cC}$。
这些算子是不可交换的,必须严格遵守顺序 --- 正确的符号有助于理解:`AB' 链与 `BC' 链创建 `AC'.}
\label{fig:composition}
\end{center}
\end{figure}
%

\paragraph{符号法评论}
一个正确的符号有助于确定组成中各因素的正确顺序,特别是对于几个旋转的组合 (参见 \figRef{fig:composition}).
例如,让 $\bfq_{ji}$ (分别是 $\bfR_{ji}$) 表示从情形 $i$ 到情形 $j$的旋转,即 $\bfx_j=\bfq_{ji}\ot\bfx_i\ot\bfq_{ji}^*$ (分别是 $\bfx_j=\bfR_{ji}\bfx_i$)。
然后,给定由四元数 $\bfq_{OA},\bfq_{AB},\bfq_{BC},\bfq_{OX},\bfq_{XZ}$表示的多个旋转,我们只需链接索引并得到:
%
\begin{align*}
\bfq_{OC} &= \bfq_{OA}\ot\bfq_{AB}\ot\bfq_{BC}
&
\bfR_{OC} &= \bfR_{OA}\,\bfR_{AB}\,\bfR_{BC}
~,
\end{align*}
%
并且已知相反的旋转对应于共轭, $\bfq_{ji}=\bfq_{ij}^*$,或转置,$\bfR_{ji}=\bfR_{ij}\tr$,我们也有
%
\begin{align*}
\bfq_{ZA} &= \bfq_{XZ}^*\ot\bfq_{OX}^*\ot\bfq_{OA} 
&
\bfR_{ZA} &= \bfR_{XZ}\tr\,\bfR_{OX}\tr\,\bfR_{OA} 
\\
&= \bfq_{ZX}\ot\bfq_{XO}\ot\bfq_{OA}
&
&= \bfR_{ZX}\,\bfR_{XO}\,\bfR_{OA}
~.
\end{align*}




\subsection{球面线性插值 (SLERP)}
\label{sec:slerp}

四元数对于计算正确的方向插值非常方便。 
给定由两个四元数 $\bfq_0$ 和 $\bfq_1$表示的两个方向,我们要找到一个四元数函数 $\bfq(t),~ t\in[0,1]$, 计算从 $\bfq(0)=\bfq_0$ 到 $\bfq(1)=\bfq_1$ 的线性插值。
这种插值是这样的,当 $t$ 从 $0$ 到 $1$ 演变时,物体将沿着方向 $\bfq_0$ 连续地沿固定轴以恒定速度旋转到方向 $\bfq_1$ 。

\paragraph{方法 1}
第一种方法使用四元数代数,并在 $\bbR^3$ 中遵循几何推理,这应该很容易与目前已知的资料相关连。
首先,像这样 $\bfq_1=\bfq_0\ot\Delta\bfq$ 计算从 $\bfq_0$ 到 $\bfq_1$ 的方向增量 $\Delta\bfq$ ,
%
\begin{align}
\Delta\bfq = \bfq_0^*\ot\bfq_1
~.
\end{align}
%
然后,获得相关的旋转向量, $\Delta\bfphi=\bfu\Delta\phi$ ,
%Then take a linear fraction of the involved rotation, 
通过使用对数映射,\footnote{我们可以在这里使用 $\log()$ 映射和 $\exp()$ 映射,也可以使用它们的大写形式的 $\Log()$ 映射和 $\Exp()$映射。最终得到的角度中所涉及的因子2最终是不相关的,因为它在最终公式中被抵消了。}
%
\begin{align}\label{equ:LogDq}
\bfu\,\Delta\phi = \Log(\Delta\bfq)
~.
\end{align}
%
最后,保持旋转轴 $\bfu$ 并取旋转角的线性部分, $\delta\phi=t\Delta\phi$ 。
通过指数映射,将其以四元数形式表示, $\delta\bfq=\Exp(\bfu\,\delta\phi)$,并与原四元数合成得到插值结果,
%
\begin{align}
\bfq(t) = \bfq_0\ot\Exp(t\,\bfu\,\Delta\phi)
~.
\end{align}
%
整个过程可以写成 $\bfq(t)=\bfq_0\ot \Exp(t\Log(\bfq_0^*\ot\bfq_1))$,它可以简化为
%
\begin{align}
\eqbox{
\bfq(t)=\bfq_0\ot(\bfq_0^*\ot\bfq_1)^t
}
~,
\end{align}
%
并且通常实现 (参见方程 \eqRef{equ:qa}) 为,
%
\begin{align}
\bfq(t)=\bfq_0\ot
\begin{bmatrix}
\cos (t\,\Delta\phi/2) \\ \bfu \sin (t\,\Delta\phi/2)
\end{bmatrix}
~.
\end{align}


\begin{figure}[htbp]
\begin{center}
\includegraphics{figures/slerp_S4}
\caption{在 $\bbR^4$ 中的单位球中的四元数插值,以及在 $\bbR^4$ 中的旋转平面 $\pi$ 上的情况的正面视图。}
\label{fig:slerp_S4}
\end{center}
\end{figure}

\paragraph{注:} 类似的乘积可用于定义旋转矩阵的Slerp,产生
%
\begin{align}\label{equ:Rslerp}
\bfR(t)= \bfR_0\Exp(t\Log(\bfR_0\tr\bfR_1))=\bfR_0(\bfR_0\tr\bfR_1)^t 
~,
\end{align}
%
其中矩阵指数 $\bfR^t$ 可以使用 Rodrigues 公式 \eqRef{equ:rodrigues} 实现,导致
%
\begin{align}
\bfR(t)= \bfR_0 \left(\bfI + \sin(t\Delta\phi)\hatx{\bfu} + (1-\cos(t\Delta\phi))\hatx{\bfu}^2\right)
~.
\end{align}


\paragraph{方法 2}
可以开发出与四元数代数的内部无关,甚至与嵌入弧形的空间的维数无关的其它Slerp方法。 
特别是, 参见 \figRef{fig:slerp_S4},我们可以将四元数 $\bfq_0$ 和 $\bfq_1$ 视为单位球体中的两个单位向量,并在同一空间中进行插值。 
插值 $\bfq(t)$ 是单位向量,以恒定角速度
以最短球面路径从 $\bfq_0$ 到 $\bfq_1$ 进行连接。
该路径是单位球体与由 $\bfq_0$, $\bfq_1$ 和原点 (图中的虚线圆周)定义的平面相交产生的平面弧的结果。
有关这些方法与上述方法等效的证明,请参见 \cite{DAM-1998} 的论文。

这些方法的第一方法是使用向量代数,并严格遵循上述思想。 
把 $\bfq_0$ 和 $\bfq_1$ 做为两个单位向量;
%
%This angle%
它们之间的夹角%
\footnote{这个角度 $\Delta\theta=\arccos(\bfq_0\tr\bfq_1)$ 是欧几里德 4-space 中两个四元数向量之间的夹角,而不是 3D 空间中的实际旋转角度,这旋转角度从方程 \eqRef{equ:LogDq} 得到的是 $\Delta\phi=\norm{\Log(\bfq_0^*\ot\bfq_1)}$。参见 \secRef{sec:double_cover} 获得更多详细信息。}
由标量积导出,
%
\begin{align}\label{equ:slerp_angle}
\cos(\Delta\theta)&=\bfq_0\tr\bfq_1 & \Delta\theta&=\arccos(\bfq_0\tr\bfq_1)
~.
\end{align}
%
我们按以下步骤进行。 
我们识别了旋转平面,我们在这里命名为 $\pi$,
并且建立了它的正交正态基(ortho-normal basis) $\{\bfq_0,\bfq_\bot\}$,其中 $\bfq_\bot$ 来自于对 $\bfq_0$ 的正交正态化(ortho-normalizing) $\bfq_1$ ,
%
\begin{align}
\bfq_\bot &= \frac{\bfq_1-(\bfq_0\tr\bfq_1)\bfq_0}{\norm{\bfq_1-(\bfq_0\tr\bfq_1)\bfq_0}}
~,
\end{align}
%
所以 (参见 \figRef{fig:slerp_S4} --- 右边)
%
\begin{align} \label{equ:q1}
\bfq_1 = \bfq_0 \cos \Delta\theta + \bfq_\bot \sin \Delta\theta
~.
\end{align}
%
然后,我们只需要转动 $\bfq_0$ 一个角度的一部分, $t\Delta\theta$,在平面 $\pi$上,
以得到球面插值,
%
\begin{align}\label{equ:slerp_rot}
\eqbox{
\bfq(t) = \bfq_0 \cos(t\Delta\theta) + \bfq_\bot\sin(t\Delta\theta)
}
~.
\end{align}


\paragraph{方法 3}
一个类似的方法,归功于 Glenn Davis 在论文 \cite{SHOEMAKE-1985}中的观点,它从一个事实出发:连接 $\bfq_0$ 到 $\bfq_1$ 的大弧线上的任何一点都必须是其端点的线性组合 (因为这三个向量是共面的)。
利用方程 \eqRef{equ:slerp_angle} 计算出角度 $\Delta\theta$ 后,
%
%%\footnote{
%This formula can also be derived from \eqRef{equ:slerp_rot} by noticing that $\bfq_1 = \bfq_0 \cos \Delta\theta + \bfq_\bot \sin \Delta\theta$, isolating 
%
我们可以从方程 \eqRef{equ:q1} 中分离出 $\bfq_\bot$ 并且把它注入到方程 \eqRef{equ:slerp_rot} 。 应用恒等式 $\sin(\Delta\theta-t\Delta\theta)=\sin \Delta\theta\cos t\Delta\theta-\cos \Delta\theta\sin t\Delta\theta$,我们得到戴维斯公式(Davis' formula) (另一个推导参见 \cite{EBERLY-2010}论文),
%
\begin{align}
\eqbox{
\bfq(t)=\bfq_0\frac{\sin((1-t)\Delta\theta)}{\sin(\Delta\theta)}+\bfq_1\frac{\sin(t\Delta\theta)}{\sin(\Delta\theta)}
}
~.
\end{align}
%
这公式具有对称性的优点:定义反向插值器  $s=1-t$ 产生
%
\begin{align*}
\bfq(s)=
\bfq_1\frac{\sin((1-s)\Delta\theta)}{\sin(\Delta\theta)}
+
\bfq_0\frac{\sin(s\Delta\theta)}{\sin(\Delta\theta)}
~.
\end{align*}
%
这与 $\bfq_0$ 和 $\bfq_1$ 的角色互换的公式完全相同。
$\bfq_1'=-\bfq_1$$\bfq'(t)$,则沿$\bfx_0$到$\bfx_1$的最短路径产生向量$\bfx'(t)$。

\begin{figure}[htbp]
\begin{center}
\includegraphics{figures/slerp_fix}
\caption{确保沿 $\bfq_0$ 和 $\bfq_1$表示的方向之间的最短路径进行Slerp。左边:4D 空间中的四元数旋转平面,显示初始和最终方向四元数,以及两个可能的插值,$\bfq(t)$ 从 $\bfq_0$ 到 $\bfq_1$,并且 $\bfq'(t)$ 从 $\bfq_0$ 到 $-\bfq_1$。右边:3D空间中的向量旋转平面: 
因为 $\bfq_1=-\bfq_1$,我们有 $\bfx_1=\bfq_1\ot\bfx_0\ot\bfq_1^*=\bfq_1'\ot\bfx_0\ot\bfq_1'^*$,也就是说,两个四元数产生相同的旋转。
但是,插值的四元数 $\bfq(t)$ 乘以向量 $\bfx(t)$ 将带来从 $\bfx_0$ 到 $\bfx_1$ 的长路径, 
而用 $\bfq_1'=-\bfq_1$ 修正后产生的 $\bfq'(t)$, 乘以向量 $\bfx'(t)$ ,则沿着从 $\bfx_0$ 到 $\bfx_1$ 的最短路径旋转。}
\label{fig:slerp_fix}
\end{center}
\end{figure}

所有这些基于四元数的SLERP方法都需要注意确保沿最短路径,即旋转角 $\phi\leq\pi$,进行适当的插值。
由于 $SO(3)$ 的四元数双倍覆盖(参见 \secRef{sec:double_cover}),只有锐角 $\Delta\theta\leq\pi/2$ 的四元数之间的插值是按照最短路径进行的 (\figRef{fig:slerp_fix})。
测试这种情况并解决它很简单:如果 $\cos(\Delta\theta)=\bfq_0\tr\bfq_1<0$,则将 \eg~$\bfq_1$ 替换为 $-\bfq_1$ 并重新开始。



\subsection{四元数和等倾旋转:解释魔法}
\label{sec:isoclinic}

本节提供了关于四元数的两个有趣问题的几何见解,我们称之为“魔法”:
\begin{itemize}
\item
乘积 $\bfq\ot\bfx\ot\bfq^*$ 如何旋转向量 $\bfx$ ?
\item
为什么我们在通过 $\bfq=e^{\bfphi/2}=[\cos \phi/2 , \bfu\sin\phi/2]$ 构造四元数时需要考虑半角?
\end{itemize}
%
我们需要一个几何解释,就是超出代数证明方程 \eqRef{equ:quatRotFormula} 和 \secRef{sec:double_cover} 中的双倍覆盖事实的一些基本原理。

首先,让我们在这里再现方程 \eqRef{equ:quatRotMatrixForm} ,通过在方程 \eqRef{equ:quatMatrix} 中定义的 $\QL{\bfq}$ 和 $\QR{\bfq^*}$ 的四元数乘积矩阵,来表示四元数旋转动作,
%
\begin{align*}
\bfq\otimes
\bfx\otimes\bfq^*
&= \QR{\bfq^*}\,\QL{\bfq}\begin{bmatrix}
0 \\ \bfx
\end{bmatrix} 
= \begin{bmatrix}
0 \\ \bfR\,\bfx
\end{bmatrix} 
~.
\end{align*}
%
对于单位四元数 $\bfq$,四元数乘积矩阵 $\QL{\bfq}$ 和 $\QR{\bfq^*}$ 满足两个显著性质,
%
%
\begin{align}
 \Q{\bfq}\,\Q{\bfq}\tr &= \bfI_4 \\
 \det(\Q{\bfq}) &= +1 
 ~, 
\end{align}%
%
并且是 $SO(4)$的元素,即是 $\bbR^4$ 空间中的适当旋转矩阵。 
更具体地说,它们代表了一种特殊的旋转类型,叫做等倾旋转(\emph{isoclinic rotation}),就如我们在后面解释的那样。
因此,根据方程 \eqRef{equ:quatRotMatrixForm} ,四元数旋转对应于 $\bbR^4$ 中的两个链式等倾旋转。

为了解释四元数旋转的含义, % in $\bbR^3$, 
我们需要理解 $\bbR^4$ 中的等倾旋转。
为此,我们首先需要理解 $\bbR^4$ 中的一般旋转。
并且为了理解 $\bbR^4$中的旋转,我们需要回到 $\bbR^3$ ,其旋转实际上是平面旋转。
让我们一步一步地来讨论这个问题。


\paragraph{在 $\bbR^3$ 中的旋转:} 
%
在 $\bbR^3$ 中,让我们考虑向量 $\bfx$ 围绕表示的任意轴的向量 $\bfu$的旋转 --- 参见 \figRef{fig:isoclinic3},并回忆 \figRef{fig:rotation3d}。
在旋转时,与旋转轴 $\bfu$ 平行的向量不移动,而垂直于轴的向量在垂直于轴线的平面 $\pi$ 中旋转。 
对于一般向量 $\bfx$,向量在平面内的两个分量在这个平面内旋转,而轴向分量保持静止。
%
\begin{figure}[tb]
\begin{center}
\includegraphics{figures/isoclinic3}
\caption{%
在 $\bbR^3$ 中的旋转。 
向量 $\bfx$ 围绕旋转轴 $\bfu$ 的旋转描述了一个与轴垂直的平面上的圆周。 
与轴 $\bfx$ 平行的量不移动,并且由轴上的小红点表示。
右边的草图说明了平面和轴子空间上旋转点的根本不同的行为。 
}
\label{fig:isoclinic3}
\end{center}
\end{figure}


\paragraph{在 $\bbR^4$ 中的旋转} 
%
在 $\bbR^4$ 中,参见 \figRef{fig:isoclinic4},由于额外的维度,在 $\bbR^3$ 中的一维旋转轴变为新的二维平面。
%
\begin{figure}[tb]
\begin{center}
\includegraphics{figures/isoclinic4}
\caption{%
在 $\bbR^4$ 中的旋转。 
两个正交旋转是可能的,在两个正交平面 $\pi_1$ 和 $\pi_2$ 上面。 
在 $\pi_1$ 平面上旋转向量 $\bfx$ (未绘制) 导致向量的平行于该平面的两个分量 ($\pi_1$ 上的红点) 描绘了一个圆周 (红圈),
剩下在 $\pi_2$ 中的两部分不变 (红点)。
相反,在 $\pi_2$ 平面上的旋转(在 $\pi_2$ 中蓝色圆上的蓝点),剩下在 $\pi_1$ 中的两部分不变 (蓝点)。
右边的草图更好地说明了这一情况,它放弃在 $\bbR^4$中绘制不可显示的透视图,这可能会产生误导。
}
\label{fig:isoclinic4}
\end{center}
\end{figure}
%
第二个平面为第二个旋转提供了空间。
实际上,在 $\bbR^4$ 中的旋转包含在 4-space 中的两个正交平面上的两个独立的旋转。 
这意味着每个平面的每个 4-vector 在其自身的平面中旋转,并且一般 4-vectors 相对于一个平面的旋转不影响另一个平面中的向量分量。 
因为这个原因,这些平面被称为“不变量”(\emph{`invariant'})。

\paragraph{在 $\bbR^4$ 中的等倾旋转:} 
%
等倾旋转 (Isoclinic,来自希腊语,\emph{iso:} ``equal'', \emph{klinein:} ``to incline'') 是指在 $\bbR^4$ 中的旋转,在两个不变的平面中的旋转角度具有相同的大小。
然后,当两个角也有相同的符号时,%
\footnote{给定这两个不变平面,我们任意选择它们的方向,这样我们就可以在它们中关联正旋转角度和负旋转角度。}
我们说的是左等倾旋转(\emph{left-isoclinic rotations})。 
并且当它们有相反的符号时,我们说的是右等倾旋转(\emph{right-isoclinic rotations})。
%
等倾旋转的一个显著性质,我们已经在方程 \eqRef{equ:PQ_commute} 中看到过,就是左倾旋转和右等倾旋转相互交换,
\begin{align}
 \QR{\bfp}\,\QL{\bfq} = \QL{\bfq}\,\QR{\bfp}~. \label{equ:isoclinic_commute}
\end{align}

\paragraph{四元数在 $\bbR^4$ 和 $\bbR^3$ 中的旋转:}
% 
给定单位四元数 $\bfq=e^{\bfu\,\theta/2}$, 表示在 $\bbR^3$ 中围绕着旋转轴 $\bfu$ 转过一个角度 $\theta$ 的一个旋转,
矩阵 $\QL{\bfq}$ 是在 $\bbR^4$ 中的左倾旋转,对应于四元数 $\bfq$的左乘,
而  $\QR{\bfq^*}$ 是右等倾旋转,对应于四元数 $\bfq^*$的右乘。 
这些等倾旋转的角度大小正好是 $\theta/2$,%
\footnote{这可以通过提取等倾旋转矩阵的特征值来检查:它们由相位等于 $\pm\theta/2$的共轭复数对构成。}
并且不变平面是相同的。
然后,旋转表达方程 \eqRef{equ:quatRotMatrixForm} 在这里再次再现,
%
\begin{align*}
\begin{bmatrix}
0 \\ \bfx'
\end{bmatrix}=\bfq\otimes\bfx\otimes\bfq^*
&= \QR{\bfq^*}\,\QL{\bfq}\begin{bmatrix}
0 \\ \bfx
\end{bmatrix} 
~,
\end{align*}
%
表示到 4-vector $(0,\bfx)\tr$ 的两个链式等倾旋转,一个是左旋转,一个是右旋转,每个是 $\bbR^3$ 中所需旋转角度的一半。 
在 $\bbR^4$ 中的一个不变平面(参见 \figRef{fig:isoclinicQ})中,这两个半角会抵消,因为它们有相反的符号。 
%
\begin{figure}[tb]
\begin{center}
\includegraphics{figures/isoclinicQ}
\caption{%
在 $\bbR^4$ 中的四元数旋转。
两个链式等倾旋转,一个左 (具有相等的半角),一个右 (具有相反的半角),仅在一个不变平面上产生全角度的纯旋转。
}
\label{fig:isoclinicQ}
\end{center}
\end{figure}
%
在另一个平面上,它们相加得到总旋转角 $\theta$ 。 
如果我们从方程 \eqRef{equ:quatRotMatrixForm} 定义旋转矩阵, $\bfR_4$, 的结果,人们很容易意识到这点 (另见方程 \eqRef{equ:isoclinic_commute} ),
%
\begin{align}\label{equ:R4}
\bfR_4 \triangleq \QR{\bfq^*}\QL{\bfq}=\QL{\bfq}\QR{\bfq^*}= \begin{bmatrix}
1 & \bf0 \\
\bf0 & \bfR
\end{bmatrix}
~,
\end{align}
%
其中 $\bfR$ 是 $\bbR^3$中的旋转矩阵,它清晰地旋转 $\bbR^4$ 的子空间 $\bbR^3$ 中的向量,保持第四维不变。



这一论述有些超出了本文件的范围。 
它也是不完整的,因为除了方程 \eqRef{equ:R4} 中的结果之外,它并没有提供一个直观的或几何的解释,解释为什么我们需要做 $\bfq\ot\bfx\ot\bfq^*$ 而不是 \eg~$\bfq\ot\bfx\ot\bfq$ 。\footnote{就这么说吧, $\bfq\ot\bfx\ot\bfq^*$ 为旋转工作,如果 $\bfq$ 是单位四元数。实际上在 $\bbR^3$ 中如果 $\qv$ 是一个单位纯(\emph{pure})四元数,乘积 $\qv\ot\bfx\ot\qv$ 产生反射 (而不是旋转!) 。最后,乘积 $\bfq\ot\bfx\ot\bfq$,用 $\bfq$ ,一个单位非纯(\emph{non-pure})四元数,没有显著的特性。}
我们把它放在这里只是为了提供另一种解释四元数旋转的方法,希望读者能对它的机制有更多的直觉。
建议感兴趣的读者查阅有关 $\bbR^4$ 中等倾旋转的适当文献。






%=============================================================
\section{四元数约定。我的选择。}
\label{sec:conventions}

\subsection{四元数风味}

有几种方法可以确定四元数。它们基本上与四种二进制选择相关:
%
\begin{itemize}
\item
元素的顺序 --- 实数部分是第一个或最后一个:
\begin{align}
\bfq = \begin{bmatrix}
q_w \\ \qv
\end{bmatrix} 
\qquad \textit{vs.} \qquad 
\bfq = \begin{bmatrix}
\qv \\ q_w
\end{bmatrix}~.
\label{equ:quatOrder}
\end{align}

\item
乘法公式 --- 四元数代数的定义:
%
\begin{subequations}
\label{equ:quatAlg}
\begin{align}
ij=-ji=k 
\qquad \textit{vs.} \qquad 
ji=-ij=k~,
\label{equ:quatAlgDef}
\end{align}
%
分别对应不同的惯用手:
%
\begin{align}
\textit{右手(right-handed)
\qquad vs. \qquad
左手(left-handed)}~.
\label{equ:quatHand}
\end{align}
\end{subequations}
%
这意味着,给定一个旋转轴 $\bfu$,一个四元数 $\bfq_{right}\{\bfu\,\theta\}$ 使用右手规则,围绕 $\bfu$ 旋转向量一个角度 $\theta$ ,而另一个四元数 $\bfq_{left}\{\bfu\,\theta\}$ 使用左手规则。

\item
旋转算子的功能 --- 旋转坐标系或旋转向量:
%
\begin{align}
\textit{被动(Passive)
\qquad vs. \qquad
主动(Active)。}
\label{equ:quatAlibi}
\end{align}

\item
在被动情况下,操作方向 --- 局部到全局(local-to-global)或全局到局部(global-to-local):
%
\begin{align}
\bfx_{global} = \bfq\otimes\bfx_{local}\otimes\bfq^*
\qquad vs. \qquad 
\bfx_{local} = \bfq\otimes\bfx_{global}\otimes\bfq^*
\label{equ:quatInterpret}
\end{align}


\end{itemize}

这种多样的选择导致12种不同的组合。历史的发展使一些约定优于其它约定~\citep{CHOU-92,yazell-09}。
今天,在现有的文献中,我们发现了许多四元数的口味,如 
the Hamilton, 
the STS\footnote{Space Transportation System, commonly known as NASA's Space Shuttle.}, 
the JPL\footnote{Jet Propulsion Laboratory.}, 
the ISS\footnote{International Space Station.}, 
the ESA\footnote{European Space Agency.}, 
the Engineering, 
the Robotics, 
可能还有更多的宗派。 
其中许多形式可能是相同的,其它的则不是,但很少明确说明这一事实, 
很多工作对四元数关于上述四种选择的描述都不够充分。

这些差异以不明显的方式影响各自的旋转、组合等公式。 
因此,这些公式是不相容的,我们需要从一开始就作出明确的选择。

最常用的两个约定,也有最好的文档化,是 Hamilton (在方程 \eqsRef{equ:quatOrder}{equ:quatInterpret} 中左边的选项) 和 JPL (右边的选项,除了方程 \eqRef{equ:quatAlibi} 之外)。
 \tabRef{tab:Hamilton_vs_JPL} 显示了它们的特征概要。 
JPL 主要用于航空航天领域,而 Hamilton 在机器人等其它工程领域更为常见 --- 尽管这不应被视为一条规则。


\begin{table*}
\renewcommand{\arraystretch}{1.3}
\centering
\caption{Hamilton vs. JPL quaternion conventions \wrt the 4 binary choices}
\vspace{1ex}
\begin{tabular}{|cl|c|c|}
\hline
& Quaternion type & Hamilton & JPL \\
\hline\hline
1 & Components order & $(q_w \,,\, \qv)$ & $(\qv \,,\, q_w)$ \\
\hline
\multirow{2}{*}{2} & Algebra & $ij=k$ & $ij=-k$ \\
& Handedness & Right-handed & Left-handed \\
\hline
3 & Function & Passive & Passive
\\
\hline
\multirow{3}{*}{4} & Right-to-left products mean & Local-to-Global & Global-to-Local \\
& Default notation, $\bfq$ & $\bfq \triangleq \bfq_{\cG\cL}$ & $\bfq \triangleq \bfq_{\cL\cG}$ \\
& Default operation & $\bfx_\cG = \bfq\otimes\bfx_\cL\otimes\bfq^*$ & $\bfx_\cL = \bfq\otimes\bfx_\cG\otimes\bfq^*$ \\
\hline
\end{tabular}
\label{tab:Hamilton_vs_JPL}
\end{table*}

我的选择,早在方程 \eqRef{equ:quatAlgebra} 中就已经被采纳,是采用 Hamilton 约定, 
它是右手规则,在机器人学中广泛使用的许多软件库,如 Eigen, ROS, Google Ceres, 
以及大量关于利用IMUs进行姿态估计的 Kalman 滤波的文献中广泛使用~\citep[还有更多其它的例子]{CHOU-92,KUIPERS-99,PINIES-07,ROUSSILLON-11a,MARTINELLI-12}。

而 JPL 约定可能不太常用,至少在机器人领域是这样。 
这在~\citep{TRAWNY-05-QUAT}中有广泛的描述,这是一个目标和范围非常接近于当前的参考著作,但仅集中在 JPL 约定。 
JPL 四元数被使用在 JPL 文献中 (显然的) 并在 Li, Mourikis, Roumeliotis, 和其同事撰写的关键论文中使用 (参见 \eg~\citep{LI-2012,LI-14}),这描述来自 \citeauthor{TRAWNY-05-QUAT}' 的文件。 
这些工作是处理视觉惯性测距和 SLAM 时的灵感的根本来源 --- 这就是我们所做的。 

在本节的其余部分中,我们将更深入地分析这两个四元数约定。


%-------------------------------------------------------------
\subsubsection{四元数组分的顺序}

虽然不是最基本的,但 Hamilton 和 JPL 四元数之间最显著的区别在于组分的顺序,标量部分要么在第一位置 (Hamilton) ,要么在最后位置 (JPL) 。 
这种变化的影响是相当明显的,不应代表对解释的巨大挑战。 
事实上,一些四元数实数组分在最后的软件 (\eg, the C++ library Eigen) 仍然被认为是使用 Hamilton 约定,只要其它三个方面得以维持。

我们使用下标 $(w, x, y, z)$ 表示四元数组分,以提高清晰度,而不是其它常用的 $(0, 1, 2, 3)$ 。
当改变顺序时, $q_w$ 总是表示实部,而 $q_0$ 是否也会这样做还不清楚 
--- 某些情况下,人们可能会发现一些问题,比如 $\bfq=(q_1, q_2, q_3, q_0)$ ,用 $q_0$ 表示实数并放在最后,但是在一般情况下 $\bfq=(q_0, q_1, q_2, q_3)$,最后的实部是 $q_3$。%
\footnote{另见脚注 \ref{ftn:quatComponents}。} 
当从一个约定变化到另一个约定时,我们必须小心公式中涉及全部 $4\times4$ 或 $3\times4$ 四元数相关矩阵, 
因为它们的行和/或列需要交换。 
这不难做到,但可能很难检测,因此容易出错。

关于组分顺序的两个奇怪之处是:
%
\begin{itemize}
\item
先有实数部分,四元数自然被解释为一个扩展复数,这是是熟悉的形式 \emph{real+imaginary}。
我们中的一些人对此表示满意,可能是因为这个原因。
\item
实数部分在最后,四元数以向量形式表示,
%
$\bfq=\begin{bmatrix}
x,y,z,w
\end{bmatrix}\in\bbH$, 
%
其格式与射影 3D 空间中的齐次向量完全等价, 
%
$\bfp=\begin{bmatrix}
x,y,z,w
\end{bmatrix}\in\bbP^3$, 
%
在这两种情况下 $x,y,z$ 都与三个笛卡尔轴有明确地标识。 
在处理 3D 几何问题时,这使得四元数和齐次向量上的运算代数更加一致, 
特别是 (但不仅如此) 如果齐次向量被约束到单位球面 $\norm{\bfp}=1$ 上。
\end{itemize}

%-------------------------------------------------------------
\subsubsection{四元数代数的规范}


Hamilton 约定定义了 $ij=k$ 并因此,
%
\begin{align}
i^2 = j^2 = k^2 = ijk = -1~,\quad ij = -ji = k~, \quad jk = -kj = i~, \quad ki = -ik = j~,
\end{align}
%
而 JPL 约定定义 $ji=k$ 并因此其四元数代数变成为,
%
\begin{align}
i^2 = j^2 = k^2 = -ijk = -1~,\quad -ij = ji = k~, \quad -jk = kj = i~, \quad -ki = ik = j~.
\end{align}

有趣的是,这些细微的符号变化保留了四元数作为旋转算子的基本性质。 
数学上,关键的结果是方程 \eqRef{equ:quatProdVec} 中叉积符号的变化,这导致四元数惯用手的变化~\citep{SHUSTER-93}:
Hamilton 使用 $ij=k$ 并因此是右手的,即,它按照右手规则旋转向量; JPL 使用 $ji=k$ 并因此是左手的~\citep{TRAWNY-05-QUAT}. 
作为相反符号的左手和右手旋转,我们可以说它们的四元数 $\bfq_{left}$ 和 $\bfq_{right}$ 相关于,
%
\begin{align}
\bfq_{\textit{left}}= \bfq_{\textit{right}}^*~.
\end{align}



%-------------------------------------------------------------
\subsubsection{旋转算子的功能}

我们已经看到了如何在3D中旋转向量。这在~\citep{SHUSTER-93} 中被称为主动(\emph{active})解释, 
因为算子 (这影响所有的旋转算子) 主动旋转向量,
%
\begin{align}
\bfx' &=\bfq_{\textit{active}}\otimes\bfx\otimes\bfq_{\textit{active}}^* ~,
& 
\bfx' &= \bfR_{\textit{active}}\,\bfx~.
\end{align}

观察向量 $\bfx$ 上 $\bfq$ 和 $\bfR$ 的效果的另一种方法是,考虑向量是稳定的,但我们已经将我们的视角旋转了 $\bfq$ 或 $\bfR$ 指定的量。 
这被称为坐标系变换(\emph{frame transformation}),并且它在~\citep{SHUSTER-93}中称为被动(\emph{passive})解释,因为向量不移动,
%
\begin{align}
\bfx_\cB &= \bfq_{\textit{passive}}\ot\bfx_\cA\ot\bfq_{\textit{passive}}^*~,
&
\bfx_\cB&=\bfR_{\textit{passive}}\,\bfx_\cA~,
\end{align}
%
其中 $\cA$ 和 $\cB$ 是两个笛卡尔坐标系,并且 $\bfx_\cA$ 和 $\bfx_\cB$ 是这些坐标系中相同向量 $\bfx$ 的表达式。 
有关解释和正确的符号,请参阅下面的内容。


主动解释和被动解释由相互相反的算子控制,即, 
%
\begin{align*}
\bfq_{\textit{active}} &= \bfq_{\textit{passive}}^* ~,
& 
\bfR_{active} &= \bfR_{passive}\tr ~.
\end{align*}
%
Hamilton 和 JPL 都使用被动约定。 

\paragraph{方向余弦矩阵}
有一些作者认为被动算子不是旋转算子,而是一个方向规范,叫做方向余弦矩阵(\emph{direction cosine matrix}),
%
\begin{align}
\bfC = \begin{bmatrix}
c_{xx} & c_{xy} & c_{zx} \\
c_{xy} & c_{yy} & c_{zy} \\
c_{xz} & c_{yz} & c_{zz} 
\end{bmatrix}
~,
\end{align}
%
其中每个组分 $c_{ij}$ 都是源坐标系中的 $i$ 轴与目标坐标系中的 $j$ 轴之间的夹角余弦。我们有特征值,
%
\begin{align}
\bfC \equiv \bfR_{\textit{passive}}
~.
\end{align}


%-------------------------------------------------------------
\subsubsection{旋转算子的方向}

在被动情况下,第二解释源与旋转矩阵和四元数的操作方向相关, 
可以从局部坐标系转换为全局坐标系,也可以从全局坐标系转换为局部坐标系。 

给定两个笛卡尔坐标系 $\cG$ 和 $\cL$,我们将 $\cG$ 和 $\cL$ 标识为全局和局部坐标系。
``全局(Global)'' 和 ``局部(local)'' 是相对定义,即 $\cG$ 是全局的,相对于 $\cL$,并且 $\cL$ 是局部的,相对于 $\cG$ --- 换句话说, $\cL$ 是在参考坐标系 $\cG$中指定的坐标系。\footnote{其它通用的 \{global, local\} 坐标系是 \{parent, child\} 和 \{world, body\}。第一种方法在一个系统中涉及两个以上的坐标系时比较方便 (例如,仿人机器人中每个运动环节的坐标系);第二种方法对一个被标识为世界的唯一参考坐标系中的实体 (例如,飞机、汽车) 的移动比较方便。} 
我们指定 $\bfq_{\cG\cL}$ 和 $\bfR_{\cG\cL}$ 分别是从坐标系 $\cL$ 到坐标系 $\cG$ 的四元数和旋转矩阵转换向量, 
从某种意义上说,向量 $\bfx_\cL$ 在坐标系 $\cL$ 中被坐标系 $\cG$ 用四元数和矩阵乘积表示,
%
\begin{align}
\bfx_\cG &= \bfq_{\cG\cL}\otimes\bfx_\cL\otimes\bfq_{\cG\cL}^*~, &
\bfx_\cG &= \bfR_{\cG\cL}\,\bfx_\cL~.
\label{equ:local_to_global}
\end{align}
%
相反的转换,从 $\cG$ 到 $\cL$,是用
%
\begin{align}
\bfx_\cL &= \bfq_{\cL\cG}\otimes\bfx_\cG\otimes\bfq_{\cL\cG}^*
~,
&
\bfx_\cL &= \bfR_{\cL\cG}\,\bfx_\cG~,
\end{align}
%
其中
%
\begin{align}
\bfq_{\cL\cG} &= \bfq_{\cG\cL}^*~,
& 
\bfR_{\cL\cG} &= \bfR_{\cG\cL}\tr~. \label{equ:localVsGlobal}
\end{align}


Hamilton 使用局部到全局(local-to-global)作为以坐标系 $\cG$ 表示的坐标系 $\cL$ 的默认规范, 
%
\begin{align}
\bfq_{\textit{Hamilton}} \triangleq \bfq_{[\textit{with~respect~to}][\textit{of\,}]}=\bfq_{[\textit{to}][\textit{from}]}=\bfq_{\cG\cL}~, 
\end{align}
%
而 JPL 使用相反的全局到局部(global-to-local)转换,
%
\begin{align}
\bfq_{\textit{JPL}} \triangleq \bfq_{[\textit{of}\,][\textit{with~respect~to}]}=\bfq_{[\textit{to}][\textit{from}]}=\bfq_{\cL\cG}~.
\end{align}

请注意
%
\begin{align}
\bfq_{\textit{JPL}}
\triangleq    \bfq_{\cL\cG,left}
=    \bfq_{\cL\cG,\textit{right}}^*
=    \bfq_{\cG\cL,\textit{right}}
\triangleq    \bfq_{\textit{Hamilton}}%^*
~,
\label{equ:quatEquivalences}
\end{align}
%
这并不特别有用,但它说明了在混合约定时容易混淆。
还要注意,我们可以得出结论, $\bfq_{JPL} = \bfq_{Hamilton}$,但这远远不是一个漂亮的结果,只是一个非常混乱的来源, 
因为等式只存在于四元数值中, 
但是这两种四元数在公式中使用时,意味着和代表着不同的东西。




%%-------------------------------------------------------------
%\subsection{Notation}
%
%An often underestimated source of confusion when dealing with quaternion algebra is related to notation. 
%We believe notation should be clear, lightweight and unambiguous. 
%Of course, this applies not only to quaternions and rotation matrices, but also to the points and vectors manipulated by these. 
%A good notation requires considering many conflicting aspects: 
%%
%\begin{itemize}
%
%\item 
%Clearly distinguish scalars, vectors, matrices and functions with different font styles.
%
%\item
%Use the main letter to signify the physical dimension. Use prefixes, subscripts, superscripts and/or accents for details and particularities.
%
%\item Avoid tiny elements such as tildes $\tilde\bfq$, hats $\hat\bfq$, bars $\bar\bfq$, $\ul\bfq$, and other accents, as much as possible.
%
%%\item Avoid making extensive use of unusual Greek symbols. 
%%If we cannot tell a symbol name, then we cannot read a formula. What is $\Xi$? And $\Upsilon$?
%
%\item Avoid certain combinations of subscripts and superscripts on the left and right hand sides, %\eg, $^ix_j$, 
%especially when they appear in multiple levels, \eg, $^{C_i}x_{F_j}$. 
%They produce formulas such as $^iu_j=\frac{^{C_i}x_{F_j}}{^{C_i}z_{F_j}}$ (which is the pinhole camera model) that are difficult to read because the main variables, $x$ and $z$, are not salient enough.
%
%\item Make composition of chains obviously and unambiguously readable. 
%For example, the rotation matrix $\bfR_\cA^\cB$ or the quaternion $ _\cA^\cB\bfq$ are ambiguous because we do not know if they transform ``$\cA$ to $\cB$\,'' or ``$\cB$ to $\cA$''.
%
%
%\item
%Provide easy 1:1 translation to readable programming code. This last point becomes increasingly important as most of the works we produce are meant to be translated into algorithms.
%
%
%\end{itemize}
%
%Our notation derives from the following rules,
%%
%\begin{enumerate}
%\item
%Scalars are $a,x,\omega$; vectors are $\bfa,\bfx,{\bm \omega}$, matrices are $\bfA, \bfX, {\bf\Omega}$; functions are $f(), g()$. 
%
%\item
%We use decorations to provide details of a given magnitude. As an example, we use~$\bfa$ for \emph{acceleration}, and $\delta\bfa,~ \bfa_m,~ \bfa_b,~ \hat\bfa,~ \bfA,~ \bfa_n,~ \bfA_n$ respectively for \emph{acceleration error, measured acceleration, accelerometer bias, mean of the acceleration estimate, covariance of the acceleration estimate, acceleration noise, covariance of the acceleration noise}. 
%
%\item
%The only accent we use is the hat, $\hat\bfx$, to signify the mean of the Kalman filter Gaussian estimate. Error-state values are noted $\delta\bfx$. 
%
%\item
%The general translation specification $\bft$ needs 3 decorations: a translation \emph{from} point $\cB$ / \emph{to} point $\cC$ / \emph{expressed in} frame $\cA$,
%%
%\begin{align}
%\bft_{[expressed~in],[initial~point][end~point]}~.
%\end{align}
%%
%This produces forms such as $\bft_{\cA,\cB\cC}$.  
%We define two convenient simplifications:
%%
%\begin{itemize}
%\item
%When $\bft_{\cA,\cB\cC}$ refers to a point $\cP$, we consider its origin at the origin of the reference frame, \ie, $\cB=\cA$, giving 
%%
%\begin{align}
%\bfp_\cA\triangleq\bft_{\cA,\cA \cP}~. \label{equ:point}
%\end{align}
%%
%\item
%When $\bft_{\cA,\cB\cC}$ refers to a free vector $\bfv=\ol{\cB\cC}$, we may simply write 
%%
%\begin{align}
%\bfv_\cA\triangleq\bft_{\cA,\ol{\cB\cC}}~. \label{equ:vector}
%\end{align}
%%
%\end{itemize}
%%
%In both cases we keep the reference frame $\cA$ where the point or vector is expressed in. In case of no ambiguity, when this frame is the world frame $\cW$ of our application we allow us to drop this decoration too. For example, the position $\cP$ of the robot in the world frame may be denoted simply by $\bfp$,
%%
%\begin{align}
%\bfp \triangleq \bfp_\cW = \bft_{\cW,\cW \cP}~.  \label{equ:worldPoint}
%\end{align}
%
%
%\item
%The general orientation specification, $\bfq$ or $\bfR$, requires 2 decorations: a transformation \emph{from} frame $\cB$ / \emph{to} frame $\cA$, or equivalently, the orientation \emph{of} frame $\cB$ / \emph{\wrt} frame $\cA$,
%%
%\begin{align}
%\bfq_{[to][from]} &\equiv \bfq_{[with~respect~to][of]}
%~,
%&
%\bfR_{[to][from]} &\equiv \bfR_{[with~respect~to][of]}~.
%\end{align}
%%
%This produces forms such as $\bfR_{\cA\cB}$ that lead unambiguously to $\bfx_\cA=\bfR_{\cA\cB}\bfx_\cB$. 
%Stacked or composed transforms produce  chains such as 
%%
%\begin{align}
%\bfx_\cA &= \bfq_{\cA\cB}\otimes\bfq_{\cB\cC}\otimes\bfx_\cC\otimes\bfq_{\cB\cC}^*\otimes\bfq_{\cA\cB}^*~,
%& 
%\bfx_\cA &= \bfR_{\cA\cB}\,\bfR_{\cB\cC}\,\bfx_\cC 
%~. \label{equ:2frames}
%\end{align}
%%
%which are readable and not prone to error. 
%Notice how each of the two frame identifiers is always at the side of the entity nearby, creating a \emph{chain}  of identifiers.
%% (though in the quaternion transform above, this only applies to the quaternions chain on the left of the vector, the ones that are not conjugated). 
%In the quaternion case, the chain of identifiers can be made more salient by recalling that $\bfq_{\cA\cB}^*=\bfq_{\cB\cA}$ and thus 
%%
%\begin{align}
%\bfx_\cA = \bfq_{\cA\cB}\otimes\bfq_{\cB\cC}\otimes\bfx_\cC\otimes\bfq_{\cC\cB}\otimes\bfq_{\cB\cA}~.
%\end{align} 
%%
%This allows us to easily construct and/or identify frame composition chains with absolutely no ambiguity. 
%
%Oftentimes, the world frame $\cW$ and the frame of the main moving body $\cB$ may be omitted (in cases where there is no ambiguity), yielding
%%
%\begin{align}
%\bfq &\triangleq \bfq_{\cW \cB}~,
%&
%\bfR &\triangleq \bfR_{\cW \cB}~. \label{equ:rotSimp}
%\end{align}
%% 
%This, and the simplifications for points and vectors \eqsRef{equ:point}{equ:worldPoint} above, lead to 
%%
%\begin{align}
%\bfx &= \bfq\otimes\bfx_\cB\otimes\bfq^* ~, 
%&
%\bfx &= \bfR\,\bfx_\cB ~,
%\end{align}
%%
%which are very light and easy to read.
%They express the transformation of vector $X$ from the body frame $\cB$ to the world frame $\cW$.
%
%\item We use only right-hand subscripts for frame decorations. This way, translating formulas into code and vice-versa becomes straightforward. In the code, the first letter before the first underscore is always the physical magnitude. For example, these formula and code are equivalent,
%%
%\begin{align}
%\bfx_\cA &= \bfR_{\cA\cB}\,\bfR_{\cB\cC}\,\bfx_\cC~,
%& 
%\texttt{ x\_A = R\_A\_B * R\_B\_C * x\_C}~.
%\end{align}
%
%
%
%\end{enumerate}
%




%=============================================================
\section{扰动、导数和积分}


%\section{Definition of the derivatives }

\subsection{在 $SO(3)$ 中的加法和减法算子}

\begin{figure}[tb]
\centering
\includegraphics{figures/manifold}
\caption{在 $\bbR^4$ 中,$S3$流形是一个单位球体,这里用一个单位圆(蓝色)表示,所有单位四元数都在其中。
流形的切线空间是超平面 $\bbR^3$,这里用一条线(红色)表示。
左(\emph{Left}):$\Exp()$ 和 $\Log()$ 算子将 $\bbR^3$ 的元素映射到/回 $S3$的元素。
右(\emph{Right}):$\oplus$ 和 $\ominus$ 算将流形的元素与切线空间中的元素相关联。(同样,这些插图说明了 $SO(3)$ 的流形。)}
\label{fig:manifold}
\end{figure}

在向量空间 $\bbR^n$中,加法和减法运算是用正规的加法 `$+$' 和减法 `$-$' 运算来执行的。
在 $SO(3)$ 中这是不可能的,但是可以定义等价算子来建立一个合适的微积分语料库。 

因此,我们定义了在 $\sR\in SO(3)$ 中的元素之间的加法和减法算子, $\oplus,\ominus$,还有 $\sR$ 处在切线空间 $\bth\in\bbR^3$ 中的元素 $\bth\in\bbR^3$ ,如下所示。

\paragraph{加法算子。}
加法(`plus')算子 $\oplus:SO(3)\times\bbR^3\to SO(3)$ 生成一个在 $SO(3)$ 中的元素 $\sS$ ,其结果为一个在 $SO(3)$ 中的参考元素 $\sR$ 的组合,并有一个(通常较小)的旋转。这个旋转是由参考元素 $\sR$ 处与 $SO(3)$ 流形相切的向量空间 $\bth\in\bbR^3$ 中的向量指定的,也就是说,
%
\begin{align}
\sS = \sR\oplus \bth &\te \sR\circ\Exp(\bth) && \sR,\sS\in SO(3),~ \bth\in\bbR^3 
~.
\end{align}
%
注意,可以为 $SO(3)$的任何表示定义此算子。特别是,对于四元数和旋转矩阵,我们有,
%
\begin{align}
\bfq_\sS &= \,\bfq_\sR\oplus\bth = \bfq_\sR\ot\Exp(\bth) \\
\bfR_\sS &= \bfR_\sR\oplus \bth = \bfR_\sR\tdot\Exp(\bth) 
~.
\end{align}

\paragraph{减法算子。}
减法(`minus')算子 $\ominus:SO(3)\times SO(3)\to\bbR^3$ 是加法算子的逆运算。它返回在 $SO(3)$ 中的两个元素之间的向量角差 $\bth\in\bbR^3$ 。这个差值在与参考元素 $\sR$ 相切的向量空间中表示,
%
\begin{align}
\bth=\sS\ominus \sR
&\te \Log(\sR\inv \circ \sS)     && \sR,\sS\in SO(3),~ \bth\in\bbR^3  
~,
\end{align}
%
其中对于四元数和旋转矩阵来说读到,
%
\begin{align}
\bth &= \,\,\bfq_\sS\ominus\bfq_\sR\, = \Log(\bfq_\sR^*\ot\bfq_\sS)                      \\
\bth &= \bfR_\sS\ominus\bfR_\sR = \Log(\bfR_\sR\tr\,\bfR_\sS)
~.
\end{align}

\bigskip
在这两种情况下,请注意,即使向量差值 $\bftheta$ 通常被认为是很小的值,上述定义对于 $\bftheta$ 的任何值(直到 $SO(3)$ 流形的第一个覆盖,即对于角 $\theta<\pi$)仍然有效。

\subsection{四种可能的导数定义}



\subsubsection{从向量空间到向量空间的函数}

标量和向量情况遵循导数的经典定义:给定一个函数 $f:\bbR^m\to\bbR^n$,我们使用 $\{+,-\}$ 将导数定义为
%
\begin{align}
\dpar{f(\bfx)}{\bfx} &\te \lim_{\delta\bfx\to0}\frac{f(\bfx+\delta\bfx)-f(\bfx)}{\delta\bfx} &&\in \bbR^{n\times m} \label{equ:derivative_vector}
\end{align}
%
欧拉积分产生形式的线性表达式
%
\begin{align*}
f(\bfx+\Delta\bfx) &\approx f(\bfx) + \dpar{f(\bfx)}{\bfx}\Delta\bfx
& \in \bbR^n
\end{align*}

\subsubsection{从 $SO(3)$ 到 $SO(3)$ 的函数}

给定一个函数 $f:SO(3) \to SO(3)$ 并有 $\sR\in SO(3)$ 和一个局部的小角度变量 $\bth\in\bbR^3$,我们使用 $\{\oplus,\ominus\}$ 将导数定义为
%
\begin{align}
\dpar{f(\sR)}{\bth} 
&\te \lim_{\delta\bth\to0}\frac{f(\sR\oplus\delta\bth)\ominus f(\sR)}{\delta\bth}  && \in \bbR^{3\times 3}\\
&= \lim_{\delta\bth\to0}\frac{\Log\big(f\inv(\sR)\,f(\sR\Exp(\delta\bth))\big)}{\delta\bth} \label{equ:derivative_SO3}
\end{align}
%
欧拉积分产生形式的表达式,
%
\begin{align*}
f(\sR\oplus\Delta\bth) &\approx f(\sR)\,\oplus\,\dpar{f(\sR)}{\bth}\,\Delta\bth
 \te f(\sR)\Exp\left(\dpar{f(\sR)}{\bth}\Delta\bth\right)
 & \in SO(3)
\end{align*}




\subsubsection{从向量空间到 $SO(3)$ 的函数}

这种情况,对于函数 $f:\bbR^m\to SO(3)$,我们使用 `+' 表示向量的扰动,并使用 `$\ominus$' 表示 $SO(3)$ 的差分,
%
\begin{align}
\dpar{f(\bfx)}{\bfx} &\te \lim_{\delta\bfx\to0} \frac{ f(\bfx+\delta\bfx)\ominus f(\bfx)}{\delta\bfx} && \in \bbR^{3\times m} \label{equ:dif_RtoSO3}\\
&= \lim_{\delta\bfx\to0} \frac{\Log(f\inv(\bfx) f(\bfx+\delta\bfx))}{\delta\bfx}
\end{align}
%
欧拉积分产生形式的表达式,
%
\begin{align*}
f(\bfx+\Delta\bfx) &\approx f(\bfx)\,\oplus\,\dpar{f(\bfx)}{\bfx}\,\Delta\bfx
 \te f(\bfx)\,\Exp\left(\dpar{f(\bfx)}{\bfx}\Delta\bfx\right)
 & \in SO(3)
\end{align*}

\subsubsection{从 $SO(3)$ 到向量空间的函数}

这种情况,对于函数 $f: SO(3)\to\bbR^n$,我们使用 `$\oplus$' 表示 $SO(3)$ 的扰动,并使用 `$-$' 表示向量差分,
%
\begin{align}
\dpar{f(\sR)}{\bth} &\te \lim_{\delta\bth\to0} \frac{f(\sR\oplus\delta\bth) - f(\sR)}{\delta\bth} && \in \bbR^{n\times 3} \label{equ:jacobian_SO3_Rn}\\
&= \lim_{\delta\bth\to0} \frac{f(\sR\Exp(\delta\bth)) - f(\sR)}{\delta\bth}
\end{align}
%
欧拉积分产生形式的表达式,
%
\begin{align*}
f(\sR\oplus\delta\bth) &\approx f(\sR)+\dpar{f(\sR)}{\bth}\,\Delta\bth
 \te f(\sR)+\Exp\left(\dpar{f(\sR)}{\bth}\Delta\bth\right)
 & \in SO(3)
\end{align*}

%=============================================================

\subsection{有用,并且非常有用,旋转的 Jacobians 矩阵}

让我们考虑围绕单位轴 $\bfu$,以 $\theta$ 弧度旋转到向量 $\bfa$。让我们用三种等价的形式来表示旋转规范,即 $\bftheta=\theta\bfu$, $\bfq=\bfq\{\bftheta\}$ 和 $\bfR=\bfR\{\bftheta\}$。 
我们对相对于不同量值的旋转结果的 Jacobians 矩阵感兴趣。


%-------------------------------------------------------------
\subsubsection{相对于向量的 Jacobian 矩阵}

向量 $\bfa$ 旋转的导数相对于这个向量是平凡的,
%
\begin{align}
\eqbox{
\dpar{(\bfq\ot\bfa\ot\bfq*)}{\bfa} = \dpar{(\bfR\,\bfa)}{\bfa} = \bfR
}
~.
\end{align}
%




%-------------------------------------------------------------
\subsubsection{相对于四元数的 Jacobian 矩阵}


相反,相对于四元数 $\bfq$ 的旋转导数是很棘手的。 
为了方便起见,我们对四元数使用了轻量级符号, $\bfq=[w~\bfv] = w+\bfv$ 。
我们使用方程 \eqRef{equ:quatProdPure} , \eqRef{equ:quatCommutatorPure},并标识为 $\bfa \times (\bfb \times \bfc) = (\bfc \times \bfb) \times \bfa = (\bfa \tr \bfc)\,\bfb - (\bfa \tr \bfb)\,\bfc$,以扩展基于旋转的四元数方程 \eqRef{equ:sandwichProd} ,如下所示,
%
\begin{align} \label{equ:drot_dtheta}
\begin{split}
\bfa' &= \bfq\ot\bfa\ot\bfq* \\
&= (w+\bfv)\ot\bfa\ot(w-\bfv) \\
&= w^2\bfa + w(\bfv\ot\bfa - \bfa\ot\bfv) - \bfv\ot\bfa\ot\bfv \\
&= w^2\bfa + 2w(\bfv\tcross\bfa) - \big[(-\bfv\tr\bfa+\bfv\tcross\bfa)\ot\bfv \big]\\
&= w^2\bfa + 2w(\bfv\tcross\bfa) - \big[(-\bfv\tr\bfa)\,\bfv+(\bfv\tcross\bfa)\ot\bfv \big]\\
&= w^2\bfa + 2w(\bfv\tcross\bfa) - \big[(-\bfv\tr\bfa)\,\bfv - \cancel{(\bfv\tcross\bfa)\tr\bfv}+(\bfv\tcross\bfa)\tcross\bfv \big]\\
%&= w^2\bfa + 2w(\bfv\tcross\bfa) - \big[(-\bfv\tr\bfa)\,\bfv+(\bfv\tcross\bfa)\tcross\bfv \big]\\
%&= w^2\bfa + 2w(\bfv\tcross\bfa) - \big[(-\bfv\tr\bfa)\,\bfv - \bfv\tcross(\bfv\tcross\bfa) \big]\\
&= w^2\bfa + 2w(\bfv\tcross\bfa) - \big[(-\bfv\tr\bfa)\,\bfv + (\bfv\tr\bfv)\,\bfa - (\bfv\tr\bfa)\,\bfv \big]\\
%&= w^2\bfa + 2w(\bfv\tcross\bfa) + (\bfv\tr\bfa)\,\bfv + (\bfv\tr\bfa)\,\bfv-(\bfv\tr\bfv)\,\bfa\\
&= w^2\bfa + 2w(\bfv\tcross\bfa) + 2(\bfv\tr\bfa)\,\bfv - (\bfv\tr\bfv)\,\bfa
~.
\end{split}
\end{align}%
%
有了这个,我们可以提取导数 $\dparil{\bfa'}{w}$ 和 $\dparil{\bfa'}{\bfv}$,
%
\begin{align}
\dpar{\bfa'}{w} &= 2(w\bfa + \bfv\tcross\bfa) \\
\begin{split}
\dpar{\bfa'}{\bfv} &= -2w\hatx{\bfa} + 2(\bfv\tr\bfa\,\bfI+\bfv\,\bfa\tr) - 2\bfa\,\bfv\tr 
\\
&= 2(\bfv\tr\bfa\,\bfI+\bfv\,\bfa\tr - \bfa\,\bfv\tr - w\hatx{\bfa} )
~,
\end{split}
\end{align}%
%
产生
%
\begin{align} \label{equ:drot_dq}
\eqbox{
\dpar{(\bfq\ot\bfa\ot\bfq*)}{\bfq} = 
%\dpar{(\bfR\,\bfa)}{\bfq} = 
2\begin{bmatrix}
~w\,\bfa + \bfv\tcross\bfa ~~ \big| ~  \bfv\tr\bfa\,\bfI_3+\bfv\,\bfa\tr - \bfa\,\bfv\tr -w\hatx{\bfa}
~\end{bmatrix} \in \bbR^{3\times 4}
}~.
\end{align}



%\subsubsection{Right Jacobian of $SO(3)$}
%
%\begin{figure}[tbp]
%\begin{center}
%\includegraphics{figures/right_jac}
%\caption{The right Jacobian $\bfJ_r=\dparil{\delta\bfphi}{\delta\bftheta}$ maps variations $\delta\bftheta$ around the parameter $\bftheta$ into variations $\delta\bfphi$ on the vector space tangent to the manifold at the point $\Exp{\bftheta}$. }
%\label{fig:right_jac}
%\end{center}
%\end{figure}
%
%
%Let us define the `minus' operator $\ominus$ in $SO(3)$ that returns the rotation increment as a vector $\delta\bfphi$ in the tangent space $\so(3)$, that is,
%%
%\begin{align}
%\delta\bfphi = r_2 \ominus r_1 \triangleq \Log(r_1\inv\circ r_2) = \Log(\bfR_1\tr\bfR_2) = \Log(\bfq_1^*\ot\bfq_2) \in \bbR^3
%\end{align}
%%
%where $r_1,r_2$ are two elements of $SO(3)$, and $\circ$ is their composition. This `minus' operator allows us to define derivatives in $SO(3)$ in a way akin to those in Euclidean space, 
%%
%\begin{align}
%\bfJ_r(\bftheta) = \dpar{\delta\bfphi}{\delta\bftheta} 
%\triangleq 
%\lim_{\delta\bftheta\to0} \frac{r(\bftheta+\delta\bftheta)\ominus r(\bftheta)}{\delta\bftheta}
%\end{align}
%%
%%
%This derivative is a matrix, $\bfJ_r(\bftheta)\in\bbR^{3\times3}$, known as the right Jacobian of $SO(3)$.
%It maps variations around $\bftheta$ in the parameter space into variations in the  space tangent to the manifold at the point $r(\bftheta)$, see \figRef{fig:right_jac}. 
%The right Jacobian is therefore independent of the representation chosen for $SO(3)$; we express it here in matrix and quaternion forms:
%%
%\begin{align}
%\bfJ_r(\bftheta) 
%&= \dpar{\Log(\bfR\tr\{\bftheta\}\,\bfR\{\bftheta+\delta\bftheta\})}{\delta\bftheta} \\
%\bfJ_r(\bftheta) 
%&= \dpar{\Log(\bfq^*\{\bftheta\}\ot\bfq\{\bftheta+\delta\bftheta\})}{\delta\bftheta}
%\end{align}
%%
%Applying Euler integration to the above we get,
%%
%\begin{align}
%\bfJ_r(\bftheta)\,\delta\bftheta \approx \Log( r\inv(\bftheta) \circ \,r(\bftheta+\delta\bftheta))
%~,
%\end{align}
%%
%which leads after taking the $\Exp$ on both sides to an expression of the perturbed rotation operator,
%%
%\begin{align}
% r(\bftheta+\delta\bftheta) \approx  r(\bftheta)\circ\Exp(\bfJ_r(\bftheta)\,\delta\bftheta)
% ~.
%\end{align}
%%
%This result is also independent of the representation of the manifold (\ie, $\bfR$ or $\bfq$). We show the expressions of the perturbed rotation matrix,
%%
%\begin{align}
%\eqbox{\bfR\{\bftheta+\delta\bftheta\} \approx \bfR\{\bftheta\}\Exp(\bfJ_r(\bftheta)\,\delta\bftheta)}
%\end{align}
%%
%and the perturbed quaternion,
%%
%\begin{align}
%\eqbox{\bfq\{\bftheta+\delta\bftheta\} \approx \bfq\{\bftheta\}\ot\Exp(\bfJ_r(\bftheta)\,\delta\bftheta)}
%\end{align}
%
%The right Jacobian of $SO(3)$ admits a closed form \citep[page 40]{CHIRIKJIAN-12},
%%
%\begin{align}
%\bfJ_r(\bftheta) = \bfI - \frac{1-\cos\norm{\bftheta}}{\norm{\bftheta}^2}\hatx{\bftheta} + \frac{\norm{\bftheta}-\sin\norm{\bftheta}}{\norm{\bftheta}^3}\hatx{\bftheta}^2
%%\in\bbR^{3\times3}
%~.
%\end{align}
%%
%For small angles $\bftheta$ it can be approximated to
%%
%\begin{align}
%\bfJ_r(\bftheta) \approx \bfI - \frac12\hatx{\bftheta}~.
%\end{align}
%
%May I find the time and inspiration to develop it here some time in the future.
%
%
%Let's try
%%
%\begin{align}
%\bfJ_r(\bftheta) 
%&= \dpar{\Log(\bfR\tr\{\bftheta\}\,\bfR\{\bftheta+\delta\bftheta\})}{\delta\bftheta} \\
%&= \lim_{\delta\bftheta\to0} \frac{\Log(\bfR\tr\{\bftheta\}\,\bfR\{\bftheta+\delta\bftheta\})}{\delta\bftheta} \\
%&= \lim_{\delta\bftheta\to0} \frac{\Log(\bfR\tr\{\bftheta\}\,\bfR\{\bftheta+\delta\bftheta\})}{\delta\bftheta} \\
%\end{align}





\subsubsection{在 $SO(3)$ 中的右 Jacobian 矩阵}

让我们考虑 (参见 \figRef{fig:right_jac}) 在 $\sR\in SO(3)$ 中的一个元素和一个旋转向量 $\bth\in\bbR^3$ 致使 $\sR=\Exp(\bth)$。当 $\bth$ 被 $\dth$更改时,元素 $\sR$ 会发生变化。用旋转向量 $\delta\bfphi\in\bbR^3$ 来表示处于 $\sR$ 在 $SO(3)$ 中的切线空间中的变化,我们得到了 (请看图,这里我没有发明任何东西)
%
\begin{align}
\Exp(\bth)\oplus\delta\bfphi = \Exp(\bth+\dth)
\end{align}
%
也可以写成,
%
\begin{align}
\Exp(\bth)\circ\Exp(\delta\bfphi) &= \Exp(\bth+\dth)
~,
\end{align}
%
甚至
%
\begin{align}
\delta\bfphi &= \Log\Big(\Exp(\bth)\inv\circ\Exp(\bth+\dth)\Big) = \Exp(\bth+\dth) \ominus \Exp(\bth)
~.
\end{align}

在极限中,$\delta\bfphi$ 作为 $\dth$ 的函数的变化定义了 Jacobian 矩阵
%
\begin{align}
\dpar{\delta\bfphi}{\dth} 
&= \lim_{\dth\to0}\frac{\delta\bfphi}{\dth} 
= \lim_{\dth\to0}\frac{\Exp(\bth+\dth) \ominus \Exp(\bth)}{\dth} 
%&= \lim_{\dth\to0}\frac{\Exp(\bth)\inv\Exp(\bth+\dth)}{\dth} 
~,
\end{align}
%
其表达式是方程 \eqRef{equ:dif_RtoSO3} 的一个特殊情况,即它是函数 $f(\bth)=\Exp(\bth)$,从 $\bbR^3$ 到 $SO(3)$ 的导数。
%
\begin{figure}[tbp]
\begin{center}
\includegraphics{figures/right_jac}
\caption{右 Jacobian 矩阵 $\bfJ_r=\dparil{\delta\bfphi}{\delta\bftheta}$ 将参数 $\bftheta$ 周围的变量 $\delta\bftheta$ 映射进入变量 $\delta\bfphi$ 在向量空间与流形相切的点 $\Exp{\bftheta}$ 处。}
\label{fig:right_jac}
\end{center}
\end{figure}
%
这个 Jacobian 矩阵称为 $SO(3)$ 的右雅可比矩阵(right Jacobian),并被定义为,
%We define the right Jacobian of $SO(3)$ as, 
%
\begin{align}
\bfJ_r(\bth) &\te \dpar{\Exp(\bth)}{\bth} 
~.
\end{align}
%
它的表达式独立于所使用的参数化,尽管它确实可以特别针对每个参数化来表示。使用方程 \eqRef{equ:dif_RtoSO3} 我们有,
%
\begin{align}
\bfJ_r(\bth) &= \lim_{\dth\to0}\frac{\Exp(\bth+\dth)\ominus\Exp(\bth)}{\dth} \\
 &= \lim_{\dth\to0}\frac{\Log(\Exp(\bth)\tr\Exp(\bth+\dth))}{\dth} && \textrm{if using $\bfR$} \\
 &= \lim_{\dth\to0}\frac{\Log(\Exp(\bth)^*\ot\Exp(\bth+\dth))}{\dth} && \textrm{if using $\bfq$} 
 ~.
\end{align}
%

右 Jacobian 矩阵及其逆矩阵可以进行闭式计算 \citep[page 40]{CHIRIKJIAN-12},
%
\begin{align}
\bfJ_r(\bth) &= \bfI - \frac{1-\cos\nth}{\nth^2}\hatx{\bth} + \frac{\nth-\sin\nth}{\nth^3}\hatx{\bth}^2 \\
\bfJ_r\inv(\bth) &= \bfI + \frac12\hatx{\bth} + \left(\frac1{\nth^2} - \frac{1+\cos\nth}{2\nth\sin\nth}\right)\hatx{\bth}^2
\end{align}



$SO(3)$ 的右 Jacobian 具有以下性质,对于任何 $\bth$ 和小的 $\dth$,
%
\begin{align}
\Exp(\bth+\dth) &\approx \Exp(\bth)\Exp(\bfJ_r(\bth)\dth) \label{equ:Jr1} \\
\Exp(\bth)\Exp(\dth) &\approx \Exp(\bth+\bfJ_r\inv(\bth)\,\dth) \\
\Log(\Exp(\bth)\Exp(\dth)) &\approx \bth+\bfJ_r\inv(\bth)\,\dth 
\end{align}


%-------------------------------------------------------------
\subsubsection{相对于旋转向量的 Jacobian 矩阵}

向量 $\bfa'=\bfR\{\bftheta\}\,\bfa$ 的旋转相对于旋转向量 $\bth$ 是一个从 $\bbR^3$ 到 $\bbR^3$ 的函数。它相对于旋转向量 $\bftheta$ 的导数使用方程 \eqRef{equ:derivative_vector} 并根据先前的结果扩展而来,使用方程 \eqRef{equ:Jr1},
%
\begin{align*} 
\dpar{(\bfq\ot\bfa\ot\bfq^*)}{\delta\bftheta} 
= \dpar{(\bfR\,\bfa)}{\delta\bftheta} 
&= \lim_{\delta\bftheta\to 0} \frac{\bfR\{\bftheta+\delta\bftheta\}\,\bfa-\bfR\{\bftheta\}\,\bfa}{\delta\bftheta} &&\gets\eqRef{equ:derivative_vector}\\
&= \lim_{\delta\bftheta\to 0} \frac{(\bfR\{\bftheta\}\Exp(\bfJ_r(\bftheta)\,\delta\bftheta)-\bfR\{\bftheta\})\bfa}{\delta\bftheta} && \gets\eqRef{equ:Jr1} \\
&= \lim_{\delta\bftheta\to 0} \frac{(\bfR\{\bftheta\}(\bfI+\hatx{\bfJ_r(\bftheta)\,\delta\bftheta})-\bfR\{\bftheta\})\bfa}{\delta\bftheta} \\
&= \lim_{\delta\bftheta\to 0} \frac{\bfR\{\bftheta\}\hatx{\bfJ_r(\bftheta)\,\delta\bftheta}\bfa}{\delta\bftheta} \\
&= \lim_{\delta\bftheta\to 0} -\frac{\bfR\{\bftheta\}\hatx{\bfa}\bfJ_r(\bftheta)\,\delta\bftheta}{\delta\bftheta} \\
&= -\bfR\{\bftheta\}\hatx{\bfa}\bfJ_r (\bftheta)
~,
\end{align*}
%
其中 $\bfR\{\bftheta\}\te\Exp(\bftheta)$。总结一下,
%
\begin{align} \label{equ:drot_da}
\eqbox{\dpar{(\bfq\ot\bfa\ot\bfq^*)}{\delta\bftheta} 
= \dpar{(\bfR\,\bfa)}{\delta\bftheta} 
= -\bfR\{\bftheta\}\hatx{\bfa}\bfJ_r(\bftheta) 
}~.
\end{align}


%-------------------------------------------------------------
\subsubsection{旋转组合的 Jacobians 矩阵}

考虑到 $SO(3)$ 的组合 $\sP=\sQ\circ\sR$,它可以以四元数或矩阵的形式实现,
%
\begin{align*}
\bfp &= \bfq_\theta\ot\bfr_\phi & \bfP &= \bfQ_\theta\,\bfR_\phi
\end{align*}
%
其中下标表示切线空间中向量扰动的名字。 
这些是 从 $SO(3)$ 到 $SO(3)$的函数,因此我们使用方程 \eqRef{equ:derivative_SO3} 来写出导数,
%
\begin{align*}
\dpar{\sQ\circ\sR}{\sQ} 
= \dpar{\bfq_\theta\ot\bfr_\phi}{\bth} 
= \dpar{\bfQ_\theta \bfR_\phi}{\bth} 
&= \lim_{\delta\bftheta\to 0}\frac{((\bfQ_\theta\oplus\dth)\bfR_\phi)\ominus(\bfQ_\theta\bfR_\phi)}{\dth} \\
&= \lim_{\delta\bftheta\to 0}\frac{\Log[(\bfQ_\theta\bfR_\phi)\tr(\bfQ_\theta\Exp(\dth)\bfR_\phi)]}{\dth} \\
&= \lim_{\delta\bftheta\to 0}\frac{\Log[\bfR_\phi\tr\Exp(\dth)\bfR_\phi]}{\dth} \\
&= \lim_{\delta\bftheta\to 0}\frac{\Log[\Exp(\bfR_\phi\tr\dth)]}{\dth} \\
&= \lim_{\delta\bftheta\to 0}\frac{\bfR_\phi\tr\dth}{\dth}  = \bfR_\phi\tr 
\end{align*}
%
\begin{align*}
\dpar{\sQ\circ\sR}{\sR} 
= \dpar{\bfq_\theta\ot\bfr_\phi}{\bfphi} 
= \dpar{\bfQ_\theta \bfR_\phi}{\bfphi} 
&= \lim_{\delta\bfphi\to 0}\frac{(\bfQ_\theta(\bfR_\phi\oplus\delta\bfphi))\ominus(\bfQ_\theta\bfR_\phi)}{\delta\bfphi} \\
&= \lim_{\delta\bfphi\to 0}\frac{\Log[(\bfQ_\theta\bfR_\phi)\tr(\bfQ_\theta\bfR_\phi\Exp(\delta\bfphi))]}{\delta\bfphi} \\
&= \lim_{\delta\bfphi\to 0}\frac{\Log[\Exp(\delta\bfphi)]}{\delta\bfphi} \\
&= \lim_{\delta\bfphi\to 0}\frac{\delta\bfphi}{\delta\bfphi}  = \bfI 
\end{align*}

%=============================================================
\subsection{扰动、不确定性、噪声}

%-------------------------------------------------------------
\subsubsection{局部扰动}

扰动方向 $\tilde{\bfq}$ 可表示为未扰动方向 $\bfq$ 与一个小的局部扰动 ${\Delta\bfq_\cL}$ 的组合。 
因为 Hamilton 约定,该局部扰动出现在组合乘积的右手侧(\emph{right hand side}) --- 我们也给出了用于比较的等效矩阵,
%
\begin{align}
\tilde{\bfq} &= \bfq\ot{\Delta\bfq_\cL}
~, &
\tilde\bfR &= \bfR\,\Delta\bfR_\cL
~.
\end{align}%
%
使用指数映射,这些局部扰动 $\Delta\bfq_\cL$ (或 $\Delta\bfR_\cL$) 很容易从切线空间中定义的等效向量形式 $\Delta\bfphi_\cL=\bfu\Delta\phi_\cL$ 获得。这给了
%
\begin{align}
\tilde\bfq_\cL &= \bfq_\cL\ot\Exp(\Delta\bfphi_\cL)
~,& 
\tilde\bfR_\cL &= \bfR_\cL\tdot\Exp(\Delta\bfphi_\cL)
\end{align}
%
导致局部扰动的表达式 
%
\begin{align}
\Delta\bfphi_\cL = \Log(\bfq_\cL^*\ot\tilde\bfq_\cL) = \Log(\bfR_\cL\tr\tdot\tilde\bfR_\cL)
\end{align}
 

如果扰动角度 $\Delta\phi_\cL$ 较小,则四元数和旋转矩阵形式的扰动可以由方程 \eqRef{equ:vectoquat} 和 \eqRef{equ:vectomat} 的泰勒展开近似到线性项,
%
\begin{align}
{\Delta\bfq_\cL}& \approx \begin{bmatrix}
1\\\frac{1}{2}\Delta\bfphi_\cL
\end{bmatrix}
~,
&
\Delta\bfR_\cL& \approx
\bfI+\hatx{\Delta\bfphi_\cL}
~.
\end{align}%
%
因此,扰动可以被指定为在局部向量空间 $\Delta\bfphi_\cL$ 相切于在与实际方向上的 $SO(3)$ 的流形。这是很方便的,例如,在这个向量空间中表示这些扰动的协方差矩阵,也就是说,使用规则的 $3\times 3$ 协方差矩阵。

\subsubsection{全局扰动}

考虑全局定义的扰动是可能的,而且确实有趣,对于相关的导数也是如此。 
全局扰动出现在组合乘积的左手侧(\emph{left hand side}),即,
%
%\begin{align}
%\tilde\bfq &= \Delta\bfq_\cG\otimes\bfq~, 
%&
%\tilde\bfR &= \Delta\bfR_\cG\,\bfR~.
%\end{align}
%
\begin{align}
\tilde\bfq_\cG &= \Exp(\Delta\bfphi_\cG)\ot\bfq_\cG
~,& 
\tilde\bfR_\cG &= \Exp(\Delta\bfphi_\cG)\tdot\bfR_\cG
\end{align}
%
导致全局扰动的表达式 
%
\begin{align}
\Delta\bfphi_\cG = \Log(\tilde\bfq_\cG\ot\bfq_\cG^*) = \Log(\tilde\bfR_\cG\tdot\bfR_\cG\tr)
\end{align}


同样,这些扰动可以被指定为在向量空间 $\Delta\bfphi_\cG$ 相切于在原点处的 $SO(3)$ 的流形。

\subsection{时间导数}

在向量空间中表示局部扰动,我们可以很容易地得到时间导数的表达式。 
只需将 $\bfq=\bfq(t)$ 作为初始状态,将 $\tilde{\bfq}=\bfq(t+\Dt)$ 作为扰动状态,并将导数的定义应用
%
\begin{align}
\dif{\bfq(t)}{t} \triangleq \lim_{\Dt\to0} \frac{\bfq(t+\Dt)-\bfq(t)}{\Dt}~, \label{equ:derivative}
\end{align}%
%
对上面,还有
%
\begin{align}
\bfomega_\cL(t) \triangleq \dif{\bfphi_\cL(t)}{t} \triangleq \lim_{\Dt\to0} \frac{\Delta\bfphi_\cL}{\Dt}~,
\end{align}%
%
其中,做为 $\Delta\bfphi_\cL$ 它是一个局部角扰动,对应于由 $\bfq$ 定义的局部坐标系中的角速率向量。

四元数时间导数的扩展如下 (旋转矩阵将使用类似的推理)
%
%
\begin{align}
\dot{\bfq} &\triangleq \lim_{\Dt\to0} \frac{\bfq(t+\Dt)-\bfq(t)}{\Dt} \nonumber\\
&= \lim_{\Dt\to0} \frac{\bfq\ot{\Delta\bfq_\cL}-\bfq}{\Dt} \nonumber\\
&= \lim_{\Dt\to0} \frac{\bfq\ot\left(\begin{bmatrix}
1\\\Delta\bfphi_\cL/2
\end{bmatrix}-\begin{bmatrix}
1\\\bf0
\end{bmatrix}\right)}{\Dt} \nonumber\\ 
&= \lim_{\Dt\to0} \frac{\bfq\ot\begin{bmatrix}
0\\\Delta\bfphi_\cL/2
\end{bmatrix}}{\Dt} \nonumber\\ 
&= \frac12\,\bfq\ot\begin{bmatrix}
0\\\bfomega_\cL
\end{bmatrix}
~. \label{equ:lastquatdev}
\end{align}%
%
定义
%
\begin{align}
\bfOmega(\bfomega) 
\triangleq \QR{\bfomega} 
= \begin{bmatrix}
0 & -\bfomega\tr \\
\bfomega & -\hatx{\bfomega}
\end{bmatrix} = \begin{bmatrix}
0        & -\omega_x & -\omega_y & -\omega_z \\
\omega_x & 0         &  \omega_z & -\omega_y \\
\omega_y & -\omega_z & 0         & \omega_x \\
\omega_z &  \omega_y & -\omega_x & 0
\end{bmatrix} ~, \label{equ:Omega}
\end{align}%
%
我们从方程 \eqRef{equ:lastquatdev} 和 \eqRef{equ:quatMatProd} 得到了 (我们也给出了它的等价矩阵)
%
\begin{empheq}[box=\widefbox]{align}
\label{equ:qdotLocal}
\dot{\bfq} &= \frac{1}{2}\bfOmega(\bfomega_\cL)\,\bfq = \frac{1}{2}\bfq\ot\bfomega_\cL
~,
&
\dot\bfR &= \bfR\hatx{\bfomega_\cL}
~.
\end{empheq}

这些表达式当然与方程 \eqRef{equ:qdot} 和 \eqRef{equ:Rdot}相同,是在旋转群 $SO(3)$ 的框架下发展起来的。 
然而,在这里,有趣的是,我们能够清楚地将角速率 $\bfomega_\cL$ 参考到一个特定的参考系,在这种情况下,该参考系是由方向 $\bfq$ 或 $\bfR$定义的局部坐标系。
这现在是可能的,因为我们已经给了算子 $\bfq$ 和 $\bfR$ 一个精确的几何意义。
从这个观点来看,方程 \eqRef{equ:qdotLocal} 表示当角速率在这个坐标系中的局部表示时,参考坐标系的方向的演变。

%-------------------------------------------------------------

与整体扰动相关的时间导数来自类似于方程 \eqRef{equ:lastquatdev}的扩展,其结果是
%
\begin{empheq}[box=\widefbox]{align}
\label{equ:qdotGlobal}
\dot\bfq &= \frac12\,\bfomega_\cG\ot\bfq~,
&
\dot\bfR &= \hatx{\bfomega_\cG}\bfR~,
\end{empheq}
%
其中
%
\begin{align}
\bfomega_\cG(t)\triangleq\dif{\bfphi_\cG(t)}{t}
\end{align}
%
是在全局坐标系中表示的角速率向量。
方程~\eqRef{equ:qdotGlobal}表示当角速率在全局参考坐标系中表示时,参考坐标系的方向的演变。
%-------------------------------------------------------------
\subsubsection{全局到局部(Global-to-local)的关系}

从上一段中,值得注意的是,局部角速率和全局角速率之间存在以下关系,
%
\begin{align}
\frac12\,\bfomega_\cG\ot\bfq = \dot\bfq = \frac12\,\bfq\ot\bfomega_\cL ~.
\end{align}
%
然后,后乘共轭四元数,我们有
%
\begin{align}
\bfomega_\cG = \bfq\ot\bfomega_\cL\ot\bfq^* = \bfR\,\bfomega_\cL~.
\end{align}
%
同样地,考虑到 $\Delta\bfphi_R \approx \bfomega\Dt$ 对于较小的 $\Dt$,我们有这些
%
\begin{align}
\Delta\bfphi_\cG = \bfq\ot\Delta\bfphi_\cL\ot\bfq^* = \bfR\,\Delta\bfphi_\cL~.
\end{align}
%
也就是说我们可以使用四元数或旋转矩阵,通过坐标系变换来变换角速率向量 $\bfomega$ 和小角度扰动 $\Delta\bfphi$ ,就好像它们是正则向量一样。同样可以通过姿态 $\bfomega=\bfu\omega$,或 $\Delta\bfphi=\bfu\Delta\phi$来理解,并且注意到旋转轴向量 $\bfu$ 正常地变换,用
% 
\begin{align}
\bfu_\cG=\bfq\ot\bfu_\cL\ot\bfq^*=\bfR\,\bfu_\cL ~.
\end{align}



%-------------------------------------------------------------
\subsubsection{四元数乘积的时间导数}

我们用正则公式求出乘积的导数,
%
\begin{align}
\dot{({\bfq_1\ot\bfq_2})} &= \dot{\bfq_1}\ot{\bfq_2} + \bfq_1\ot\dot{{\bfq_2}}~, 
&
\dot{(\bfR_1\bfR_2)} &= \dot\bfR_1\bfR_2+\bfR_1\dot\bfR_2~,
\end{align}%
%
但是要注意,由于乘积是非交换的,我们需要严格遵守操作数的顺序。
这意味着 $\dot{(\bfq^2)}\neq2\,\bfq\ot\dot\bfq
$~,就像标量情况一样,但是
%
\begin{align}
\dot{(\bfq^2)} = \dot\bfq\ot\bfq + \bfq\ot\dot\bfq 
%= \frac12\,\bfq\ot(\bfomega\ot\bfq+\bfq\ot\bfomega)
~.
\end{align}


%-------------------------------------------------------------
\subsubsection{其它有用的导数表达式}

我们可以导出局部旋转率的表达式
%
\begin{align}
\bfomega_\cL &= 2\,\bfq^*\ot\dot{\bfq}~, 
&
\hatx{\bfomega_\cL} &= \bfR\tr\,\dot\bfR~.
\end{align}%
%
以及全局自转率,
%
\begin{align}
\bfomega_\cG &= 2\,\dot{\bfq}\ot\bfq^*~, 
&
\hatx{\bfomega_\cG} &= \dot\bfR\,\bfR\tr~.
\end{align}%




%=============================================================
\subsection{转速的时间积分}

四元数形式的累积旋转是通过积分适合于旋转速率定义的微分方程来完成的,即,方程 \eqRef{equ:qdotLocal} 用于局部旋转速率定义,而方程 \eqRef{equ:qdotGlobal} 用于全局旋转速率定义。 
在我们感兴趣的情况下,角速率由局部传感器测量,从而在离散时间 $t_n=n\Dt$ 里提供局部测量 $\bfomega(t_n)$ 。
我们仅在此集中讨论这种情况,对于这种情况,我们再现微分方程 \eqRef{equ:qdotLocal},
%
\begin{align}
\dot\bfq(t) 
= \frac12\bfq(t)\ot\bfomega(t) 
~.
\label{equ:intLocal}
\end{align}

我们扩展了零阶和一阶积分方法,参见图(\ref{fig:quatInt} 和 \ref{fig:integrate}),都是在时间 $t=t_n$ 左右,基于 $\bfq(t_n+\Dt)$ 的泰勒级数。 
我们注意到 $\bfq\triangleq\bfq(t)$ 和 $\bfq_n\triangleq\bfq(t_n)$,并且对 $\bfomega$ 也一样。 
%
泰勒级数显示, 
%
\begin{align}
\q_{n+1} = \q_n + \dq_n\Dt + \frac1{2!}\ddq_n\Dt^2 + \frac1{3!}\dddq_n\Dt^3 + \frac1{4!}\ddddq_n\Dt^4 + \cdots~.
\label{equ:qnTaylor}
\end{align}
%
通过反复应用四元数导数方程 \eqRef{equ:intLocal},取 $\ddot\bfomega=0$,上述 $\bfq_n$ 的连续导数很容易得到。我们得到
%
\begin{subequations}
%
\begin{align}
\dq_n 
\e \frac12\q_n\w_n \\
\ddq_n 
\e \frac1{2^2}\q_n\w_n^2+\frac12\q_n\dw \\
\dddq_n 
\e \frac1{2^3}\q_n\w_n^3 + \frac14\q_n\dw\w_n + \frac12\q\w_n\dw \\
\q_n^{(i\,\ge\,4)} 
\e \frac1{2^i}\q_n\w_n^i + \cdots~,
\end{align}%
\label{equ:qnDerivatives}%
\end{subequations}%
%
这里我们为了节省符号而省略了 $\ot$ 的符号,也就是说,所有的 $\w$ 的乘积和幂必须用四元数乘积来解释。

\begin{figure}[tb]
\centering
\includegraphics{figures/integral}
\caption{积分的角速度近似:红色:真实速度。蓝色:零阶近似 (下到上:前、中、后). 绿色:一阶近似。}
\label{fig:quatInt}
\end{figure}

\begin{figure}[tb]
\begin{center}
\includegraphics{figures/integrate}
\caption{两个连续时间步 (灰色和黑色箭头集合)的积分方案,其中共享相同时间戳的变量被组织成列。左:前向积分。中心:中间和一阶积分。右图:后向积分。}
\label{fig:integrate}
\end{center}
\end{figure}


%-------------------------------------------------------------
\subsubsection{零阶积分}


\paragraph{前向积分}
在角速率 $\bfomega_n$ 在周期 $[t_n,t_{n+1}]$ 内保持不变的情况下, %with $\Dt=t_{n+1}-t_n$~, 
我们有 $\dot\bfomega=0$ 并且方程 \eqRef{equ:qnTaylor} 降阶为,
%
\begin{align}
\q_{n+1} = \q_n\ot\left(
1 
+ \frac12\bfomega_n\Dt 
+ \frac1{2!}\Big(\frac12\bfomega_n\Dt\Big)^2 
+ \frac1{3!}\Big(\frac12\bfomega_n\Dt\Big)^3 
%+ \frac1{4!}\Big(\frac12\bfomega_n\Dt\Big)^4 
+  \cdots \right)~,
\end{align}
%
在这里我们确定指数 $e^{\bfomega_n\Dt/2}$ 的泰勒级数方程 \eqRef{equ:pureQuatExpSeries} 。 
从
方程 \eqRef{equ:vectoquat} 可知,
这个指数对应于四元数,代表增量旋转 $\Delta\theta=\bfomega_n\Dt$ ,
%
\begin{align*}
e^{\bfomega\Dt/2} = \Exp(\bfomega\Dt) = \bfq\{\bfomega\Dt\} = \begin{bmatrix}
\cos(\norm{\bfomega}\Dt/2) \\
\frac{\bfomega}{\norm{\bfomega}}\sin(\norm{\bfomega}\Dt/2)
\end{bmatrix}~,
\end{align*}
%
因此,
%
\begin{align}
\eqbox{
\bfq_{n+1} = \bfq_n\ot\bfq\{\bfomega_n\Dt\} 
} 
~.
\label{equ:intZeroth}
\end{align}


\paragraph{后向积分}
我们还可以认为,周期 $\Dt$ 上的恒定速度对应于 $\bfomega_{n+1}$,周期结束时测量的速度。这可以用类似的方式扩展,在 $t_{n+1}$ 附近用 $\bfq_n$ 的泰勒展开,导致
%
\begin{align}
{\bfq_{n+1} \approx \bfq_n\ot\bfq\{\bfomega_{n+1}\Dt\}} ~.
\end{align}

这里我们要指出的是,当到达的运动测量要实时处理时,这是典型的积分方法,因为积分视界对应于最后的测量 (在这种情况下, $t_{n+1}$,参见 \figRef{fig:integrate})。为了使这一点更加突出,我们可以重新标记时间索引,使用 $\{n-1,n\}$ 而不是 $\{n,n+1\}$,然后写,
%
\begin{align}
\eqbox{\bfq_n = \bfq_{n-1}\ot\bfq\{\bfomega_n\Dt\}} ~.
\end{align}

\paragraph{中间积分}

类似地,如果速度被认为在周期 $\Dt$ 的中间速率恒定(这不是周期中点的速度所必需的),
%
\begin{align}
\aw = \frac{\w_{n+1}+\w_n}{2}~,
\label{equ:wbar}
\end{align}
%
我们有,
%
\begin{align}
\eqbox{
\q_{n+1} = \bfq_n\ot\bfq\{\aw\Dt\} 
}
~.
\label{equ:intFirstC}
\end{align}

%-------------------------------------------------------------
\subsubsection{一阶积分}



角速率 $\w(t)$ 现在与时间成线性关系。它的一阶导数是常数,所有的高阶导数都是零,
%
%
\begin{align}
\dw \e \frac{\w_{n+1}-\w_n}{\Dt} \label{equ:wdot} \\
\ddot\w = \dddot\w = \cdots \e 0 \label{equ:wddot}
~.
\end{align}%
%
我们可以依据 $\w_n$ 和 $\dw$ 写出中间速率 $\aw$ ,
%
\begin{align}
\aw = \w_n+\frac12\dw\Dt~,
\end{align}
%
并依据更实用的 $\aw$ 和 $\dw$,导出出现在四元数导数方程 \eqRef{equ:qnDerivatives} 中的 $\w_n$ 的幂的表达式。
%
\begin{subequations}
%
\begin{align}
\w_n \e \aw-\frac12\dw\Dt \label{equ:wone}\\
\w_n^2 \e \aw^2 - \frac12\aw\dw\Dt - \frac12\dw\aw\Dt + \frac14\dw^2\Dt^2 \\
\w_n^3 \e \aw^3 - \frac32\aw^2\dw\Dt + \frac34\aw\dw^2\Dt^2 + \frac18\dw^3\Dt^3 \\
\w_n^4 \e \aw^4 + \cdots \label{equ:wthree}
~.
\end{align}%
\end{subequations}
%
将它们注入四元数导数,代入泰勒级数方程 \eqRef{equ:qnTaylor},经过适当的重排序后,我们有,
%
\begin{subequations}
%
\begin{align}
\q_{n+1} 
\e
\q\left(1+\frac12\aw\Dt+\frac1{2!}\left(\frac12\aw\Dt\right)^2+\frac1{3!}\left(\frac12\aw\Dt\right)^3+\cdots\right) \label{equ:exp} \\
&+\: \q\left(-\frac14\dw + \frac14\dw \right)\Dt^2 \label{equ:vanish}\\
&+\: \q\left(-\frac1{16}\aw\dw - \frac1{16}\dw\aw + \frac1{24}\dw\aw + \frac1{12}\aw\dw\right)\Dt^3 \label{equ:bracket} \\
&+\: \q\,\bigg(\:\cdots\:\bigg)\Dt^4 \:+\: \cdots 
\label{equ:neglected}
~,
\end{align}%
\end{subequations}
%
其中在方程 \eqRef{equ:exp} 中,我们识别到指数级数 $e^{\aw\Dt/2}= \q\{\aw\Dt\}$ ,方程 \eqRef{equ:vanish} 消失,并且方程 \eqRef{equ:neglected} 表示我们将忽略高幂次项。 
%
这在简化后产生 (我们现在恢复正常的 $\ot$ 符号),
%
\begin{align}
\q_{n+1} = \q_n\ot\q\{\aw\Dt\} + \frac{\Dt^3}{48}\q_n\ot(\aw\ot\dw - \dw\ot\aw) + \cdots ~.
\end{align}
%
代入 $\dw$ 和 $\aw$ ,根据它们的定义方程 \eqRef{equ:wdot} 和 \eqRef{equ:wbar} ,我们得到,
%
\begin{align}
\q_{n+1} = \q_n\ot\q\{\aw\Dt\} + \frac{\Dt^2}{48}\q_n\ot(\w_n\ot\w_{n+1} - \w_{n+1}\ot\w_n) + \cdots ~,
\label{equ:intFirstA}
\end{align}
%
这是一个等价于 \citep{TRAWNY-05-QUAT}的结果,但使用 Hamilton 约定,并且四元数积形式而不是矩阵积形式。 
最后,由于 $\bfa_v\ot\bfb_v-\bfb_v\ot\bfa_v=2\,\bfa_v\tcross\bfb_v$,参见方程 \eqRef{equ:quatCommutatorPure},我们有另一种形式,
%
%\begin{align}
%\eqbox{\q_{n+1} \approx \bfq_n\ot\bfq\{\aw\Dt\} + \frac{\Dt^2}{24}\,\bfq_n\ot\begin{bmatrix}
%0\\\w_n\tcross\w_{n+1}
%\end{bmatrix}} 
%~,
%\label{equ:intFirstB}
%\end{align}
%%
%or even
%
\begin{align} \label{equ:intFirstB}
\eqbox{\q_{n+1} \approx \bfq_n\ot\left(\bfq\{\aw\Dt\} + \frac{\Dt^2}{24}\,\begin{bmatrix}
0\\\w_n\tcross\w_{n+1}
\end{bmatrix}\right)
}~.
\end{align}
%
在这个表达式中,加法的第一项是零阶中间积分方程 \eqRef{equ:intFirstC}。
第二项是一个二阶校正,它会消失,当 $\w_n$ 和 $\w_{n+1}$ 共线时,%
\footnote{从方程 \eqRef{equ:intFirstA} 还可以注意到,如果四元数乘积是可交换的,此项将总是(\emph{always})消失,而不会存在。}
即当旋转轴从 $t_n$ 到 $t_{n+1}$ 没有变化时 。


\paragraph{固定旋转轴情况}

让我们写出 $\bfomega(t)=\bfu(t)\,\omega(t)$ 并称 $\bfu$ 为旋转轴。
在恒定旋转轴 $\bfu(t) = \bfu$情况下,我们有 $\w_n\tcross\w_{n+1}=0$ 并因此,
%
\begin{align}\label{equ:first_order_constant_axis}
\q_{n+1} = \bfq_n\ot\bfq\{\bfu\,\ol\omega\,\Dt\}~.
\end{align}
%
事实上,这个结果对于不限于 $\bfomega(t)$ 一阶导数的情况是有趣的。
实际上,如果旋转轴是常数,旋转进入四元数的无穷小贡献可交换,即, 
%
$$\exp (\bfu\,\omega_1\,\delta t_1)\exp (\bfu\,\omega_2\,\delta t_2)=\exp (\bfu\,\omega_2\,\delta t_2)\exp (\bfu\,\omega_1\,\delta t_1)=\exp (\bfu\,(\omega_1\delta t_1 + \omega_2\delta t_2))~,$$ 
%
并且因此我们有特征值,
%
\begin{subequations}
\begin{align}
\q_{n+1} 
  &= \bfq_n\ot \exp\left(\frac{\bfu}2\,\int_{t_n}^{t_{n+1}}\omega(t)\,\dt\right) \\
  &= \bfq_n\ot \exp(\bfu\,\Delta\theta_n/2) \\
  &= \bfq_n\ot \bfq\{\bfu\,\Delta\theta_n\}~.
\end{align}
\end{subequations}
%
其中 $\Delta\theta_n = \int_{t_n}^{t_{n+1}} \omega(t)dt \in \bbR$ 为区间 $[t_n,t_{n+1}]$ 内旋转的总角度。


\paragraph{变化旋转轴情况}

显然,在方程 \eqRef{equ:intFirstB} 中加法的第二项通过 $\w_n\tcross\w_{n+1}\neq0$ 捕获变化的旋转轴对积分方向的影响。
对于它的实际应用,我们注意到给定通常的 IMU 采样时间 $\Dt\le0.01s$,并且由于惯性,通常的 $\w_n$ 和 $\w_{n+1}$ 为近似共线性,这个二阶项的值为 $10^{-6}\nw^2$ 阶,或容易变得更小。 
具有更高幂次的 $\w\Dt$ 的项甚至更小,并且被忽略。 


还请注意,虽然所有零阶积分器通过构造产生单位四元数 (因为它们是作为两个单位四元数的乘积计算的),但由于方程 \eqRef{equ:intFirstB} 中的加法,这不是一阶积分器的情况。因此,在使用一阶积分器时,即使所述的求和项很小,用户也应注意检查四元数范数随时间的演化,并最终在需要时使用形式 $\bfq\gets\bfq/\norm{\bfq}$的四元数更新来重新规范化四元数。只有当定轴假设成立时,则方程 \eqRef{equ:first_order_constant_axis} 也成立,而这种规范化不再是必要的。 


