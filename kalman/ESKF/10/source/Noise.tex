% !TEX root = kinematics.tex

%%%%%%%%%%%%%%%%%%%%%%%%%%%%%%%%%%%%%%%%%%%%%%%%%%%%%%%%%%%%%%
\section{随机噪声和扰动的积分}
\label{sec:IntNoise}

我们现在的目标是给出适当的方法来积分动力学系统中的随机变量。 
当然,我们不能对未知的随机值进行积分,但是为了不确定性的传播,我们可以对它们的方差和协方差进行积分。 
这是必要的,为了在具有连续性质(并且在连续时间中指定),但是以离散方式估计的系统中的估计器建立协方差矩阵。

考虑连续时间动力学系统,
%
\begin{equation}
\dot\bfx = f(\bfx,\bfu,\bfw)~,
\label{equ:contSysWithNoise}
\end{equation}
%
其中 $\bfx$ 是状态向量, $\bfu$ 是包含噪声 $\tilde\bfu$ 的控制信号的向量,因此控制测量为 $\bfu_m=\bfu+\tilde\bfu$,并且 $\bfw$ 是随机扰动的向量。 
噪声和扰动都是假设的白高斯过程,由,
%
\begin{equation}
\tilde\bfu \sim \cN\{0,\bfU^c\} \quad , \quad \bfw^c\sim\cN\{0,\bfW^c\}~,
\end{equation}
%
其中超级索引 $\bullet^c$ 表示连续时间不确定性规范,我们希望对其进行积分。

在控制信号中的噪声水平的性质,$\tilde\bfu$,和随机扰动,$\bfw$,之间存在着重要的区别:
%
\begin{itemize}
\item 
在离散化时,控制信号在时间瞬间 $n\Dt$ 进行采样,具有 $\bfu_{m,n}\triangleq\bfu_m(n\Dt)=\bfu(n\Dt)+\tilde\bfu(n\Dt)$ 。
很明显,测量部分在积分区间内被视为常数,即, $\bfu_m(t)=\bfu_{m,n}$,因此采样时间 $n\Dt$ 处的噪声水平也保持不变,
%
\begin{equation}
\tilde\bfu(t)=\tilde\bfu(n\Dt) = \tilde\bfu_n, \quad n\Dt<t<(n+1)\Dt~. \label{equ:uint}
\end{equation}
%
\item  
扰动 $\bfw$ 从未采样。 

\end{itemize}
%


因此,这两个随机过程在 $\Dt$ 上的积分是不同的。 
让我们检查一下。 

\bigskip
连续时间误差状态动力学方程 \eqRef{equ:contSysWithNoise} 可以线性化为
%
\begin{equation}
\dot{\delta\bfx} = \bfA\delta\bfx + \bfB\tilde\bfu + \bfC \bfw~, \label{equ:contTime}
\end{equation}
%
其中 
%
\begin{equation}
\bfA\triangleq\pjac{f}{\delta\bfx}{\bfx,\bfu_m}\quad,\quad
\bfB\triangleq\pjac{f}{\tilde\bfu}{\bfx,\bfu_m}\quad,\quad
\bfC\triangleq\pjac{f}{\bfw}{\bfx,\bfu_m}~,
\end{equation}
%
并在取样周期 $\Dt$ 内进行积分,得出,
%
\begin{align}
\delta\bfx_{n+1} &= \delta\bfx_n 
+ \int_{n\Dt}^{(n+1)\Dt} \left(
\bfA\delta\bfx(\tau) + \bfB\tilde\bfu(\tau) + \bfC\bfw^c(\tau)
\right) d\tau \\
 &= \delta\bfx_n 
+ \int_{n\Dt}^{(n+1)\Dt} \bfA\delta\bfx(\tau)   d\tau
+ \int_{n\Dt}^{(n+1)\Dt} \bfB\tilde\bfu(\tau)   d\tau 
+ \int_{n\Dt}^{(n+1)\Dt} \bfC\bfw^c(\tau) d\tau 
\end{align}
%
它有三个性质截然不同的本质。 
它们可以积分如下:
%
\begin{enumerate}
\item 从 \appRef{sec:ClosedFormInt} ,我们知道动力学部分是可积分的,并给出了转换矩阵,
%
\begin{equation}
\delta\bfx_n 
+ \int_{n\Dt}^{(n+1)\Dt} \bfA\delta\bfx(\tau)   d\tau = \Phi\tdot\delta\bfx_n
\end{equation}
%
其中 $\Phi=e^{\bfA\Dt}$ 可以以封闭形式计算或以不同精度的近似值计算。

\item 从方程 \eqRef{equ:uint} 我们有
%
\begin{equation}
\int_{n\Dt}^{(n+1)\Dt} \bfB\tilde\bfu(\tau)   d\tau = \bfB\Dt\tilde\bfu_n
\end{equation}
%
这意味着,一旦采样,测量噪声将以确定的方式进行积分,因为其在积分区间内的行为是已知的。

\item 从概率论我们知道,在一个周期 $\Dt$ 内连续的高斯白噪声的积分产生一个离散的高斯白脉冲 $\bfw_n$ ,描述如下
%
\begin{equation}
\bfw_n\triangleq\int_{n\Dt}^{(n+1)\Dt}\bfw(\tau)d\tau 
\quad,\quad
\bfw_n \sim \cN\{0,\bfW\} 
\quad,\quad
 \text{with } \bfW = \bfW^c\Dt
\end{equation}
%
我们发现,与上述测量噪声相反,扰动在积分区间内不具有确定性行为,因此必须随机积分。

\end{enumerate}%
%
因此,离散时间误差状态动力学系统可以写成
%
\begin{equation}
\delta\bfx_{n+1} = \bfF_\bfx\delta\bfx_n + \bfF_\bfu\tilde\bfu_n + \bfF_\bfw \bfw_n \label{equ:discTime}
\end{equation}
%
其中得出转换矩阵、控制矩阵和扰动矩阵
%
\begin{equation}
\bfF_\bfx=\Phi = e^{\bfA\Dt} \quad , \quad \bfF_\bfu=\bfB\Dt \quad , \quad \bfF_\bfw = \bfC \quad , \quad 
\end{equation}
%
%where it is illustrative to appreciate the different contributions of the step time $\Dt$ in the integrated values. The
噪声和扰动水平定义为
%
\begin{equation}
\tilde\bfu_n\sim\cN\{0,\bfU\} \quad , \quad \bfw_n\sim\cN\{0,\bfW\} \label{equ:discMat}
\end{equation}
%
其中 
%
\begin{equation}
\bfU = \bfU^c \quad , \quad \bfW = \bfW^c\Dt~. \label{equ:discCov}
\end{equation}
%
\begin{table*}
\renewcommand{\arraystretch}{1.3}
\caption{积分对系统和协方差矩阵的影响。}
\centering
\vspace{1ex}
\begin{tabular}{|c|c|c|}
\hline
说明 & 连续时间 $t$ & 离散时间 $n\Dt$\\
\hline
\hline
状态 & $\dot\bfx=f^c(\bfx,\bfu,\bfw)$ & $\bfx_{n+1} = f(\bfx_n,\bfu_n,\bfw_n)$ \\
误差状态 & $\dot{\delta\bfx}=\bfA\delta\bfx+\bfB\tilde\bfu+\bfC\bfw$ & $\delta\bfx_{n+1}=\bfF_\bfx\delta\bfx_n+\bfF_\bfu\tilde\bfu_n+\bfF_\bfw\bfw_n$ \\
\hline
系统矩阵 & $\bfA$ & $\bfF_\bfx=\Phi=e^{\bfA\Dt}$ \\
控制矩阵 & $\bfB$ & $\bfF_\bfu=\bfB\Dt$ \\
扰动矩阵 & $\bfC$ & $\bfF_\bfw=\bfC$ \\
\hline
控制协方差 & $\bfU^c$ & $\bfU=\bfU^c$  \\
扰动协方差 & $\bfW^c$ & $\bfW=\bfW^c\Dt$  \\
\hline
\end{tabular}
\label{tab:IntEffects}
\end{table*}%

这些结果汇总在 \tabRef{tab:IntEffects} 中。
EKF 的预测阶段将根据以下公式传播误差状态的均值和协方差矩阵
%
\begin{align}
\hat{\delta\bfx}_{n+1} &= \bfF_\bfx \hat{\delta\bfx}_n \\
\bfP_{n+1} &= \bfF_\bfx\bfP_n\bfF_\bfx\tr + \bfF_\bfu\bfU\bfF_\bfu\tr + \bfF_\bfw\bfW\bfF_\bfw\tr \nonumber\\
&= e^{\bfA\Dt}\bfP_n(e^{\bfA\Dt})\tr + \Dt^2\bfB\bfU^c\bfB\tr + \Dt\bfC\bfW^c\bfC\tr \label{equ:NoisePertCovUpdate}
\end{align}

在这里,观察积分间隔 $\Dt$对协方差更新方程 \eqRef{equ:NoisePertCovUpdate}的三项的不同影响是重要的和说明性的:动力学误差项是指数的,测量误差项是二次的,扰动误差项是线性的。 


%=============================================================
\subsection{噪声和扰动脉冲}
\label{sec:pertImpulses}

一个是经常遇到的 (例如,当重用现有代码或解释其他作者的文档) 时,EKF预测方程的形式比我们在这里使用的更简单,即,
%
\begin{equation}
\bfP_{n+1} = \bfF_\bfx\bfP_n\bfF_\bfx\tr + \bfQ~.
\end{equation}
%
这对应于一般的离散时间动力系统,
%
\begin{equation}
\delta\bfx_{n+1} = \bfF_\bfx\delta\bfx_n+\bfi
\end{equation}
%
其中
%
\begin{equation}
\bfi \sim \cN\{0,\bfQ\}
\end{equation}
%
是在时间 $t_{n+1}$ 瞬间直接添加到状态向量的随机 (白色,高斯) 脉冲的向量。  
矩阵 $\bfQ$ 被简单地认为是脉冲协方差矩阵。 
根据我们所看到的,我们应该计算这个协方差矩阵如下,
%
\begin{equation}
\bfQ = \Dt^2\,\bfB\,\bfU^c\,\bfB\tr + \Dt\,\bfC\,\bfW^c\,\bfC\tr~.
\end{equation}

在脉冲不影响全状态的情况下,通常情况下,矩阵 $\bfQ$ 不是全对角的,并且可能包含大量的零。 
然后可以写出等价的形式
%
\begin{equation}
\delta\bfx_{n+1} = \bfF_\bfx\,\delta\bfx_n+\bfF_\bfi\,\bfi
\end{equation}
%
其中
%
\begin{equation}
\bfi \sim \cN\{0,\bfQ_\bfi\}~,
\end{equation}
%
其中,矩阵 $\bfF_\bfi$ 简单地将每个单独的脉冲映射到它影响到的状态向量的一部分。 
相关的协方差 $\bfQ_\bfi$ 则更小且全对角线。 
请参阅下一节以获取示例。 
在这种情况下,ESKF 时间更新变成
%
\begin{align}
\hat{\delta\bfx}_{n+1} &= \bfF_\bfx\,\hat{\delta\bfx}_n \\
\bfP_{n+1} &= \bfF_\bfx\,\bfP_n\,\bfF_\bfx\tr + \bfF_\bfi\,\bfQ_\bfi\,\bfF_\bfi\tr~.
\end{align}
%


显然,所有这些形式都是等价的,如下面关于一般扰动 $\bfQ$ 的双特征值所示,
%
\begin{equation}
\bfF_\bfi\,\bfQ_\bfi\,\bfF_\bfi\tr = \eqbox{\bfQ} =  \Dt^2\,\bfB\,\bfU^c\,\bfB\tr + \Dt\,\bfC\,\bfW^c\,\bfC\tr~.
\end{equation}



%=============================================================
\subsection{完整的 IMU 示例}

我们研究了对于 IMU 的误差状态 Kalman 滤波器的构造。 
误差状态系统在方程 \eqRef{equ:efull} 中定义,并包括标称状态 $\bfx$ ,误差状态 $\delta\bfx$ ,噪声控制信号 $\bfu_m=\bfu+\tilde\bfu$ 和扰动 $\bfw$, 由以下指定,
%
\begin{equation}
\bfx = \begin{bmatrix}
\bfp \\
\bfv \\
\bfq \\
\bfa_b \\
\bfomega_b \\
\bfg
\end{bmatrix}
\quad , \quad
\delta\bfx = \begin{bmatrix}
\delta\bfp \\
\delta\bfv \\
\delta\bftheta \\
\delta\bfa_b \\
\delta\bfomega_b \\
\delta\bfg
\end{bmatrix}
\quad , \quad
\bfu_m=\begin{bmatrix}
\bfa_m \\ \bfomega_m
\end{bmatrix} 
\quad , \quad
\tilde\bfu=\begin{bmatrix}
\tilde\bfa \\ \tilde\bfomega
\end{bmatrix} 
\quad , \quad
\bfw = \begin{bmatrix}
\bfa_w \\ \bfomega_w
\end{bmatrix}
\end{equation}

在IMU的模型中,控制噪声对应于IMU测量中的加性噪声。 
扰动会影响偏差,从而产生它们的随机游走行为。 
动力学矩阵、控制矩阵和扰动矩阵是 (参见方程 \eqRef{equ:contTime}, \eqRef{equ:FullIMU_Fmat} 和 \eqRef{equ:efull}),
%
\begin{equation}
\label{equ:IMU_ABC}
\bfA = \begin{bmatrix}
  0& \bfP_\bfv&  0&  0&  0&  0\\
  0&  0& \bfV_\bftheta& \bfV_\bfa&  0& \bfV_\bfg\\
  0&  0& \Theta_\bftheta&  0& \Theta_\bfomega&  0\\
  0&  0&  0&  0&  0&  0\\
  0&  0&  0&  0&  0&  0\\
  0&  0&  0&  0&  0&  0
\end{bmatrix}
\quad,\quad
\bfB = \begin{bmatrix}
0 & 0 \\
-\bfR & 0 \\
0 & -\bfI \\
0 & 0 \\
0 & 0 \\
0 & 0 \\
\end{bmatrix}
\quad,\quad
\bfC = \begin{bmatrix}
0 & 0 \\
0 & 0 \\
0 & 0 \\
\bfI & 0 \\
0 & \bfI \\
0 & 0 \\
\end{bmatrix}
\end{equation}

在三个轴上具有相同类型的加速度计和陀螺仪三元组的 IMUs 的常规情况下,噪声和扰动是各向同性的。
它们的标准偏差以标量形式指定,如下所示
%
\begin{equation}
\sigma_{\tilde\bfa}\ [m/s^2] \quad,\quad \sigma_{\tilde\bfomega} \  [rad/s] \quad,\quad \sigma_{\bfa_w}\ [m/s^2\sqrt{s}] \quad,\quad \sigma_{\bfomega_w}\ [rad/s\sqrt{s}]
\end{equation}
%
并且它们的协方差矩阵是纯对角的,给出为
%
\begin{equation}
\bfU^c=\begin{bmatrix}
\sigma_{\tilde\bfa}^2\bfI & 0 \\ 0 & \sigma_{\tilde\bfomega}^2\bfI  
\end{bmatrix} \qquad ,\qquad
\bfW^c=\begin{bmatrix}
\sigma_{\bfa_w}^2\bfI & 0 \\ 0 & \sigma_{\bfomega_w}^2\bfI  
\end{bmatrix} ~.
\end{equation}

系统随采样测量值变化,间隔为 $\Dt$ ,接着是方程 \eqsRef{equ:discTime}{equ:discCov},其中转换矩阵 $\bfF_\bfx=\Phi$ 可通过多种方式计算 --- 见之前的附录。 

%%-------------------------------------------------------------
\subsubsection{噪声和扰动脉冲}
%\label{sec:pertImpulsesIMU}

%\TODO{Describe first the general case, and then the isotropic case}

在脉冲 $\bfi$ 形式的扰动规范的情况下,我们可以如下重新定义我们的系统,
%
\begin{equation}
\delta\bfx_{n+1} = \bfF_\bfx(\bfx_n, \bfu_m)\tdot \delta\bfx_n+\bfF_\bfi\tdot\bfi
\end{equation}
%
其中标称状态、误差状态、控制和脉冲向量定义为,
%
\begin{equation}
\bfx=\begin{bmatrix}\bfp\\\bfv\\\bfq\\\bfa_b\\\bfomega_b\\\bfg\end{bmatrix} \quad, \quad
\delta\bfx=\begin{bmatrix}\delta\bfp\\\delta\bfv\\\delta\bftheta\\\delta\bfa_b\\\delta\bfomega_b\\\delta\bfg\end{bmatrix} \quad, \quad
\bfu_m = \begin{bmatrix}
\bfa_m \\
\bfomega_m
\end{bmatrix} 
\quad,  \quad
\bfi = \begin{bmatrix}
\bfv_\bfi \\
\bftheta_\bfi \\
\bfa_\bfi \\
\bfomega_\bfi
\end{bmatrix}~,
\end{equation}
%
转换矩阵和扰动矩阵定义为,
%
\begin{equation}
\bfF_\bfx = \Phi = e^{\bfA\Dt}
\qquad,\qquad
\bfF_\bfi 
%= \begin{bmatrix}
%\bfB & \bfC
%\end{bmatrix} 
= \begin{bmatrix}
 0 &  0 & 0 & 0 \\
 \bfI &  0 & 0 & 0 \\
 0 &  \bfI & 0 & 0 \\
 0 &  0 & \bfI & 0 \\
 0 & 0 & 0 & \bfI \\
 0 &  0 & 0 & 0 
\end{bmatrix} ~,
\end{equation}
%
并且脉冲方差指定为
%
\begin{equation}
\bfi \sim \cN\{0,\bfQ_\bfi\}
\quad,\quad
\bfQ_\bfi = \begin{bmatrix}
\sigma_{\tilde\bfa}^2\Dt^2\bfI &&0&\\
& \sigma_{\tilde\bfomega}^2\Dt^2\bfI &&\\
&& \sigma_{\bfa_w}^2\Dt\bfI &\\
&0&& \sigma_{\bfomega_w}^2\Dt\bfI 
\end{bmatrix}~.
\end{equation}



$\bfF_\bfi$ 的平凡的规范可能看起来令人惊讶,特别是对于方程 \eqRef{equ:IMU_ABC} 所给出的 $\bfB$ 的规范。
所发生的情况是,误差在  $\bfQ_\bfi$ 中定义为各向同性,并因此 $-\bfR\sigma^2\bfI(-\bfR)\tr = \sigma^2\bfI$ 和 $-\bfI\sigma^2\bfI(-\bfI)\tr = \sigma^2\bfI$,导致对 $\bfF_\bfi$ 给定该表达式。
当考虑非各向同性 IMUs 时,这是不可能的,这里一个适当的 Jacobian 矩阵 $\bfF_\bfi=\begin{bmatrix}
\bfB & \bfC
\end{bmatrix}$ 应当与适当的 $\bfQ_\bfi$ 规范一起使用。

%\begin{bmatrix}
% 0 &  0 & 0 & 0 \\
% \bfI &  0 & 0 & 0 \\
% 0 &  \bfI & 0 & 0 \\
% 0 &  0 & \bfI & 0 \\
% 0 & 0 & 0 & \bfI \\
% 0 &  0 & 0 & 0 
%\end{bmatrix}

\bigskip
我们当然可以用全态扰动脉冲,
%
\begin{equation}
\delta\bfx_{n+1} = \bfF_\bfx(\bfx_n, \bfu_m)\tdot \delta\bfx_n+\bfi
\end{equation}
%
其中
%
\begin{equation}
\bfi = \begin{bmatrix}
0 \\
\bfv_\bfi \\
\bftheta_\bfi \\
\bfa_\bfi \\
\bfomega_\bfi \\
0
\end{bmatrix}
\quad,\quad
\bfi \sim \cN\{0,\bfQ\}
\quad,\quad
\bfQ = \begin{bmatrix}
0 & \\
& \sigma_{\tilde\bfa}^2\Dt^2\bfI &&&0&\\
&& \sigma_{\tilde\bfomega}^2\Dt^2\bfI &&\\
&&& \sigma_{\bfa_w}^2\Dt\bfI &\\
&0&&& \sigma_{\bfomega_w}^2\Dt\bfI \\
&&&&&0
\end{bmatrix}~.
\end{equation}



\bigskip
\bigskip
\bigskip

再见。

